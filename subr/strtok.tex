\clearpage
\phantomsection
\addcontentsline{toc}{subsection}{strtok}
\label{subr:strtok}
\subsection*{strtok: string tokenizing subroutine}

\subsubsection*{Calling convention}

\begin{description}
\item[\registerop{rd}] pointer to the next token or NULL
\end{description}

\subsubsection*{Description}

The \subroutine{strtok} subroutine returns a sequential set of tokens
from a string passed as its first argument, based on a separator
passed as its second.  Once the string has been exhausted NULL is
returned.  In order to find subsequent tokens NULL is passed as the
first argument.  See this operating system's \subroutine{strtok}
manual page (strtok(3)) for an example.

\subsubsection*{Pseudocode}

\begin{verbatim}
\end{verbatim}

\subsubsection*{Constraints}

\subsubsection*{Failure modes}

This subroutine has no run-time failure modes beyond its constraints.
