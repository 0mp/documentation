\clearpage
\phantomsection
\addcontentsline{toc}{subsection}{hotnll}
\label{subr:htonll}
\subsection*{hotnll: convert a long long (64 bit) value from host to  network byte order}

\subsubsection*{Calling convention}

\begin{description}
\item[\registerop{rd}] A 64 bit value in network byte order
\end{description}

\subsubsection*{Description}

The \subroutine{htonll} routine takes a 64 bit value as its only
argument and returns that value in network byte order.

\subsubsection*{Pseudocode}

\begin{verbatim}
\end{verbatim}

\subsubsection*{Constraints}

\subsubsection*{Failure modes}

This subroutine has no run-time failure modes beyond its constraints.
