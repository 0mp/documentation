\clearpage
\phantomsection
\addcontentsline{toc}{subsection}{inet-ntop}
\label{subr:inet-ntop}
\subsection*{inet\_ntop: convert an arbitrary Internet address to a string}

\subsubsection*{Calling convention}

\begin{description}
\item[\registerop{rd}] Internet address as a string
\end{description}

\subsubsection*{Description}

The \subroutine{inet_ntop} subroutine takes either a 128 bit, IPv6,
address or a 32 bit, IPv4 address, and converts it to a string
suitable for humans.  The type of address supplied is indeicated by
the second argument, which must either be \verb|AF_INET| or
\verb|AF_INET6|.

\subsubsection*{Pseudocode}

\begin{verbatim}
\end{verbatim}

\subsubsection*{Constraints}

\subsubsection*{Failure modes}

This subroutine has no run-time failure modes beyond its constraints.
