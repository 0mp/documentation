%%%% University of Cambridge tech-report formatting; enable when producing
%%%% tech-report versions of these documents; otherwise, disable.
%\documentclass[12pt,twoside,openright,a4paper]{report}
%\setlength{\oddsidemargin}{-0.4mm} % 25 mm left margin
%\setlength{\evensidemargin}{\oddsidemargin}
%\setlength{\textwidth}{160mm}      % 25 mm right margin
%\setlength{\topmargin}{-5.4mm}     % 20 mm top margin
%\setlength{\headheight}{5mm}
%\setlength{\headsep}{5mm}
%\setlength{\footskip}{10mm}
%\setlength{\textheight}{237mm}     % 20 mm bottom margin
%%%% .. or regular document
\documentclass[12pt,letterpaper,twoside,openright,fleqn]{report}
%%%% End of tech-report vs. regular
%%%%

%\renewcommand{\baselinestretch}{2} % double space for editors 
\usepackage{fullpage}
\usepackage{graphicx}
\usepackage{marginnote}
\usepackage{appendix}
\usepackage{booktabs}
\usepackage{bytefield}
\usepackage{color}
\definecolor{lightgray}{gray}{0.8}
\usepackage{times}
\usepackage{algpseudocode}
\newcommand{\note}[2]{{\color{blue}[ Note: #1 - #2]}}
\usepackage{listings}
\usepackage{setspace}
\usepackage{amsmath}
% Must be included later than setspace, otherwise all footnote hyperlinks
% point to the title page.
\usepackage{hyperref}
\definecolor{CodeColour}{rgb}{0.9,0.9,0.9} %Light grey
\lstset{basicstyle=\small\ttfamily,
        stringstyle=\textit, %italic strings
        keywordstyle=\color{red}\textbf, %Bold keywords
        commentstyle=\color{blue},
        breaklines=true, % Wrap long lines
        numbers=left, % Line numbers on the left
        frame=l, %Border on the left
        framerule=0.8pt, % Thick border
        backgroundcolor=\color{CodeColour}, %Coloured code listings
        numberstyle={\small \oldstylenums},  %tiny, old style line numbers
		%stepnumber=5, % Number every fifth line
        numbersep=5pt, % Five points between the line numbers and the text
        tabsize=4
}
\lstdefinelanguage{llvm}
{
	morekeywords={private, constant, i8, i32, define, icmp, label, i64, call, void, ret, getelementptr, br, load, align, nounwind},
	morekeywords={addrspace, inttoptr, ptrtoint, tail},
	morecomment=[l];
}%

\lstnewenvironment{ccodelisting}{\lstset{language=C}}{}
\lstnewenvironment{llvmlisting}{\lstset{language={llvm}}}{}
\newcommand{\ccode}[1]{\lstinline[backgroundcolor=\color{white},language=C]|#1|}
\newcommand{\llvmir}[1]{\lstinline[backgroundcolor=\color{white},language={llvm}]|#1|}
\newcommand{\asm}[1]{\lstinline[backgroundcolor=\color{white},language={}]|#1|}
\lstnewenvironment{asmcode}{\lstset{language=}}{}
\newcommand{\regname}[1]{{\small\ttfamily\$#1}}

\hyphenation{CADETS}

\reversemarginpar
\setlength{\marginparwidth}{1.2in}
\let\oldmarginpar\marginpar
\renewcommand\marginpar[1]{\-\oldmarginpar[\raggedright\footnotesize #1]%
{\raggedright\footnotesize #1}}

\newcommand{\pathname}[1]{{\texttt{\detokenize{#1}}}}
\newcommand{\literal}[1]{{\texttt{\detokenize{#1}}}}
\newcommand{\function}[1]{{\texttt{\detokenize{#1}}}}
\newcommand{\instruction}[1]{{\texttt{#1}}}
\newcommand{\register}[1]{{\texttt{\%#1}}}
\newcommand{\registerop}[1]{{\texttt{\textbf{#1}}}}
\newcommand{\nregs}{{\texttt{\textbf{NREGS}}}}
\newcommand{\subroutine}[1]{{\texttt{\detokenize{#1}}}}
\newcommand{\struct}[1]{{\texttt{\detokenize{#1}}}}
