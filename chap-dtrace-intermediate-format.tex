
\section{DTrace Instruction Version}

This specification describes the DTrace Intermediate Format version 2, as
shipped in Solaris \textbf{XXXRW}, FreeBSD \textbf{XXXRW}, and Mac OS X
\textbf{XXXRW}.

\section{The DIF Interpreter}

\textbf{XXXRW: Something about instructions}

\textbf{XXXRW: Something about branches}

\textbf{XXXRW: Something about calling subroutines}

\textbf{XXXRW: Something about returning}

\textbf{XXXRW: Something about DIF registers, \nregs{}}

\textbf{XXXRW: Something about the zero register}

\textbf{XXXRW: Something about implied status bits}

\textbf{XXXRW: Something about scratch memory (alloca, user copyin)}

\textbf{XXXRW: Something about the tuple stack}

\textbf{XXXRW: Something about kernel memory access}

\textbf{XXXRW: Something about user memory access}

\textbf{XXXRW: Something about ``by reference''}

\textbf{XXXRW: Something about the integer table}

\textbf{XXXRW: Something about the string table}

\textbf{XXXRW: Something about local variables}

\textbf{XXXRW: Something about thread-local variables}

\textbf{XXXRW: Something about global variables}

\textbf{XXXRW: Something about scalars}

\textbf{XXXRW: Something about arrays}

\textbf{XXXRW: Something about aggregations}

\textbf{XXXRW: Something about exceptions}

\section{Instruction Format}

Each instruction consists of a 32-bit, native-endian integer whose most
significant 8 bits contain an opcode allowing the remainder of the instruction
to be decoded.
To ease parsing, three major formats (R, B, and W) are used for all DTrace
instructions, capturing different types of operations: register-to-register
instructions accepting zero or more register operands; branch instructions
accepting a target label as a single operand; and wide-immediate instructions
that accept a 16-bit immediate used to capture both small constant values and
also indices into various tables.

\subsection{Register Format (R-Format)}

This format accepts zero or more register operands, supporting instructions
that include arithmetic and boolean operations, comparison and test
operations, load and store operations, tuple-stack operations, and the no-op
instruction.

\begin{center}
\begin{bytefield}[endianness=big,bitformatting=\scriptsize]{32}
\bitheader{0,7,8,15,16,23,24,31}\\
\bitbox{8}{op}
\bitbox{8}{r1}
\bitbox{8}{r2}
\bitbox{8}{rd}
\end{bytefield}
\end{center}

\begin{description}
\item[op] Mandatory 8-bit operation identifier
\item[r1, r2] Optional source registers providing input values to the
  operation
\item[rd] Optional destination register acting as the destination of the
  operation
\end{description}

\subsection{Branch Format (B-Format)}

This format accepts a single 24-bit integer operand identifying the label that
is the branch target.
It is used solely for the \textbf{BRANCH} instruction.

\begin{center}
\begin{bytefield}[endianness=big,bitformatting=\scriptsize]{32}
\bitheader{0,23,24,31}\\
\bitbox{8}{op}
\bitbox{24}{label}
\end{bytefield}
\end{center}

\begin{description}
\item[op] Mandatory 8-bit operation identifier
\item[label] Mandatory 24-bit integer label
\end{description}

\subsection{Wide-Immediate Format (W-Format)}

This format accepts an 8-bit register and 16-bit integer argument (frequently
an index).
It is used for a range of instructions including those to load values from
integer and string constant tables, and to load and store variable values.

\begin{center}
\begin{bytefield}[endianness=big,bitformatting=\scriptsize]{32}
\bitheader{0,7,8,23,24,31}\\
\bitbox{8}{op}
\bitbox{16}{idx}
\bitbox{8}{rs\textbar rd}
\end{bytefield}
\end{center}

\begin{description}
\item[op] Mandatory 8-bit operation identifier
\item[idx] Mandatory 16-bit integer index
\item[rs\textbar rd] Optional 8-bit register acting as the source or
  destination of the operation
\end{description}

\section{Instructions}

Chapter~\ref{chap:dtrace-instruction-reference} contains a comprehensive
description of DTrace's instructions.

\section{Subroutines}

Chapter~\ref{chap:dtrace-subroutines} contains a comprehensive description of
DTrace's built-in subroutines.
