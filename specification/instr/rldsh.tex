\clearpage
\phantomsection
\addcontentsline{toc}{subsection}{RLDSH}
\label{insn:rldsh}
\subsection*{RLDSH: restricted load of a signed 16 bit quantity}

\subsubsection*{Format}

\textrm{RLDSH \%rd, \%r1, \%r2}

\begin{center}
\begin{bytefield}[endianness=big,bitformatting=\scriptsize]{32}
\bitheader{0,7,8,15,16,23,24,31} \\
\bitbox{8}{0x48}
\bitbox{8}{r1}
\bitbox{8}{r2}
\bitbox{8}{rd}
\end{bytefield}
\end{center}

\subsubsection*{Description}

The \instruction{rldsh} instruction performs a privilege check on the
memory it is about to read from before loading a signed, 16 bit,
quantity into \registerop{rd}, indexed by \registerop{r1}.

\subsubsection*{Pseudocode}

\begin{verbatim}
%rd = mem[%r1]
\end{verbatim}

\subsubsection*{Load-time constraints}
The registers \registerop{r1} and \registerop{rd} must be valid registers,
\registerop{r2} must be \registerop{r0} and \registerop{rd} must not be
\registerop{r0}.

\subsubsection*{Failure modes}

\textbf{XXXDS: This depends on dtrace\_canload(). We have to enumerate these.}
