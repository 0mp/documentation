\clearpage
\phantomsection
\addcontentsline{toc}{subsection}{SDIV}
\label{insn:sdiv}
\subsection*{SDIV: signed division}

\subsubsection*{Format}

\textrm{SDIV \%rd, \%r1, \%r2}

\begin{center}
\begin{bytefield}[endianness=big,bitformatting=\scriptsize]{32}
\bitheader{0,7,8,15,16,23,24,31} \\
\bitbox{8}{0x09}
\bitbox{8}{r1}
\bitbox{8}{r2}
\bitbox{8}{rd}
\end{bytefield}
\end{center}

\subsubsection*{Description}

The \instruction{sdiv} instruction divides the value contained in
\registerop{r2} into that contained in \registerop{r1} placing the
results into \registerop{rd}.  The values in both \registerop{r1} and
\registerop{r2} are first promoted to signed, 64 bit values, before
the division operation is carried out.

\subsubsection*{Pseudocode}

\begin{verbatim}
%rd = (int64_t)%r1 / (inst64_t)%r2
\end{verbatim}

\subsubsection*{Load-time constraints}
The registers \registerop{r1}, \registerop{r2} and \registerop{rd} must be
valid registers and \registerop{rd} must not be \registerop{r0}.

\subsubsection*{Failure modes}

This instruction has no run-time failure modes beyond its constraints.
