\clearpage
\phantomsection
\addcontentsline{toc}{subsection}{BA}
\label{insn:ba}
\subsection*{BA: branch absolute}

\subsubsection*{Format}

\textrm{BA label}

\begin{center}
\begin{bytefield}[endianness=big,bitformatting=\scriptsize]{32}
\bitheader{0,23,24,31} \\
\bitbox{8}{0x11}
\bitbox{24}{label}
\end{bytefield}
\end{center}

\subsubsection*{Description}

The \instruction{ba} instruction branches to the label indicated by
setting the Program Counter (\verb+pc+) to the instruction indicated
at the \verb+label+.

\subsubsection*{Pseudocode}

\begin{verbatim}
%pc = label
\end{verbatim}

\subsubsection*{Load-time constraints}
DIF does not allow backwards branches, therefore \verb+label+ must be
greater than \verb+pc+. Moreover, \verb+label+ must not go past the last
address of the current DIF object.

\subsubsection*{Failure modes}

This instruction has no run-time failure modes beyond its constraints.
