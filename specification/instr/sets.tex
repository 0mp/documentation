\clearpage
\phantomsection
\addcontentsline{toc}{subsection}{SETS}
\label{insn:sets}
\subsection*{SETS: retrieve string from the string table}

\subsubsection*{Format}

\textrm{SETS \%rd, strindex}

\begin{center}
\begin{bytefield}[endianness=big,bitformatting=\scriptsize]{32}
\bitheader{0,7,8,15,16,31} \\
\bitbox{8}{0x26}
\bitbox{8}{rd}
\bitbox{16}{index}
\end{bytefield}
\end{center}

\subsubsection*{Description}

The \instruction{sets} instruction looks up a string stored in the DIF
string table and places a pointer to the value into \registerop{rd}. This
instruction performs no bounds checking.
\subsubsection*{Pseudocode}

\begin{verbatim}
%rd = strtab + index
\end{verbatim}

\subsubsection*{Failure modes}

This instruction has no run-time failure modes beyond its constraints.
