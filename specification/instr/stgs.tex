\clearpage
\phantomsection
\addcontentsline{toc}{subsection}{STGS}
\label{insn:stgs}
\subsection*{STGS: store a value into a variable}

\subsubsection*{Format}

\textrm{STGS \%rd, \%r1, \%r2}

\begin{center}
\begin{bytefield}[endianness=big,bitformatting=\scriptsize]{32}
\bitheader{0,7,8,23,24,31} \\
\bitbox{8}{0x2A}
\bitbox{16}{var}
\bitbox{8}{rd}
\end{bytefield}
\end{center}

\subsubsection*{Description}

Similar to \instruction{ldgs}, the instruction \instruction{stgs} operates
exclusively on scalar variables and can not contain indices. However, the
instruction may allow loading of data by reference using the DIF\_TF\_BYREF
flag, which allows loading of data bounded by the limits found in the
dtrace\_vcanload() function. Unlike \instruction{ldgs}, \instruction{stgs} can
not store to pre-defined variables in DTrace, and instead allows access only to
user defined variables. The variable is accessed by the \registerop{var} field
and is required to be large or equal to DIF\_VAR\_OTHER\_UBASE. The result of
this operation is stored in the \registerop{rd} register.

\subsubsection*{Pseudocode}

\begin{verbatim}
assert(var >= DIF_VAR_OTHER_UBASE)
var -= DIF_VAR_OTHER_UBASE
if (flags & DIF_TF_BYREF)
    var = copyin(%rd)
else
    var = %rd
\end{verbatim}

\subsubsection*{Failure modes}

This instruction will fail if the supplied value in the \registerop{var} field
is less than DIF\_VAR\_OTHER\_UBASE.
