\clearpage
\phantomsection
\addcontentsline{toc}{subsection}{BNE}
\label{insn:bne}
\subsection*{BNE: branch not equal}

\subsubsection*{Format}

\textrm{BNE label}

\begin{center}
\begin{bytefield}[endianness=big,bitformatting=\scriptsize]{32}
\bitheader{0,23,24,31} \\
\bitbox{8}{0x13}
\bitbox{24}{label}
\end{bytefield}
\end{center}

\subsubsection*{Description}

The \instruction{bne} instruction sets the PC to a new label if, and
only if the result of the last \instruction{cmp} restulted in the zero
bit (cc\_z) being cleared, or set to 0.

\subsubsection*{Pseudocode}

\begin{verbatim}
if (cc_z == 0)
	%pc = label
\end{verbatim}

\subsubsection*{Failure modes}

This instruction has no run-time failure modes beyond its constraints.
