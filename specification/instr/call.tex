\clearpage
\phantomsection
\addcontentsline{toc}{subsection}{CALL}
\label{insn:call}
\subsection*{CALL: subroutine call}

\subsubsection*{Format}

\textrm{CALL \%rd, \%r1, \%r2}

\begin{center}
\begin{bytefield}[endianness=big,bitformatting=\scriptsize]{32}
\bitheader{0,7,8,23,24,31} \\
\bitbox{8}{0x2F}
\bitbox{16}{subroutine}
\bitbox{8}{rd}
\end{bytefield}
\end{center}

\subsubsection*{Description}

The \instruction{call} instruction executes a known DTrace subroutine,
such as copyinstr(), copyout() etc. and returns any value into
\registerop{rd}.  Valid subroutines are documented in
\ref{sec:subroutines}.

\subsubsection*{Pseudocode}

\begin{verbatim}
%rd = subr()
\end{verbatim}

\subsubsection*{Failure modes}

This instruction has no run-time failure modes beyond its constraints.
