\clearpage
\phantomsection
\addcontentsline{toc}{subsection}{UDIV}
\label{insn:udiv}
\subsection*{UDIV: unsigned division}

\subsubsection*{Format}

\textrm{UDIV \%rd, \%r1, \%r2}

\begin{center}
\begin{bytefield}[endianness=big,bitformatting=\scriptsize]{32}
\bitheader{0,7,8,15,16,23,24,31} \\
\bitbox{8}{0x0A}
\bitbox{8}{r1}
\bitbox{8}{r2}
\bitbox{8}{rd}
\end{bytefield}
\end{center}

\subsubsection*{Description}

The \instruction{udiv} instruction divides the value contained in
\registerop{r2} into that contained in \registerop{r1} placing the
results into \registerop{rd}.

\subsubsection*{Pseudocode}

\begin{verbatim}
%rd = %r1 / %r2
\end{verbatim}

\subsubsection*{Failure modes}

This instruction has no run-time failure modes beyond its constraints.
