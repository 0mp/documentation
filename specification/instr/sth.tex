\clearpage
\phantomsection
\addcontentsline{toc}{subsection}{STH}
\label{insn:sth}
\subsection*{STH: store a 16 bit value into memory}

\subsubsection*{Format}

\textrm{STH \%rd, \%r1}

\begin{center}
\begin{bytefield}[endianness=big,bitformatting=\scriptsize]{32}
\bitheader{0,7,8,15,16,23,24,31} \\
\bitbox{8}{0x3D}
\bitbox{8}{r1}
\bitbox{8}{r2}
\bitbox{8}{rd}
\end{bytefield}
\end{center}

\subsubsection*{Description}

The \instruction{sth} instruction takes a 16 bit value from \registerop{r1}
and stores it into the memory location pointed to by \registerop{rd}.
\subsubsection*{Pseudocode}

\begin{verbatim}
mem[%rd] = %r1
\end{verbatim}

\subsubsection*{Load-time constraints}
The registers \registerop{r1} and \registerop{rd} must be valid registers,
\registerop{r2} must be \registerop{r0} and \registerop{rd} must not be
\registerop{r0}.

\subsubsection*{Failure modes}

\textbf{XXXDS: This depends on dtrace\_canstore(). We have to enumerate these.}
