\clearpage
\phantomsection
\addcontentsline{toc}{subsection}{LDUW}
\label{insn:lduw}
\subsection*{LDUW: load an unsigned 32 bit value}

\subsubsection*{Format}

\textrm{LDSB \%rd, \%r1}

\begin{center}
\begin{bytefield}[endianness=big,bitformatting=\scriptsize]{32}
\bitheader{0,7,8,15,16,23,24,31} \\
\bitbox{8}{0x21}
\bitbox{8}{r1}
\bitbox{8}{r2}
\bitbox{8}{rd}
\end{bytefield}
\end{center}

\subsubsection*{Description}

The \instruction{lduw} instruction loads a 32-bit value pointed to by
\registerop{r1} into \registerop{rd}, the results register. This is an
\textbf{unsigned} instruction and will not perform sign extension in any case.

\subsubsection*{Pseudocode}

\begin{verbatim}
%rd = %r1
\end{verbatim}

\subsubsection*{Constraints}

\subsubsection*{Failure modes}

This instruction has no run-time failure modes beyond its constraints.
