\clearpage
\phantomsection
\addcontentsline{toc}{subsection}{BL}
\label{insn:bl}
\subsection*{BL: branch less than}

\subsubsection*{Format}

\textrm{BL label}

\begin{center}
\begin{bytefield}[endianness=big,bitformatting=\scriptsize]{32}
\bitheader{0,23,24,31} \\
\bitbox{8}{0x18}
\bitbox{24}{label}
\end{bytefield}
\end{center}

\subsubsection*{Description}

The \instruction{bl} instruction jumps to the specified label if and
only if the result of the previous comparison instruction indicated
that the value in \registerop{r1} was strictly less than the value in
\registerop{r2}.
\subsubsection*{Pseudocode}

\begin{verbatim}
if (cc_n ^ cc_v)
	pc = label
\end{verbatim}

\subsubsection*{Load-time constraints}
DIF does not allow backwards branches, therefore \verb+label+ must be
greater than \verb+pc+. Moreover, \verb+label+ must not go past the last
address of the current DIF object.

\subsubsection*{Failure modes}

This instruction has no run-time failure modes beyond its constraints.
