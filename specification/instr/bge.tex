\clearpage
\phantomsection
\addcontentsline{toc}{subsection}{BGE}
\label{insn:bge}
\subsection*{BGE: branch greater than or equal to}

\subsubsection*{Format}

\textrm{BGE label}

\begin{center}
\begin{bytefield}[endianness=big,bitformatting=\scriptsize]{32}
\bitheader{0,23,24,31} \\
\bitbox{8}{0x16}
\bitbox{24}{label}
\end{bytefield}
\end{center}

\subsubsection*{Description}

The \instruction{bge} instruction jumps to the supplied label if and
only if the result of the previous comparison indicates that the
value in register \registerop{r1} was greater than or equal to the
value in \registerop{r2}.

\subsubsection*{Pseudocode}

\begin{verbatim}
if ((cc_n ^ cc_v) == 0)
	pc = label;
\end{verbatim}

\subsubsection*{Failure modes}

This instruction has no run-time failure modes beyond its constraints.
