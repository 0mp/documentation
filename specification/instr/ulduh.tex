\clearpage
\phantomsection
\addcontentsline{toc}{subsection}{ULDUH}
\label{insn:ulduh}
\subsection*{ULDUH: load an unsigned 16 bit quantity from user space}

\subsubsection*{Format}

\textrm{ULDUH \%rd, \%r1, \%r2}

\begin{center}
\begin{bytefield}[endianness=big,bitformatting=\scriptsize]{32}
\bitheader{0,7,8,15,16,23,24,31} \\
\bitbox{8}{0x44}
\bitbox{8}{r1}
\bitbox{8}{r2}
\bitbox{8}{rd}
\end{bytefield}
\end{center}

\subsubsection*{Description}

The \instruction{ulduh} instruction loads an unsigned, 16 bit,
quantity from memory in a user space process into the \registerop{rd}
register, indexed by \registerop{r1}.

\subsubsection*{Pseudocode}

\begin{verbatim}
%rd = umem[r1]
\end{verbatim}

\subsubsection*{Constraints}

\subsubsection*{Failure modes}

This instruction has no run-time failure modes beyond its constraints.
