\clearpage
\phantomsection
\addcontentsline{toc}{subsection}{PUSHTR}
\label{insn:pushtr}
\subsection*{PUSHTR: push a reference onto the stack}

\subsubsection*{Format}

\textrm{PUSHTR type, \%r2, \%rs}

\begin{center}
\begin{bytefield}[endianness=big,bitformatting=\scriptsize]{32}
\bitheader{0,7,8,15,16,23,24,31} \\
\bitbox{8}{0x30}
\bitbox{8}{type}
\bitbox{8}{r2}
\bitbox{8}{rs}
\end{bytefield}
\end{center}

\subsubsection*{Description}

The \instruction{pushtr} instruction pushes a reference, contained in
the \registerop{rs} register onto the stack.  The length is stored for
a string along with the value.  For a numeric value the size of that
value is stored.

\subsubsection*{Pseudocode}

\begin{verbatim}
value = %rs
if type is string:
    size = strlen(value)
else:
    size = %r2

stack[++index].size = size
stack[index].value = value
\end{verbatim}

\subsubsection*{Failure modes}

This instruction has no run-time failure modes beyond its constraints.
