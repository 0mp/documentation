\clearpage
\phantomsection
\addcontentsline{toc}{subsection}{LDGA}
\label{insn:ldga}
\subsection*{LDGA: load a DTrace built-in variable}

\subsubsection*{Format}

\textrm{LDGA \%rd, var, \%r2}

\begin{center}
\begin{bytefield}[endianness=big,bitformatting=\scriptsize]{32}
\bitheader{0,7,8,15,16,23,24,31}\\
\bitbox{8}{op}
\bitbox{8}{var}
\bitbox{8}{r2}
\bitbox{8}{rd}
\end{bytefield}
\end{center}

\subsubsection*{Description}

The \instruction{ldga} instruction looks up the value of a DTrace
built-in variable based on the value in \registerop{var} with an
optional array index in the register \register{r2}. \\

Unlike the \instruction{ldgs}, the variable identifier is 8 bits long, and
the other 8 bits are used to identify the register which contains the index of
the array.

\subsubsection*{Pseudocode}

\begin{verbatim}
index = %r2
%rd = var[index]
\end{verbatim}

\subsubsection*{Failure modes}

This instruction has no run-time failure modes beyond its constraints.
