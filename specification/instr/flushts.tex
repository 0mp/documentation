\clearpage
\phantomsection
\addcontentsline{toc}{subsection}{FLUSHTS}
\label{insn:flushts}
\subsection*{FLUSHTS: flush the stack}

\subsubsection*{Format}

\textrm{FLUSHTS}

\begin{center}
\begin{bytefield}[endianness=big,bitformatting=\scriptsize]{32}
\bitheader{0,7,8,31} \\
\bitbox{8}{0x33}
\bitbox{24}{0}
\end{bytefield}
\end{center}

\subsubsection*{Description}

The \instruction{flushts} instruction flushes the stack, by resetting
the stack pointer to 0.
\subsubsection*{Pseudocode}

\begin{verbatim}
%sp = 0;
\end{verbatim}

\subsubsection*{Constraints}

\subsubsection*{Failure modes}

This instruction has no run-time failure modes beyond its constraints.
