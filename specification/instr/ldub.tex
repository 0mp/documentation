\clearpage
\phantomsection
\addcontentsline{toc}{subsection}{LDUB}
\label{insn:ldub}
\subsection*{LDUB: load an unsigned 8 bit value}

\subsubsection*{Format}

\textrm{LDUB \%rd, \%r1}

\begin{center}
\begin{bytefield}[endianness=big,bitformatting=\scriptsize]{32}
\bitheader{0,7,8,15,16,23,24,31} \\
\bitbox{8}{0x1F}
\bitbox{8}{r1}
\bitbox{8}{r2}
\bitbox{8}{rd}
\end{bytefield}
\end{center}

\subsubsection*{Description}

The \instruction{ldub} instruction loads the value pointed to by \registerop{r1}
into \registerop{rd}, the results register. This is an \textbf{unsigned}
instruction and will not perform sign extension in any case.

\subsubsection*{Pseudocode}

\begin{verbatim}
%rd = %r1
\end{verbatim}

\subsubsection*{Failure modes}

This instruction has no run-time failure modes beyond its constraints.
