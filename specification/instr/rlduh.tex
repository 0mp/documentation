\clearpage
\phantomsection
\addcontentsline{toc}{subsection}{RLDUH}
\label{insn:rlduh}
\subsection*{RLDUH: restricted load of an unsigned 16 bit quantity}

\subsubsection*{Format}

\textrm{RLDUH \%rd, \%r1, \%r2}

\begin{center}
\begin{bytefield}[endianness=big,bitformatting=\scriptsize]{32}
\bitheader{0,7,8,15,16,23,24,31} \\
\bitbox{8}{0x4B}
\bitbox{8}{r1}
\bitbox{8}{r2}
\bitbox{8}{rd}
\end{bytefield}
\end{center}

\subsubsection*{Description}

The \instruction{rlduh} instruction performs a privilege check on the
memory it is about to read from before loading an unsigned, 16 bit,
quantity into \registerop{rd}, indexed by \registerop{r1}.

\subsubsection*{Pseudocode}

\begin{verbatim}
%rd = mem[%r1]
\end{verbatim}

\subsubsection*{Failure modes}

This instruction has no run-time failure modes beyond its constraints.
