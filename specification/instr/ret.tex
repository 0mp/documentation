\clearpage
\phantomsection
\addcontentsline{toc}{subsection}{RET}
\label{insn:ret}
\subsection*{RET: return}

\subsubsection*{Format}

\textrm{RET \%rd}

\begin{center}
\begin{bytefield}[endianness=big,bitformatting=\scriptsize]{32}
\bitheader{0,7,8,31} \\
\bitbox{8}{0x23}
\bitbox{24}{rd}
\end{bytefield}
\end{center}

\subsubsection*{Description}

The \instruction{ret} instruction returns the value in \registerop{rd}. This
instruction also sets the \register{pc} register to the length of the DIFO text
section.

\subsubsection*{Pseudocode}
\begin{verbatim}
%pc = textlen
\end{verbatim}

\subsubsection*{Load-time constraints}
The registers \registerop{r1} and \registerop{r2} must be \registerop{r0} and
\registerop{rd} must be a valid register.

\subsubsection*{Failure modes}

This instruction has no run-time failure modes beyond its constraints.
