\clearpage
\phantomsection
\addcontentsline{toc}{subsection}{BE}
\label{insn:be}
\subsection*{BE: branch equal}

\subsubsection*{Format}

\textrm{BE label}

\begin{center}
\begin{bytefield}[endianness=big,bitformatting=\scriptsize]{32}
\bitheader{0,23,24,31} \\
\bitbox{8}{0x12}
\bitbox{24}{label}
\end{bytefield}
\end{center}

\subsubsection*{Description}

The \instruction{be} instruction sets the PC to a new label if, and
only if the result of the last \instruction{cmp} or \instruction{tst}
set the zero bit (cc\_z) to a value other than 0.

\subsubsection*{Pseudocode}

\begin{verbatim}
if (cc_z)
	%pc = label
\end{verbatim}

\subsubsection*{Load-time constraints}
DIF does not allow backwards branches, therefore \verb+label+ must be
greater than \verb+pc+. Moreover, \verb+label+ must not go past the last
address of the current DIF object.

\subsubsection*{Failure modes}

This instruction has no run-time failure modes beyond its constraints.
