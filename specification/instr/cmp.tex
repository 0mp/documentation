\clearpage
\phantomsection
\addcontentsline{toc}{subsection}{CMP}
\label{insn:cmp}
\subsection*{CMP: compare two values}

\subsubsection*{Format}

\textrm{CMP \%rd, \%r1, \%r2}

\begin{center}
\begin{bytefield}[endianness=big,bitformatting=\scriptsize]{32}
\bitheader{0,7,8,15,16,23,24,31} \\
\bitbox{8}{0x0E}
\bitbox{8}{r1}
\bitbox{8}{r2}
\bitbox{8}{rd}
\end{bytefield}
\end{center}

\subsubsection*{Description}

The \instruction{cmp} instruction compares the values in
\registerop{r1} and \registerop{r2}, via subtraction, and sets the
various comparison bits based on the results.  The comparison bits,
shown in Table\ref{tbl:cmp-vars}, are used by the branch instructions
to make decisions about where the program will execute next.

\subsubsection*{Pseudocode}

\begin{verbatim}
cc_r = %r1 - %r2;
cc_n = cc_r < 0;
cc_z = cc_r == 0;
cc_v = 0;
cc_c = %r1 < %r2;
\end{verbatim}

\subsubsection*{Failure modes}

This instruction has no run-time failure modes beyond its constraints.
