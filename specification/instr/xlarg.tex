\clearpage
\phantomsection
\addcontentsline{toc}{subsection}{XLARG}
\label{insn:xlarg}
\subsection*{XLARG: translation argument}

\subsubsection*{Format}

\textrm{XLARG \%rd, \%r1, \%r2}

\begin{center}
\begin{bytefield}[endianness=big,bitformatting=\scriptsize]{32}
\bitheader{0,7,8,15,16,23,24,31} \\
\bitbox{8}{0x4F}
\bitbox{8}{r1}
\bitbox{8}{r2}
\bitbox{8}{rd}
\end{bytefield}
\end{center}

\subsubsection*{Description}

The \instruction{xlarg} instruction translates a single argument from
a structure and returns the tranlsated value in \registerop{rd}.

\emph{NOTE:} This instruction is not used by the kernel as all
translations are handled in user space.

\subsubsection*{Failure modes}

This instruction has no run-time failure modes beyond its constraints.
