\clearpage
\phantomsection
\addcontentsline{toc}{subsection}{LDGS}
\label{insn:ldgs}
\subsection*{LDGS: Load a user defined variable}

\subsubsection*{Format}

\textrm{LDGS \%rd, \%r1, \%r2}

\begin{center}
\begin{bytefield}[endianness=big,bitformatting=\scriptsize]{32}
\bitheader{0,7,8,23,24,31} \\
\bitbox{8}{0x29}
\bitbox{16}{var}
\bitbox{8}{rd}
\end{bytefield}
\end{center}

\subsubsection*{Description}

The \instruction{ldgs} instruction has two modes of operation and is intended to
be used only for scalar values. The first mode of operation is when the value
provided in \registerop{var} is less than DIF\_VAR\_OTHER\_UBASE. This will
cause DTrace to look up a pre-defined scalar variable such as curthread, while
the second mode of operation will result in looking up a user defined variable
in a DIF program. The result of this instruction will be put into the register
\registerop{rd}. \\

Unlike the \instruction{ldga} instruction, the \registerop{var} field is
16 bits long, as opposed to 8 bits due to the fact that the variable that is
being loaded is a scalar and does not require indexing operations.
\subsubsection*{Pseudocode}

\begin{verbatim}
%rd = var
\end{verbatim}

\subsubsection*{Failure modes}

This instruction has no run-time failure modes beyond its constraints.
