\clearpage
\phantomsection
\addcontentsline{toc}{subsection}{STB}
\label{insn:stb}
\subsection*{STB: store a byte into memory}

\subsubsection*{Format}

\textrm{STB \%rd, \%r1}

\begin{center}
\begin{bytefield}[endianness=big,bitformatting=\scriptsize]{32}
\bitheader{0,7,8,15,16,23,24,31} \\
\bitbox{8}{0x3C}
\bitbox{8}{r1}
\bitbox{8}{r2}
\bitbox{8}{rd}
\end{bytefield}
\end{center}

\subsubsection*{Description}

The \instruction{stb} instruction takes a byte from \registerop{r1}
and stores it into the memory location pointed to by \registerop{rd}.
\subsubsection*{Pseudocode}

\begin{verbatim}
mem[%rd] = %r1
\end{verbatim}

\subsubsection*{Failure modes}

This instruction has no run-time failure modes beyond its constraints.
