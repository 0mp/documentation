The D language is a language inspired by the AWK programming language
\cite{Aho:1987:APL:29361} and the C programming language
\cite{Kernighan:1988}. In this chapter, we give a formal definition of
the D programming language that is deployed in OpenDTrace, as well as
elaborate on its properties in multithreaded environments.

%%% Local Variables:
%%% mode: latex
%%% TeX-master: "dtrace-specification"
%%% End:

\section{Multithreading}

Software threads are a key component of a modern operating system and were the
reason many design decisions have been made in DTrace. These threads may be
supported by having many processors implemented in hardware, preemption of a
single processor or through preemption of multiple processors using the
scheduler.

\subsection{DTrace guarantees}

% XXX: This is very vague because saying that DTrace guarantees something is
% a very bold statement. For example, saying that it guarantees thread safety
% when creating dynamic variables would be a little over the top, as this has
% not been verified at all.

When tracing, DTrace guarantees that it can not be preempted inside of the
\function{dtrace\_probe} function, but it does not guarantee that everything in
the DIF will be thread-safe.

\subsection{Variables}

As mentioned in (some previous section in this chapter), DTrace implements three
different scopes of variables. Global variables are not stored in thread-local storage,
while thread-local and clause-local variables are. This means that in a
multithreaded environment, global variables should be used sparingly. While it
is evident that storing to a global variable may be overwritten with another
probe that stores something else in the same one, there is more subtle behaviour
at hand. Consider the following example:

\begin{verbatim}
dtrace:::BEGIN
{
    num_syscalls = 0;
}

syscall:::entry
{
    num_syscalls++;
}

dtrace:::END
{
    printf("Number of syscalls: %d\n", num_syscalls);
}
\end{verbatim}

\noindent
Because DIF performs all of it operations on virtual machine's registers as opposed
to variables in memory, the ++ operator is not atomic. The above example when compiled
to DIF will look as follows:

\begin{verbatim}
ldgs %r1, num_syscalls /* Load the current value into %r1 */
setx %r2, inttab[0]    /* Load 1 into %r2 */
add  %r2, %r1, %r2     /* Add %r1 and %r2 and store into %r2 */
stgs %r2, num_syscalls /* Store the result back into num_syscalls */
\end{verbatim}

\noindent
This DIF section is safe, as long as the num\_syscalls variable is not visible from any
other thread. If it is visible and accessible from another thread, it suffers from a race
condition which results in wrong information being given to the user. Consider the following:

\begin{verbatim}
      Thread 1                     Thread 2

ldgs %r1, num_syscalls
                              ldgs %r3, num_syscalls
                              setx %r4, inttab[0]
                              add  %r4, %r3, %r4
setx %r2, inttab[0]
add  %r2, %r1, %r2
stgs %r2, num_syscalls
                              stgs %r4, num_syscalls
\end{verbatim}

\noindent
It is clear that the value in the \registerop{r2} register will be lost because the register
\registerop{r4} is stored to the same location afterwards. It is worth noting that this
behaviour is not observed because the thread was preempted, but simply by the fact that DTrace
does not guarantee any ordering outside of each CPU core.