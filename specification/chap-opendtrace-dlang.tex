The D language is a language inspired by the AWK programming language
\cite{Aho:1987:APL:29361} and the C programming language
\cite{DTrace2004}\cite{Kernighan:1988}. In this chapter, we give a
formal definition of the D programming language that is a part of
OpenDTrace, as well as elaborate on its properties in multithreaded
environments.

\section{Grammar definition}
\label{sec:grammar}


\begingroup\makeatletter%
\ifx\bnfdef\undefined%
\gdef\bnfdef{%
  {{$::=$}}%
  }%
\fi%
\ifx\bnfor\undefined%
\gdef\bnfor{%
  {{$\,|\,$}}%
  }%
\fi%
\ifx\bnfopt\undefined%
\gdef\bnfopt#1{%
  {{$\lbrack$ #1 $\rbrack$}}%
  }%
\fi%
\ifx\bnflist\undefined%
\gdef\bnflist#1{%
  {{$\lbrace$ #1 $\rbrace$}}%
  }%
\fi%
\ifx\bnfgroup\undefined%
\gdef\bnfgroup#1{%
  {{$($ #1 $)$}}%
  }%
\fi%
\ifx\bnftoken\undefined%
\gdef\bnftoken#1{%
  {{\bfseries\ttfamily #1}}%
  }%
\fi%
\ifx\bnfleftident\undefined%
\gdef\bnfleftident#1{%
  {{\itshape #1}}%
  }%
\fi%
\ifx\bnfrightident\undefined%
\gdef\bnfrightident#1{%
  {{\itshape #1}}%
  }%
\fi%
\gdef\bnftt{cmtt}%
\gdef\bnfttselect#1#2{%
\ifx\f@family\bnftt #1\else #2\fi%
}%
\endgroup%

\newcommand{\bnfunderscore}{\bnfttselect{\char`\_}{\_}}
\newcommand{\bnfbackslash}{\bnfttselect{\char`\\}{\textbackslash}}
\newcommand{\bnfleftbrace}{\bnfttselect{\char`\{}{\{}}
\newcommand{\bnfrightbrace}{\bnfttselect{\char`\}}{\}}}
\newcommand{\bnfgreater}{\bnfttselect{\char`\>}{\textgreater}}
\newcommand{\bnfless}{\bnfttselect{\char`\<}{\textless}}
\newlength\rulelhs
\newlength\rulemid
\newlength\rulerhs
\newcommand{\ebnftable}{
\settowidth\rulelhs{parameter\bnfunderscore{}declaration\bnfunderscore{}specifiers}
\settowidth\rulemid{::=}
\setlength\rulerhs{\textwidth}
\addtolength\rulerhs{-\rulelhs}
\addtolength\rulerhs{-\rulemid}
\addtolength\rulerhs{-6\tabcolsep}

\begin{longtable}{lrl}
\bnfleftident{dtrace\bnfunderscore{}program} & \bnfdef &
  \begin{minipage}[t]{\rulerhs}
  \raggedright
  \bnfgroup{\bnfrightident{d\bnfunderscore{}expression} \bnfor{} \bnfrightident{d\bnfunderscore{}program} \bnfor{} \bnfrightident{d\bnfunderscore{}type}} \bnfrightident{DT\bnfunderscore{}TOK\bnfunderscore{}EOF}
  \end{minipage}\\

\bnfleftident{d\bnfunderscore{}expression} & \bnfdef &
  \begin{minipage}[t]{\rulerhs}
  \raggedright
  \bnfrightident{DT\bnfunderscore{}CTX\bnfunderscore{}DEXPR} \bnfopt{\bnfrightident{expression}}
  \end{minipage}\\

\bnfleftident{d\bnfunderscore{}program} & \bnfdef &
  \begin{minipage}[t]{\rulerhs}
  \raggedright
  \bnfrightident{DT\bnfunderscore{}CTX\bnfunderscore{}DPROG} \bnfopt{\bnfrightident{translation\bnfunderscore{}unit}}
  \end{minipage}\\

\bnfleftident{d\bnfunderscore{}type} & \bnfdef &
  \begin{minipage}[t]{\rulerhs}
  \raggedright
  \bnfrightident{DT\bnfunderscore{}CTX\bnfunderscore{}DTYPE} \bnfopt{\bnfrightident{type\bnfunderscore{}name}}
  \end{minipage}\\

\bnfleftident{translation\bnfunderscore{}unit} & \bnfdef &
  \begin{minipage}[t]{\rulerhs}
  \raggedright
  \bnfrightident{external\bnfunderscore{}declaration} \bnflist{\bnfrightident{external\bnfunderscore{}declaration}}
  \end{minipage}\\

\bnfleftident{external\bnfunderscore{}declaration} & \bnfdef &
  \begin{minipage}[t]{\rulerhs}
  \raggedright
  \bnfrightident{inline\bnfunderscore{}definition}
  \end{minipage}\\
                     &  \bnfor{}  &
  \begin{minipage}[t]{\rulerhs}
  \raggedright
  \bnfrightident{translator\bnfunderscore{}definition}
  \end{minipage}\\
                     &  \bnfor{}  &
  \begin{minipage}[t]{\rulerhs}
  \raggedright
  \bnfrightident{provider\bnfunderscore{}definition}
  \end{minipage}\\
                     &  \bnfor{}  &
  \begin{minipage}[t]{\rulerhs}
  \raggedright
  \bnfrightident{probe\bnfunderscore{}definition}
  \end{minipage}\\
                     &  \bnfor{}  &
  \begin{minipage}[t]{\rulerhs}
  \raggedright
  \bnfrightident{declaration}
  \end{minipage}\\

\bnfleftident{inline\bnfunderscore{}definition} & \bnfdef &
  \begin{minipage}[t]{\rulerhs}
  \raggedright
  \bnfrightident{DT\bnfunderscore{}KEY\bnfunderscore{}INLINE} \bnfrightident{declaration\bnfunderscore{}specifiers} \bnfrightident{declarator} \bnfrightident{DT\bnfunderscore{}TOK\bnfunderscore{}ASGN} \bnfrightident{assignment\bnfunderscore{}expression} \bnftoken{;}
  \end{minipage}\\

\bnfleftident{translator\bnfunderscore{}definition} & \bnfdef &
  \begin{minipage}[t]{\rulerhs}
  \raggedright
  \bnfrightident{DT\bnfunderscore{}KEY\bnfunderscore{}XLATOR} \bnfrightident{type\bnfunderscore{}name} \bnfrightident{DT\bnfunderscore{}TOK\bnfunderscore{}LT} \bnfrightident{type\bnfunderscore{}name} \bnfrightident{DT\bnfunderscore{}TOK\bnfunderscore{}IDENT} \bnfrightident{DT\bnfunderscore{}TOK\bnfunderscore{}GT} \bnftoken{\bnfleftbrace{}} \bnfopt{\bnfrightident{translator\bnfunderscore{}member\bnfunderscore{}list}} \bnftoken{\bnfrightbrace{}} \bnftoken{;}
  \end{minipage}\\

\bnfleftident{translator\bnfunderscore{}member\bnfunderscore{}list} & \bnfdef &
  \begin{minipage}[t]{\rulerhs}
  \raggedright
  \bnfrightident{translator\bnfunderscore{}member} \bnflist{\bnfrightident{translator\bnfunderscore{}member}}
  \end{minipage}\\

\bnfleftident{translator\bnfunderscore{}member} & \bnfdef &
  \begin{minipage}[t]{\rulerhs}
  \raggedright
  \bnfrightident{DT\bnfunderscore{}TOK\bnfunderscore{}IDENT} \bnfrightident{DT\bnfunderscore{}TOK\bnfunderscore{}ASGN} \bnfrightident{assignment\bnfunderscore{}expression} \bnftoken{;}
  \end{minipage}\\

\bnfleftident{provider\bnfunderscore{}definition} & \bnfdef &
  \begin{minipage}[t]{\rulerhs}
  \raggedright
  \bnfrightident{DT\bnfunderscore{}KEY\bnfunderscore{}PROVIDER} \bnfrightident{DT\bnfunderscore{}TOK\bnfunderscore{}IDENT} \bnftoken{\bnfleftbrace{}} \bnfopt{\bnfrightident{provider\bnfunderscore{}probe\bnfunderscore{}list}} \bnftoken{\bnfrightbrace{}} \bnftoken{;}
  \end{minipage}\\

\bnfleftident{provider\bnfunderscore{}probe\bnfunderscore{}list} & \bnfdef &
  \begin{minipage}[t]{\rulerhs}
  \raggedright
  \bnfrightident{provider\bnfunderscore{}probe} \bnflist{\bnfrightident{provider\bnfunderscore{}probe}}
  \end{minipage}\\

\bnfleftident{provider\bnfunderscore{}probe} & \bnfdef &
  \begin{minipage}[t]{\rulerhs}
  \raggedright
  \bnfrightident{DT\bnfunderscore{}KEY\bnfunderscore{}PROBE} \bnfrightident{DT\bnfunderscore{}TOK\bnfunderscore{}IDENT} \bnfrightident{function} \bnfopt{\bnfrightident{DT\bnfunderscore{}TOK\bnfunderscore{}COLON} \bnfrightident{function}} \bnftoken{;}
  \end{minipage}\\

\bnfleftident{probe\bnfunderscore{}definition} & \bnfdef &
  \begin{minipage}[t]{\rulerhs}
  \raggedright
  \bnfrightident{probe\bnfunderscore{}specifiers} \bnfopt{\bnftoken{\bnfleftbrace{}} \bnfrightident{statement\bnfunderscore{}list} \bnftoken{\bnfrightbrace{}} \bnfor{} \bnfrightident{DT\bnfunderscore{}TOK\bnfunderscore{}DIV} \bnfrightident{expression} \bnfrightident{DT\bnfunderscore{}TOK\bnfunderscore{}EPRED} \bnfor{} \bnfrightident{DT\bnfunderscore{}TOK\bnfunderscore{}DIV} \bnfrightident{expression} \bnfrightident{DT\bnfunderscore{}TOK\bnfunderscore{}EPRED} \bnftoken{\bnfleftbrace{}} \bnfrightident{statement\bnfunderscore{}list} \bnftoken{\bnfrightbrace{}}}
  \end{minipage}\\

\bnfleftident{probe\bnfunderscore{}specifiers} & \bnfdef &
  \begin{minipage}[t]{\rulerhs}
  \raggedright
  \bnfrightident{probe\bnfunderscore{}specifier\bnfunderscore{}list}
  \end{minipage}\\

\bnfleftident{probe\bnfunderscore{}specifier\bnfunderscore{}list} & \bnfdef &
  \begin{minipage}[t]{\rulerhs}
  \raggedright
  \bnfrightident{probe\bnfunderscore{}specifier} \bnflist{\bnfrightident{DT\bnfunderscore{}TOK\bnfunderscore{}COMMA} \bnfrightident{probe\bnfunderscore{}specifier}}
  \end{minipage}\\

\bnfleftident{probe\bnfunderscore{}specifier} & \bnfdef &
  \begin{minipage}[t]{\rulerhs}
  \raggedright
  \bnfrightident{DT\bnfunderscore{}TOK\bnfunderscore{}PSPEC}
  \end{minipage}\\
                &  \bnfor{}  &
  \begin{minipage}[t]{\rulerhs}
  \raggedright
  \bnfrightident{DT\bnfunderscore{}TOK\bnfunderscore{}INT}
  \end{minipage}\\

\bnfleftident{statement\bnfunderscore{}list\bnfunderscore{}impl} & \bnfdef &
  \begin{minipage}[t]{\rulerhs}
  \raggedright
  \bnflist{\bnfrightident{statement}}
  \end{minipage}\\

\bnfleftident{statement\bnfunderscore{}list} & \bnfdef &
  \begin{minipage}[t]{\rulerhs}
  \raggedright
  \bnfrightident{statement\bnfunderscore{}list\bnfunderscore{}impl} \bnfopt{\bnfrightident{expression}}
  \end{minipage}\\

\bnfleftident{statement\bnfunderscore{}or\bnfunderscore{}block} & \bnfdef &
  \begin{minipage}[t]{\rulerhs}
  \raggedright
  \bnfrightident{statement}
  \end{minipage}\\
                   &  \bnfor{}  &
  \begin{minipage}[t]{\rulerhs}
  \raggedright
  \bnftoken{\bnfleftbrace{}} \bnfrightident{statement\bnfunderscore{}list} \bnftoken{\bnfrightbrace{}}
  \end{minipage}\\

\bnfleftident{statement} & \bnfdef &
  \begin{minipage}[t]{\rulerhs}
  \raggedright
  \bnftoken{;}
  \end{minipage}\\
          &  \bnfor{}  &
  \begin{minipage}[t]{\rulerhs}
  \raggedright
  \bnfrightident{expression} \bnftoken{;}
  \end{minipage}\\
          &  \bnfor{}  &
  \begin{minipage}[t]{\rulerhs}
  \raggedright
  \bnfrightident{DT\bnfunderscore{}KEY\bnfunderscore{}IF} \bnfrightident{DT\bnfunderscore{}TOK\bnfunderscore{}LPAR} \bnfrightident{expression} \bnfrightident{DT\bnfunderscore{}TOK\bnfunderscore{}RPAR} \bnfrightident{statement\bnfunderscore{}or\bnfunderscore{}block}
  \end{minipage}\\
          &  \bnfor{}  &
  \begin{minipage}[t]{\rulerhs}
  \raggedright
  \bnfrightident{DT\bnfunderscore{}KEY\bnfunderscore{}IF} \bnfrightident{DT\bnfunderscore{}TOK\bnfunderscore{}LPAR} \bnfrightident{expression} \bnfrightident{DT\bnfunderscore{}TOK\bnfunderscore{}RPAR} \bnfrightident{statement\bnfunderscore{}or\bnfunderscore{}block} \bnfrightident{DT\bnfunderscore{}KEY\bnfunderscore{}ELSE} \bnfrightident{statement\bnfunderscore{}or\bnfunderscore{}block}
  \end{minipage}\\

\bnfleftident{argument\bnfunderscore{}expression\bnfunderscore{}list} & \bnfdef &
  \begin{minipage}[t]{\rulerhs}
  \raggedright
  \bnfrightident{assignment\bnfunderscore{}expression} \bnflist{\bnfrightident{DT\bnfunderscore{}TOK\bnfunderscore{}COMMA} \bnfrightident{assignment\bnfunderscore{}expression}}
  \end{minipage}\\

\bnfleftident{primary\bnfunderscore{}expression} & \bnfdef &
  \begin{minipage}[t]{\rulerhs}
  \raggedright
  \bnfrightident{DT\bnfunderscore{}TOK\bnfunderscore{}IDENT}
  \end{minipage}\\
                   &  \bnfor{}  &
  \begin{minipage}[t]{\rulerhs}
  \raggedright
  \bnfrightident{DT\bnfunderscore{}TOK\bnfunderscore{}AGG}
  \end{minipage}\\
                   &  \bnfor{}  &
  \begin{minipage}[t]{\rulerhs}
  \raggedright
  \bnfrightident{DT\bnfunderscore{}TOK\bnfunderscore{}INT}
  \end{minipage}\\
                   &  \bnfor{}  &
  \begin{minipage}[t]{\rulerhs}
  \raggedright
  \bnfrightident{DT\bnfunderscore{}TOK\bnfunderscore{}STRING}
  \end{minipage}\\
                   &  \bnfor{}  &
  \begin{minipage}[t]{\rulerhs}
  \raggedright
  \bnfrightident{DT\bnfunderscore{}KEY\bnfunderscore{}SELF}
  \end{minipage}\\
                   &  \bnfor{}  &
  \begin{minipage}[t]{\rulerhs}
  \raggedright
  \bnfrightident{DT\bnfunderscore{}KEY\bnfunderscore{}THIS}
  \end{minipage}\\
                   &  \bnfor{}  &
  \begin{minipage}[t]{\rulerhs}
  \raggedright
  \bnfrightident{DT\bnfunderscore{}TOK\bnfunderscore{}LPAR} \bnfrightident{expression} \bnfrightident{DT\bnfunderscore{}TOK\bnfunderscore{}RPAR}
  \end{minipage}\\

\bnfleftident{postfix\bnfunderscore{}expression} & \bnfdef &
  \begin{minipage}[t]{\rulerhs}
  \raggedright
  \bnfgroup{\bnfrightident{primary\bnfunderscore{}expression} \bnfor{} \bnfrightident{DT\bnfunderscore{}TOK\bnfunderscore{}OFFSETOF} \bnfrightident{DT\bnfunderscore{}TOK\bnfunderscore{}LPAR} \bnfrightident{type\bnfunderscore{}name} \bnfrightident{DT\bnfunderscore{}TOK\bnfunderscore{}COMMA} \bnfrightident{DT\bnfunderscore{}TOK\bnfunderscore{}IDENT} \bnfrightident{DT\bnfunderscore{}TOK\bnfunderscore{}RPAR} \bnfor{} \bnfrightident{DT\bnfunderscore{}TOK\bnfunderscore{}OFFSETOF} \bnfrightident{DT\bnfunderscore{}TOK\bnfunderscore{}LPAR} \bnfrightident{type\bnfunderscore{}name} \bnfrightident{DT\bnfunderscore{}TOK\bnfunderscore{}COMMA} \bnfrightident{DT\bnfunderscore{}TOK\bnfunderscore{}TNAME} \bnfrightident{DT\bnfunderscore{}TOK\bnfunderscore{}RPAR} \bnfor{} \bnfrightident{DT\bnfunderscore{}TOK\bnfunderscore{}OFFSETOF} \bnfrightident{DT\bnfunderscore{}TOK\bnfunderscore{}LPAR} \bnfrightident{type\bnfunderscore{}name} \bnfrightident{DT\bnfunderscore{}TOK\bnfunderscore{}COMMA} \bnfrightident{dtrace\bnfunderscore{}keyword\bnfunderscore{}ident} \bnfrightident{DT\bnfunderscore{}TOK\bnfunderscore{}RPAR} \bnfor{} \bnfrightident{DT\bnfunderscore{}TOK\bnfunderscore{}XLATE} \bnfrightident{DT\bnfunderscore{}TOK\bnfunderscore{}LT} \bnfrightident{type\bnfunderscore{}name} \bnfrightident{DT\bnfunderscore{}TOK\bnfunderscore{}GT} \bnfrightident{DT\bnfunderscore{}TOK\bnfunderscore{}LPAR} \bnfrightident{expression} \bnfrightident{DT\bnfunderscore{}TOK\bnfunderscore{}RPAR}} \bnflist{\bnfrightident{DT\bnfunderscore{}TOK\bnfunderscore{}LBRAC} \bnfrightident{argument\bnfunderscore{}expression\bnfunderscore{}list} \bnfrightident{DT\bnfunderscore{}TOK\bnfunderscore{}RBRAC} \bnfor{} \bnfrightident{DT\bnfunderscore{}TOK\bnfunderscore{}LPAR} \bnfrightident{DT\bnfunderscore{}TOK\bnfunderscore{}RPAR} \bnfor{} \bnfrightident{DT\bnfunderscore{}TOK\bnfunderscore{}LPAR} \bnfrightident{argument\bnfunderscore{}expression\bnfunderscore{}list} \bnfrightident{DT\bnfunderscore{}TOK\bnfunderscore{}RPAR} \bnfor{} \bnfrightident{DT\bnfunderscore{}TOK\bnfunderscore{}DOT} \bnfrightident{DT\bnfunderscore{}TOK\bnfunderscore{}IDENT} \bnfor{} \bnfrightident{DT\bnfunderscore{}TOK\bnfunderscore{}DOT} \bnfrightident{DT\bnfunderscore{}TOK\bnfunderscore{}TNAME} \bnfor{} \bnfrightident{DT\bnfunderscore{}TOK\bnfunderscore{}DOT} \bnfrightident{dtrace\bnfunderscore{}keyword\bnfunderscore{}ident} \bnfor{} \bnfrightident{DT\bnfunderscore{}TOK\bnfunderscore{}PTR} \bnfrightident{DT\bnfunderscore{}TOK\bnfunderscore{}IDENT} \bnfor{} \bnfrightident{DT\bnfunderscore{}TOK\bnfunderscore{}PTR} \bnfrightident{DT\bnfunderscore{}TOK\bnfunderscore{}TNAME} \bnfor{} \bnfrightident{DT\bnfunderscore{}TOK\bnfunderscore{}PTR} \bnfrightident{dtrace\bnfunderscore{}keyword\bnfunderscore{}ident} \bnfor{} \bnfrightident{DT\bnfunderscore{}TOK\bnfunderscore{}ADDADD} \bnfor{} \bnfrightident{DT\bnfunderscore{}TOK\bnfunderscore{}SUBSUB}}
  \end{minipage}\\

\bnfleftident{unary\bnfunderscore{}expression} & \bnfdef &
  \begin{minipage}[t]{\rulerhs}
  \raggedright
  \bnflist{\bnfrightident{DT\bnfunderscore{}TOK\bnfunderscore{}STRINGOF} \bnfor{} \bnfrightident{DT\bnfunderscore{}TOK\bnfunderscore{}SIZEOF} \bnfor{} \bnfrightident{DT\bnfunderscore{}TOK\bnfunderscore{}SUBSUB} \bnfor{} \bnfrightident{DT\bnfunderscore{}TOK\bnfunderscore{}ADDADD}} \bnfgroup{\bnfrightident{DT\bnfunderscore{}TOK\bnfunderscore{}RPAR} \bnfrightident{type\bnfunderscore{}name} \bnfrightident{DT\bnfunderscore{}TOK\bnfunderscore{}LPAR} \bnfrightident{DT\bnfunderscore{}TOK\bnfunderscore{}SIZEOF} \bnfor{} \bnfrightident{cast\bnfunderscore{}expression} \bnfrightident{unary\bnfunderscore{}operator} \bnfor{} \bnfrightident{postfix\bnfunderscore{}expression}}
  \end{minipage}\\

\bnfleftident{unary\bnfunderscore{}operator} & \bnfdef &
  \begin{minipage}[t]{\rulerhs}
  \raggedright
  \bnfrightident{DT\bnfunderscore{}TOK\bnfunderscore{}BAND}
  \end{minipage}\\
               &  \bnfor{}  &
  \begin{minipage}[t]{\rulerhs}
  \raggedright
  \bnfrightident{DT\bnfunderscore{}TOK\bnfunderscore{}MUL}
  \end{minipage}\\
               &  \bnfor{}  &
  \begin{minipage}[t]{\rulerhs}
  \raggedright
  \bnfrightident{DT\bnfunderscore{}TOK\bnfunderscore{}ADD}
  \end{minipage}\\
               &  \bnfor{}  &
  \begin{minipage}[t]{\rulerhs}
  \raggedright
  \bnfrightident{DT\bnfunderscore{}TOK\bnfunderscore{}SUB}
  \end{minipage}\\
               &  \bnfor{}  &
  \begin{minipage}[t]{\rulerhs}
  \raggedright
  \bnfrightident{DT\bnfunderscore{}TOK\bnfunderscore{}BNEG}
  \end{minipage}\\
               &  \bnfor{}  &
  \begin{minipage}[t]{\rulerhs}
  \raggedright
  \bnfrightident{DT\bnfunderscore{}TOK\bnfunderscore{}LNEG}
  \end{minipage}\\

\bnfleftident{cast\bnfunderscore{}expression} & \bnfdef &
  \begin{minipage}[t]{\rulerhs}
  \raggedright
  \bnflist{\bnfrightident{DT\bnfunderscore{}TOK\bnfunderscore{}RPAR} \bnfrightident{type\bnfunderscore{}name} \bnfrightident{DT\bnfunderscore{}TOK\bnfunderscore{}LPAR}} \bnfrightident{unary\bnfunderscore{}expression}
  \end{minipage}\\

\bnfleftident{multiplicative\bnfunderscore{}expression} & \bnfdef &
  \begin{minipage}[t]{\rulerhs}
  \raggedright
  \bnfrightident{cast\bnfunderscore{}expression} \bnflist{\bnfgroup{\bnfrightident{DT\bnfunderscore{}TOK\bnfunderscore{}MUL} \bnfor{} \bnfrightident{DT\bnfunderscore{}TOK\bnfunderscore{}DIV} \bnfor{} \bnfrightident{DT\bnfunderscore{}TOK\bnfunderscore{}MOD}} \bnfrightident{cast\bnfunderscore{}expression}}
  \end{minipage}\\

\bnfleftident{additive\bnfunderscore{}expression} & \bnfdef &
  \begin{minipage}[t]{\rulerhs}
  \raggedright
  \bnfrightident{multiplicative\bnfunderscore{}expression} \bnflist{\bnfgroup{\bnfrightident{DT\bnfunderscore{}TOK\bnfunderscore{}ADD} \bnfor{} \bnfrightident{DT\bnfunderscore{}TOK\bnfunderscore{}SUB}} \bnfrightident{multiplicative\bnfunderscore{}expression}}
  \end{minipage}\\

\bnfleftident{shift\bnfunderscore{}expression} & \bnfdef &
  \begin{minipage}[t]{\rulerhs}
  \raggedright
  \bnfrightident{additive\bnfunderscore{}expression} \bnflist{\bnfgroup{\bnfrightident{DT\bnfunderscore{}TOK\bnfunderscore{}LSH} \bnfor{} \bnfrightident{DT\bnfunderscore{}TOK\bnfunderscore{}RSH}} \bnfrightident{additive\bnfunderscore{}expression}}
  \end{minipage}\\

\bnfleftident{relational\bnfunderscore{}expression} & \bnfdef &
  \begin{minipage}[t]{\rulerhs}
  \raggedright
  \bnfrightident{shift\bnfunderscore{}expression} \bnflist{\bnfgroup{\bnfrightident{DT\bnfunderscore{}TOK\bnfunderscore{}LT} \bnfor{} \bnfrightident{DT\bnfunderscore{}TOK\bnfunderscore{}GT} \bnfor{} \bnfrightident{DT\bnfunderscore{}TOK\bnfunderscore{}LE} \bnfor{} \bnfrightident{DT\bnfunderscore{}TOK\bnfunderscore{}GE}} \bnfrightident{shift\bnfunderscore{}expression}}
  \end{minipage}\\

\bnfleftident{equality\bnfunderscore{}expression} & \bnfdef &
  \begin{minipage}[t]{\rulerhs}
  \raggedright
  \bnfrightident{relational\bnfunderscore{}expression} \bnflist{\bnfgroup{\bnfrightident{DT\bnfunderscore{}TOK\bnfunderscore{}EQU} \bnfor{} \bnfrightident{DT\bnfunderscore{}TOK\bnfunderscore{}NEQ}} \bnfrightident{relational\bnfunderscore{}expression}}
  \end{minipage}\\

\bnfleftident{and\bnfunderscore{}expression} & \bnfdef &
  \begin{minipage}[t]{\rulerhs}
  \raggedright
  \bnfrightident{equality\bnfunderscore{}expression} \bnflist{\bnfrightident{DT\bnfunderscore{}TOK\bnfunderscore{}BAND} \bnfrightident{equality\bnfunderscore{}expression}}
  \end{minipage}\\

\bnfleftident{exclusive\bnfunderscore{}or\bnfunderscore{}expression} & \bnfdef &
  \begin{minipage}[t]{\rulerhs}
  \raggedright
  \bnfrightident{and\bnfunderscore{}expression} \bnflist{\bnfrightident{DT\bnfunderscore{}TOK\bnfunderscore{}XOR} \bnfrightident{and\bnfunderscore{}expression}}
  \end{minipage}\\

\bnfleftident{inclusive\bnfunderscore{}or\bnfunderscore{}expression} & \bnfdef &
  \begin{minipage}[t]{\rulerhs}
  \raggedright
  \bnfrightident{exclusive\bnfunderscore{}or\bnfunderscore{}expression} \bnflist{\bnfrightident{DT\bnfunderscore{}TOK\bnfunderscore{}BOR} \bnfrightident{exclusive\bnfunderscore{}or\bnfunderscore{}expression}}
  \end{minipage}\\

\bnfleftident{logical\bnfunderscore{}and\bnfunderscore{}expression} & \bnfdef &
  \begin{minipage}[t]{\rulerhs}
  \raggedright
  \bnfrightident{inclusive\bnfunderscore{}or\bnfunderscore{}expression} \bnflist{\bnfrightident{DT\bnfunderscore{}TOK\bnfunderscore{}LAND} \bnfrightident{inclusive\bnfunderscore{}or\bnfunderscore{}expression}}
  \end{minipage}\\

\bnfleftident{logical\bnfunderscore{}xor\bnfunderscore{}expression} & \bnfdef &
  \begin{minipage}[t]{\rulerhs}
  \raggedright
  \bnfrightident{logical\bnfunderscore{}and\bnfunderscore{}expression} \bnflist{\bnfrightident{DT\bnfunderscore{}TOK\bnfunderscore{}LXOR} \bnfrightident{logical\bnfunderscore{}and\bnfunderscore{}expression}}
  \end{minipage}\\

\bnfleftident{logical\bnfunderscore{}or\bnfunderscore{}expression} & \bnfdef &
  \begin{minipage}[t]{\rulerhs}
  \raggedright
  \bnfrightident{logical\bnfunderscore{}xor\bnfunderscore{}expression} \bnflist{\bnfrightident{DT\bnfunderscore{}TOK\bnfunderscore{}LOR} \bnfrightident{logical\bnfunderscore{}xor\bnfunderscore{}expression}}
  \end{minipage}\\

\bnfleftident{constant\bnfunderscore{}expression} & \bnfdef &
  \begin{minipage}[t]{\rulerhs}
  \raggedright
  \bnfrightident{conditional\bnfunderscore{}expression}
  \end{minipage}\\

\bnfleftident{conditional\bnfunderscore{}expression} & \bnfdef &
  \begin{minipage}[t]{\rulerhs}
  \raggedright
  \bnflist{\bnfrightident{DT\bnfunderscore{}TOK\bnfunderscore{}COLON} \bnfrightident{expression} \bnfrightident{DT\bnfunderscore{}TOK\bnfunderscore{}QUESTION} \bnfrightident{logical\bnfunderscore{}or\bnfunderscore{}expression}} \bnfrightident{logical\bnfunderscore{}or\bnfunderscore{}expression}
  \end{minipage}\\

\bnfleftident{assignment\bnfunderscore{}expression} & \bnfdef &
  \begin{minipage}[t]{\rulerhs}
  \raggedright
  \bnflist{\bnfrightident{assignment\bnfunderscore{}operator} \bnfrightident{unary\bnfunderscore{}expression}} \bnfrightident{conditional\bnfunderscore{}expression}
  \end{minipage}\\

\bnfleftident{assignment\bnfunderscore{}operator} & \bnfdef &
  \begin{minipage}[t]{\rulerhs}
  \raggedright
  \bnfrightident{DT\bnfunderscore{}TOK\bnfunderscore{}ASGN}
  \end{minipage}\\
                    &  \bnfor{}  &
  \begin{minipage}[t]{\rulerhs}
  \raggedright
  \bnfrightident{DT\bnfunderscore{}TOK\bnfunderscore{}MUL\bnfunderscore{}EQ}
  \end{minipage}\\
                    &  \bnfor{}  &
  \begin{minipage}[t]{\rulerhs}
  \raggedright
  \bnfrightident{DT\bnfunderscore{}TOK\bnfunderscore{}DIV\bnfunderscore{}EQ}
  \end{minipage}\\
                    &  \bnfor{}  &
  \begin{minipage}[t]{\rulerhs}
  \raggedright
  \bnfrightident{DT\bnfunderscore{}TOK\bnfunderscore{}MOD\bnfunderscore{}EQ}
  \end{minipage}\\
                    &  \bnfor{}  &
  \begin{minipage}[t]{\rulerhs}
  \raggedright
  \bnfrightident{DT\bnfunderscore{}TOK\bnfunderscore{}ADD\bnfunderscore{}EQ}
  \end{minipage}\\
                    &  \bnfor{}  &
  \begin{minipage}[t]{\rulerhs}
  \raggedright
  \bnfrightident{DT\bnfunderscore{}TOK\bnfunderscore{}SUB\bnfunderscore{}EQ}
  \end{minipage}\\
                    &  \bnfor{}  &
  \begin{minipage}[t]{\rulerhs}
  \raggedright
  \bnfrightident{DT\bnfunderscore{}TOK\bnfunderscore{}LSH\bnfunderscore{}EQ}
  \end{minipage}\\
                    &  \bnfor{}  &
  \begin{minipage}[t]{\rulerhs}
  \raggedright
  \bnfrightident{DT\bnfunderscore{}TOK\bnfunderscore{}RSH\bnfunderscore{}EQ}
  \end{minipage}\\
                    &  \bnfor{}  &
  \begin{minipage}[t]{\rulerhs}
  \raggedright
  \bnfrightident{DT\bnfunderscore{}TOK\bnfunderscore{}AND\bnfunderscore{}EQ}
  \end{minipage}\\
                    &  \bnfor{}  &
  \begin{minipage}[t]{\rulerhs}
  \raggedright
  \bnfrightident{DT\bnfunderscore{}TOK\bnfunderscore{}XOR\bnfunderscore{}EQ}
  \end{minipage}\\
                    &  \bnfor{}  &
  \begin{minipage}[t]{\rulerhs}
  \raggedright
  \bnfrightident{DT\bnfunderscore{}TOK\bnfunderscore{}OR\bnfunderscore{}EQ}
  \end{minipage}\\

\bnfleftident{expression} & \bnfdef &
  \begin{minipage}[t]{\rulerhs}
  \raggedright
  \bnfrightident{assignment\bnfunderscore{}expression} \bnflist{\bnfrightident{DT\bnfunderscore{}TOK\bnfunderscore{}COMMA} \bnfrightident{assignment\bnfunderscore{}expression}}
  \end{minipage}\\

\bnfleftident{declaration} & \bnfdef &
  \begin{minipage}[t]{\rulerhs}
  \raggedright
  \bnfrightident{declaration\bnfunderscore{}specifiers} \bnfopt{\bnfrightident{init\bnfunderscore{}declarator\bnfunderscore{}list}} \bnftoken{;}
  \end{minipage}\\

\bnfleftident{declaration\bnfunderscore{}specifiers} & \bnfdef &
  \begin{minipage}[t]{\rulerhs}
  \raggedright
  \bnflist{\bnfrightident{type\bnfunderscore{}qualifier} \bnfor{} \bnfrightident{type\bnfunderscore{}specifier} \bnfor{} \bnfrightident{d\bnfunderscore{}storage\bnfunderscore{}class\bnfunderscore{}specifier}} \bnfgroup{\bnfrightident{type\bnfunderscore{}qualifier} \bnfor{} \bnfrightident{type\bnfunderscore{}specifier} \bnfor{} \bnfrightident{d\bnfunderscore{}storage\bnfunderscore{}class\bnfunderscore{}specifier}}
  \end{minipage}\\

\bnfleftident{parameter\bnfunderscore{}declaration\bnfunderscore{}specifiers} & \bnfdef &
  \begin{minipage}[t]{\rulerhs}
  \raggedright
  \bnfrightident{storage\bnfunderscore{}class\bnfunderscore{}specifier}
  \end{minipage}\\
                                 &  \bnfor{}  &
  \begin{minipage}[t]{\rulerhs}
  \raggedright
  \bnfrightident{storage\bnfunderscore{}class\bnfunderscore{}specifier} \bnfrightident{declaration\bnfunderscore{}specifiers}
  \end{minipage}\\
                                 &  \bnfor{}  &
  \begin{minipage}[t]{\rulerhs}
  \raggedright
  \bnfrightident{type\bnfunderscore{}specifier}
  \end{minipage}\\
                                 &  \bnfor{}  &
  \begin{minipage}[t]{\rulerhs}
  \raggedright
  \bnfrightident{type\bnfunderscore{}specifier} \bnfrightident{declaration\bnfunderscore{}specifiers}
  \end{minipage}\\
                                 &  \bnfor{}  &
  \begin{minipage}[t]{\rulerhs}
  \raggedright
  \bnfrightident{type\bnfunderscore{}qualifier}
  \end{minipage}\\
                                 &  \bnfor{}  &
  \begin{minipage}[t]{\rulerhs}
  \raggedright
  \bnfrightident{type\bnfunderscore{}qualifier} \bnfrightident{declaration\bnfunderscore{}specifiers}
  \end{minipage}\\

\bnfleftident{storage\bnfunderscore{}class\bnfunderscore{}specifier} & \bnfdef &
  \begin{minipage}[t]{\rulerhs}
  \raggedright
  \bnfrightident{DT\bnfunderscore{}KEY\bnfunderscore{}AUTO}
  \end{minipage}\\
                        &  \bnfor{}  &
  \begin{minipage}[t]{\rulerhs}
  \raggedright
  \bnfrightident{DT\bnfunderscore{}KEY\bnfunderscore{}REGISTER}
  \end{minipage}\\
                        &  \bnfor{}  &
  \begin{minipage}[t]{\rulerhs}
  \raggedright
  \bnfrightident{DT\bnfunderscore{}KEY\bnfunderscore{}STATIC}
  \end{minipage}\\
                        &  \bnfor{}  &
  \begin{minipage}[t]{\rulerhs}
  \raggedright
  \bnfrightident{DT\bnfunderscore{}KEY\bnfunderscore{}EXTERN}
  \end{minipage}\\
                        &  \bnfor{}  &
  \begin{minipage}[t]{\rulerhs}
  \raggedright
  \bnfrightident{DT\bnfunderscore{}KEY\bnfunderscore{}TYPEDEF}
  \end{minipage}\\

\bnfleftident{d\bnfunderscore{}storage\bnfunderscore{}class\bnfunderscore{}specifier} & \bnfdef &
  \begin{minipage}[t]{\rulerhs}
  \raggedright
  \bnfrightident{storage\bnfunderscore{}class\bnfunderscore{}specifier}
  \end{minipage}\\
                          &  \bnfor{}  &
  \begin{minipage}[t]{\rulerhs}
  \raggedright
  \bnfrightident{DT\bnfunderscore{}KEY\bnfunderscore{}SELF}
  \end{minipage}\\
                          &  \bnfor{}  &
  \begin{minipage}[t]{\rulerhs}
  \raggedright
  \bnfrightident{DT\bnfunderscore{}KEY\bnfunderscore{}THIS}
  \end{minipage}\\

\bnfleftident{type\bnfunderscore{}specifier} & \bnfdef &
  \begin{minipage}[t]{\rulerhs}
  \raggedright
  \bnfrightident{DT\bnfunderscore{}KEY\bnfunderscore{}VOID}
  \end{minipage}\\
               &  \bnfor{}  &
  \begin{minipage}[t]{\rulerhs}
  \raggedright
  \bnfrightident{DT\bnfunderscore{}KEY\bnfunderscore{}CHAR}
  \end{minipage}\\
               &  \bnfor{}  &
  \begin{minipage}[t]{\rulerhs}
  \raggedright
  \bnfrightident{DT\bnfunderscore{}KEY\bnfunderscore{}SHORT}
  \end{minipage}\\
               &  \bnfor{}  &
  \begin{minipage}[t]{\rulerhs}
  \raggedright
  \bnfrightident{DT\bnfunderscore{}KEY\bnfunderscore{}INT}
  \end{minipage}\\
               &  \bnfor{}  &
  \begin{minipage}[t]{\rulerhs}
  \raggedright
  \bnfrightident{DT\bnfunderscore{}KEY\bnfunderscore{}LONG}
  \end{minipage}\\
               &  \bnfor{}  &
  \begin{minipage}[t]{\rulerhs}
  \raggedright
  \bnfrightident{DT\bnfunderscore{}KEY\bnfunderscore{}FLOAT}
  \end{minipage}\\
               &  \bnfor{}  &
  \begin{minipage}[t]{\rulerhs}
  \raggedright
  \bnfrightident{DT\bnfunderscore{}KEY\bnfunderscore{}DOUBLE}
  \end{minipage}\\
               &  \bnfor{}  &
  \begin{minipage}[t]{\rulerhs}
  \raggedright
  \bnfrightident{DT\bnfunderscore{}KEY\bnfunderscore{}SIGNED}
  \end{minipage}\\
               &  \bnfor{}  &
  \begin{minipage}[t]{\rulerhs}
  \raggedright
  \bnfrightident{DT\bnfunderscore{}KEY\bnfunderscore{}UNSIGNED}
  \end{minipage}\\
               &  \bnfor{}  &
  \begin{minipage}[t]{\rulerhs}
  \raggedright
  \bnfrightident{DT\bnfunderscore{}KEY\bnfunderscore{}USERLAND}
  \end{minipage}\\
               &  \bnfor{}  &
  \begin{minipage}[t]{\rulerhs}
  \raggedright
  \bnfrightident{DT\bnfunderscore{}KEY\bnfunderscore{}STRING}
  \end{minipage}\\
               &  \bnfor{}  &
  \begin{minipage}[t]{\rulerhs}
  \raggedright
  \bnfrightident{DT\bnfunderscore{}TOK\bnfunderscore{}TNAME}
  \end{minipage}\\
               &  \bnfor{}  &
  \begin{minipage}[t]{\rulerhs}
  \raggedright
  \bnfrightident{struct\bnfunderscore{}or\bnfunderscore{}union\bnfunderscore{}specifier}
  \end{minipage}\\
               &  \bnfor{}  &
  \begin{minipage}[t]{\rulerhs}
  \raggedright
  \bnfrightident{enum\bnfunderscore{}specifier}
  \end{minipage}\\

\bnfleftident{type\bnfunderscore{}qualifier} & \bnfdef &
  \begin{minipage}[t]{\rulerhs}
  \raggedright
  \bnfrightident{DT\bnfunderscore{}KEY\bnfunderscore{}CONST}
  \end{minipage}\\
               &  \bnfor{}  &
  \begin{minipage}[t]{\rulerhs}
  \raggedright
  \bnfrightident{DT\bnfunderscore{}KEY\bnfunderscore{}RESTRICT}
  \end{minipage}\\
               &  \bnfor{}  &
  \begin{minipage}[t]{\rulerhs}
  \raggedright
  \bnfrightident{DT\bnfunderscore{}KEY\bnfunderscore{}VOLATILE}
  \end{minipage}\\

\bnfleftident{struct\bnfunderscore{}or\bnfunderscore{}union\bnfunderscore{}specifier} & \bnfdef &
  \begin{minipage}[t]{\rulerhs}
  \raggedright
  \bnfrightident{struct\bnfunderscore{}or\bnfunderscore{}union\bnfunderscore{}definition} \bnfrightident{struct\bnfunderscore{}declaration\bnfunderscore{}list} \bnftoken{\bnfrightbrace{}}
  \end{minipage}\\
                          &  \bnfor{}  &
  \begin{minipage}[t]{\rulerhs}
  \raggedright
  \bnfrightident{struct\bnfunderscore{}or\bnfunderscore{}union} \bnfrightident{DT\bnfunderscore{}TOK\bnfunderscore{}IDENT}
  \end{minipage}\\
                          &  \bnfor{}  &
  \begin{minipage}[t]{\rulerhs}
  \raggedright
  \bnfrightident{struct\bnfunderscore{}or\bnfunderscore{}union} \bnfrightident{DT\bnfunderscore{}TOK\bnfunderscore{}TNAME}
  \end{minipage}\\

\bnfleftident{struct\bnfunderscore{}or\bnfunderscore{}union\bnfunderscore{}definition} & \bnfdef &
  \begin{minipage}[t]{\rulerhs}
  \raggedright
  \bnfrightident{struct\bnfunderscore{}or\bnfunderscore{}union} \bnfopt{\bnfrightident{DT\bnfunderscore{}TOK\bnfunderscore{}IDENT} \bnfor{} \bnfrightident{DT\bnfunderscore{}TOK\bnfunderscore{}TNAME}} \bnftoken{\bnfleftbrace{}}
  \end{minipage}\\

\bnfleftident{struct\bnfunderscore{}or\bnfunderscore{}union} & \bnfdef &
  \begin{minipage}[t]{\rulerhs}
  \raggedright
  \bnfrightident{DT\bnfunderscore{}KEY\bnfunderscore{}STRUCT}
  \end{minipage}\\
                &  \bnfor{}  &
  \begin{minipage}[t]{\rulerhs}
  \raggedright
  \bnfrightident{DT\bnfunderscore{}KEY\bnfunderscore{}UNION}
  \end{minipage}\\

\bnfleftident{struct\bnfunderscore{}declaration\bnfunderscore{}list} & \bnfdef &
  \begin{minipage}[t]{\rulerhs}
  \raggedright
  \bnfrightident{struct\bnfunderscore{}declaration} \bnflist{\bnfrightident{struct\bnfunderscore{}declaration}}
  \end{minipage}\\

\bnfleftident{init\bnfunderscore{}declarator\bnfunderscore{}list} & \bnfdef &
  \begin{minipage}[t]{\rulerhs}
  \raggedright
  \bnfrightident{init\bnfunderscore{}declarator} \bnflist{\bnfrightident{DT\bnfunderscore{}TOK\bnfunderscore{}COMMA} \bnfrightident{init\bnfunderscore{}declarator}}
  \end{minipage}\\

\bnfleftident{init\bnfunderscore{}declarator} & \bnfdef &
  \begin{minipage}[t]{\rulerhs}
  \raggedright
  \bnfrightident{declarator}
  \end{minipage}\\

\bnfleftident{struct\bnfunderscore{}declaration} & \bnfdef &
  \begin{minipage}[t]{\rulerhs}
  \raggedright
  \bnfrightident{specifier\bnfunderscore{}qualifier\bnfunderscore{}list} \bnfrightident{struct\bnfunderscore{}declarator\bnfunderscore{}list} \bnftoken{;}
  \end{minipage}\\

\bnfleftident{specifier\bnfunderscore{}qualifier\bnfunderscore{}list} & \bnfdef &
  \begin{minipage}[t]{\rulerhs}
  \raggedright
  \bnflist{\bnfrightident{type\bnfunderscore{}qualifier} \bnfor{} \bnfrightident{type\bnfunderscore{}specifier}} \bnfgroup{\bnfrightident{type\bnfunderscore{}qualifier} \bnfor{} \bnfrightident{type\bnfunderscore{}specifier}}
  \end{minipage}\\

\bnfleftident{struct\bnfunderscore{}declarator\bnfunderscore{}list} & \bnfdef &
  \begin{minipage}[t]{\rulerhs}
  \raggedright
  \bnfrightident{struct\bnfunderscore{}declarator} \bnflist{\bnfrightident{DT\bnfunderscore{}TOK\bnfunderscore{}COMMA} \bnfrightident{struct\bnfunderscore{}declarator}}
  \end{minipage}\\

\bnfleftident{struct\bnfunderscore{}declarator} & \bnfdef &
  \begin{minipage}[t]{\rulerhs}
  \raggedright
  \bnfrightident{declarator}
  \end{minipage}\\
                  &  \bnfor{}  &
  \begin{minipage}[t]{\rulerhs}
  \raggedright
  \bnfrightident{DT\bnfunderscore{}TOK\bnfunderscore{}COLON} \bnfrightident{constant\bnfunderscore{}expression}
  \end{minipage}\\
                  &  \bnfor{}  &
  \begin{minipage}[t]{\rulerhs}
  \raggedright
  \bnfrightident{declarator} \bnfrightident{DT\bnfunderscore{}TOK\bnfunderscore{}COLON} \bnfrightident{constant\bnfunderscore{}expression}
  \end{minipage}\\

\bnfleftident{enum\bnfunderscore{}specifier} & \bnfdef &
  \begin{minipage}[t]{\rulerhs}
  \raggedright
  \bnfrightident{enum\bnfunderscore{}definition} \bnfrightident{enumerator\bnfunderscore{}list} \bnftoken{\bnfrightbrace{}}
  \end{minipage}\\
               &  \bnfor{}  &
  \begin{minipage}[t]{\rulerhs}
  \raggedright
  \bnfrightident{DT\bnfunderscore{}KEY\bnfunderscore{}ENUM} \bnfrightident{DT\bnfunderscore{}TOK\bnfunderscore{}IDENT}
  \end{minipage}\\
               &  \bnfor{}  &
  \begin{minipage}[t]{\rulerhs}
  \raggedright
  \bnfrightident{DT\bnfunderscore{}KEY\bnfunderscore{}ENUM} \bnfrightident{DT\bnfunderscore{}TOK\bnfunderscore{}TNAME}
  \end{minipage}\\

\bnfleftident{enum\bnfunderscore{}definition} & \bnfdef &
  \begin{minipage}[t]{\rulerhs}
  \raggedright
  \bnfrightident{DT\bnfunderscore{}KEY\bnfunderscore{}ENUM} \bnfopt{\bnfrightident{DT\bnfunderscore{}TOK\bnfunderscore{}IDENT} \bnfor{} \bnfrightident{DT\bnfunderscore{}TOK\bnfunderscore{}TNAME}} \bnftoken{\bnfleftbrace{}}
  \end{minipage}\\

\bnfleftident{enumerator\bnfunderscore{}list} & \bnfdef &
  \begin{minipage}[t]{\rulerhs}
  \raggedright
  \bnfrightident{enumerator} \bnflist{\bnfrightident{DT\bnfunderscore{}TOK\bnfunderscore{}COMMA} \bnfrightident{enumerator}}
  \end{minipage}\\

\bnfleftident{enumerator} & \bnfdef &
  \begin{minipage}[t]{\rulerhs}
  \raggedright
  \bnfrightident{DT\bnfunderscore{}TOK\bnfunderscore{}IDENT} \bnfopt{\bnfrightident{DT\bnfunderscore{}TOK\bnfunderscore{}ASGN} \bnfrightident{expression}}
  \end{minipage}\\

\bnfleftident{declarator} & \bnfdef &
  \begin{minipage}[t]{\rulerhs}
  \raggedright
  \bnfopt{\bnfrightident{pointer}} \bnfrightident{direct\bnfunderscore{}declarator}
  \end{minipage}\\

\bnfleftident{direct\bnfunderscore{}declarator} & \bnfdef &
  \begin{minipage}[t]{\rulerhs}
  \raggedright
  \bnfgroup{\bnfrightident{DT\bnfunderscore{}TOK\bnfunderscore{}IDENT} \bnfor{} \bnfrightident{lparen} \bnfrightident{declarator} \bnfrightident{DT\bnfunderscore{}TOK\bnfunderscore{}RPAR}} \bnflist{\bnfrightident{array} \bnfor{} \bnfrightident{function}}
  \end{minipage}\\

\bnfleftident{lparen} & \bnfdef &
  \begin{minipage}[t]{\rulerhs}
  \raggedright
  \bnfrightident{DT\bnfunderscore{}TOK\bnfunderscore{}LPAR}
  \end{minipage}\\

\bnfleftident{pointer} & \bnfdef &
  \begin{minipage}[t]{\rulerhs}
  \raggedright
  \bnflist{\bnfopt{\bnfrightident{type\bnfunderscore{}qualifier\bnfunderscore{}list}} \bnfrightident{DT\bnfunderscore{}TOK\bnfunderscore{}MUL}} \bnfopt{\bnfrightident{type\bnfunderscore{}qualifier\bnfunderscore{}list}} \bnfrightident{DT\bnfunderscore{}TOK\bnfunderscore{}MUL}
  \end{minipage}\\

\bnfleftident{type\bnfunderscore{}qualifier\bnfunderscore{}list} & \bnfdef &
  \begin{minipage}[t]{\rulerhs}
  \raggedright
  \bnfrightident{type\bnfunderscore{}qualifier} \bnflist{\bnfrightident{type\bnfunderscore{}qualifier}}
  \end{minipage}\\

\bnfleftident{parameter\bnfunderscore{}type\bnfunderscore{}list} & \bnfdef &
  \begin{minipage}[t]{\rulerhs}
  \raggedright
  \bnfrightident{parameter\bnfunderscore{}list}
  \end{minipage}\\
                    &  \bnfor{}  &
  \begin{minipage}[t]{\rulerhs}
  \raggedright
  \bnfrightident{DT\bnfunderscore{}TOK\bnfunderscore{}ELLIPSIS}
  \end{minipage}\\
                    &  \bnfor{}  &
  \begin{minipage}[t]{\rulerhs}
  \raggedright
  \bnfrightident{parameter\bnfunderscore{}list} \bnfrightident{DT\bnfunderscore{}TOK\bnfunderscore{}COMMA} \bnfrightident{DT\bnfunderscore{}TOK\bnfunderscore{}ELLIPSIS}
  \end{minipage}\\

\bnfleftident{parameter\bnfunderscore{}list} & \bnfdef &
  \begin{minipage}[t]{\rulerhs}
  \raggedright
  \bnfrightident{parameter\bnfunderscore{}declaration} \bnflist{\bnfrightident{DT\bnfunderscore{}TOK\bnfunderscore{}COMMA} \bnfrightident{parameter\bnfunderscore{}declaration}}
  \end{minipage}\\

\bnfleftident{parameter\bnfunderscore{}declaration} & \bnfdef &
  \begin{minipage}[t]{\rulerhs}
  \raggedright
  \bnfrightident{parameter\bnfunderscore{}declaration\bnfunderscore{}specifiers} \bnfopt{\bnfrightident{declarator} \bnfor{} \bnfrightident{abstract\bnfunderscore{}declarator}}
  \end{minipage}\\

\bnfleftident{type\bnfunderscore{}name} & \bnfdef &
  \begin{minipage}[t]{\rulerhs}
  \raggedright
  \bnfrightident{specifier\bnfunderscore{}qualifier\bnfunderscore{}list} \bnfopt{\bnfrightident{abstract\bnfunderscore{}declarator}}
  \end{minipage}\\

\bnfleftident{abstract\bnfunderscore{}declarator} & \bnfdef &
  \begin{minipage}[t]{\rulerhs}
  \raggedright
  \bnfrightident{pointer}
  \end{minipage}\\
                    &  \bnfor{}  &
  \begin{minipage}[t]{\rulerhs}
  \raggedright
  \bnfrightident{direct\bnfunderscore{}abstract\bnfunderscore{}declarator}
  \end{minipage}\\
                    &  \bnfor{}  &
  \begin{minipage}[t]{\rulerhs}
  \raggedright
  \bnfrightident{pointer} \bnfrightident{direct\bnfunderscore{}abstract\bnfunderscore{}declarator}
  \end{minipage}\\

\bnfleftident{direct\bnfunderscore{}abstract\bnfunderscore{}declarator} & \bnfdef &
  \begin{minipage}[t]{\rulerhs}
  \raggedright
  \bnfgroup{\bnfrightident{lparen} \bnfrightident{abstract\bnfunderscore{}declarator} \bnfrightident{DT\bnfunderscore{}TOK\bnfunderscore{}RPAR} \bnfor{} \bnfrightident{array} \bnfor{} \bnfrightident{function}} \bnflist{\bnfrightident{array} \bnfor{} \bnfrightident{function}}
  \end{minipage}\\

\bnfleftident{array} & \bnfdef &
  \begin{minipage}[t]{\rulerhs}
  \raggedright
  \bnfrightident{DT\bnfunderscore{}TOK\bnfunderscore{}LBRAC} \bnfrightident{array\bnfunderscore{}parameters} \bnfrightident{DT\bnfunderscore{}TOK\bnfunderscore{}RBRAC}
  \end{minipage}\\

\bnfleftident{array\bnfunderscore{}parameters} & \bnfdef &
  \begin{minipage}[t]{\rulerhs}
  \raggedright
  \bnfopt{\bnfrightident{constant\bnfunderscore{}expression} \bnfor{} \bnfrightident{parameter\bnfunderscore{}type\bnfunderscore{}list}}
  \end{minipage}\\

\bnfleftident{function} & \bnfdef &
  \begin{minipage}[t]{\rulerhs}
  \raggedright
  \bnfrightident{DT\bnfunderscore{}TOK\bnfunderscore{}LPAR} \bnfrightident{function\bnfunderscore{}parameters} \bnfrightident{DT\bnfunderscore{}TOK\bnfunderscore{}RPAR}
  \end{minipage}\\

\bnfleftident{function\bnfunderscore{}parameters} & \bnfdef &
  \begin{minipage}[t]{\rulerhs}
  \raggedright
  \bnfopt{\bnfrightident{parameter\bnfunderscore{}type\bnfunderscore{}list}}
  \end{minipage}\\

\bnfleftident{dtrace\bnfunderscore{}keyword\bnfunderscore{}ident} & \bnfdef &
  \begin{minipage}[t]{\rulerhs}
  \raggedright
  \bnfrightident{DT\bnfunderscore{}KEY\bnfunderscore{}PROBE}
  \end{minipage}\\
                     &  \bnfor{}  &
  \begin{minipage}[t]{\rulerhs}
  \raggedright
  \bnfrightident{DT\bnfunderscore{}KEY\bnfunderscore{}PROVIDER}
  \end{minipage}\\
                     &  \bnfor{}  &
  \begin{minipage}[t]{\rulerhs}
  \raggedright
  \bnfrightident{DT\bnfunderscore{}KEY\bnfunderscore{}SELF}
  \end{minipage}\\
                     &  \bnfor{}  &
  \begin{minipage}[t]{\rulerhs}
  \raggedright
  \bnfrightident{DT\bnfunderscore{}KEY\bnfunderscore{}STRING}
  \end{minipage}\\
                     &  \bnfor{}  &
  \begin{minipage}[t]{\rulerhs}
  \raggedright
  \bnfrightident{DT\bnfunderscore{}TOK\bnfunderscore{}STRINGOF}
  \end{minipage}\\
                     &  \bnfor{}  &
  \begin{minipage}[t]{\rulerhs}
  \raggedright
  \bnfrightident{DT\bnfunderscore{}KEY\bnfunderscore{}USERLAND}
  \end{minipage}\\
                     &  \bnfor{}  &
  \begin{minipage}[t]{\rulerhs}
  \raggedright
  \bnfrightident{DT\bnfunderscore{}TOK\bnfunderscore{}XLATE}
  \end{minipage}\\
                     &  \bnfor{}  &
  \begin{minipage}[t]{\rulerhs}
  \raggedright
  \bnfrightident{DT\bnfunderscore{}KEY\bnfunderscore{}XLATOR}
  \end{minipage}\\

\end{longtable}
}


\ebnftable

\section{Safety}
\label{sec:safety}

The D language will look familiar to anyone who has programmed in C or
its close linguistic relatives, but in order to provide certain
safety guarantees there are features of C like languages that are
missing from D.  The most obviously missing feature is the lack of any
sort of looping mechanism.  Once they are compiled into byte code D
scripts are loaded into the kernel where they run to completion.  A
script that was allowed to loop might, due to error or intent, loop
forever, causing the operating system kernel to lock up and require a
system reset.  D lacks any form of loops to prevent such errors from
occurring.

By default, OpenDTrace runs in a mode where memory can be read but not
written by D language scripts.  A command line option to the \texttt{dtrace} (1)
program, \texttt{-w}, puts OpenDTrace into destructive mode, where both reads
and writes are possible.  Although destructive mode is not a feaure of
the D language itself, it is an important part of the system's overall
commitment to safety.

% Are limitations of table sizes and the like which are pretty clearly
% in place for safety to be included here?  Values such as:

% #define	DIF_INTOFF_MAX		0xffff	/* highest integer table offset */
% #define	DIF_STROFF_MAX		0xffff	/* highest string table offset */
% #define	DIF_REGISTER_MAX	0xff	/* highest register number */
% #define	DIF_VARIABLE_MAX	0xffff	/* highest variable identifier */
% #define	DIF_SUBROUTINE_MAX	0xffff	/* highest subroutine code */
% #define	DIF_VAR_ARRAY_MAX	0x00ff	/* highest numbered array variable */
% #define	DIF_VAR_OTHER_MAX	0xffff	/* highest numbered scalar or assc */

% from dtrace.h clearly limit the amount of resources a script can ask
% for and although generous for a scripting language clearly exist to
% prevent D scripts from asking for too much.

\section{Variables}
\label{sec:d-variables}
DTrace implements three different scopes of variables: global,
thread-local and clause-local. Global variables are visible to every
probe and across all threads, allowing the user to write scripts that
carry state across multiple threads should it be
necessary. Thread-local variables are only visible within a single
software thread, they are represented in source code as prefixed with
\texttt{self->}. Clause-local variables are implemented on a
per-thread basis and are identified by the prefix
\texttt{this->}. Clause-local variables should be initialised in each
probe before their use, as the value is otherwise considered
undefined.

\subsection{Global variables}
\label{subsec:global-variables}

Any variable introduced in a D script that is not declared as part of
a \verb|this->| or \verb|self->| is considered to be global in scope,
meaning that it can be accessed from any action associated with a
probe when a set of probes are simultaneously activated.  Global
variables are allocated and instantiated when they are first
referenced.

\subsection{Thread-local variables}
\label{subsec:thread-local-variables}

\subsection{Clause-local variables}
\label{subsec:clause-local-variables}

\section{Multithreading}

% XXX: This is very vague because saying that DTrace guarantees
% something is a very bold statement. For example, saying that it
% guarantees thread safety when creating dynamic variables would be a
% little over the top, as this has not been verified at all. We should
% think about this a bit more.

When tracing, DTrace guarantees that it can not be preempted inside of
the \function{dtrace_probe()} function, but it does not guarantee that
everything in the executing DIF will be thread-safe. DTrace does not
allow access to locking primitives, because a programming error might
violate the safety guarantees that OpenDTrace was designed to provide.

\subsection{Global variables}

Global variables are not stored in thread-local storage, while
thread-local and clause-local variables are. In a multithreaded
environment, global variables should be used sparingly. While it is
evident that a value stored in a global variable may be overwritten by
another probe at any time, there is more subtle behavior at
hand. Consider the script in Figure~\ref{fig:global-var-usage}. \newline

\begin{figure}
  \begin{lstlisting}
dtrace:::BEGIN
{
    num_syscalls = 0;
}

syscall:::entry
{
    num_syscalls++;
}
    
dtrace:::END
{
    printf("Number of syscalls: %d\n", num_syscalls);
}
  \end{lstlisting}
  \caption{Global Variable Usage}
  \label{fig:global-var-usage}
\end{figure}

\noindent
Because DIF performs all of its operations on a virtual machine's
registers as opposed to variables in memory, the ++ operator is not
atomic. Compiling the \texttt{syscall:::entry} clause from
Figure~\ref{fig:global-var-usage} generates the DIF shown in
Figure~\ref{fig:dif-asm}.  This DIF section is safe, as long as the
\texttt{num\_syscalls} variable is not visible from any other thread. If it is
visible and accessible from another thread, it suffers from a race
condition which results in wrong information being given to the
user. The race condition is shown in Figure~\ref{fig:race}. \newline

\begin{figure}
\begin{lstlisting}
ldgs %r1, num_syscalls /* Load the current value into %r1 */
setx %r2, inttab[0]    /* Load 1 into %r2 */
add  %r2, %r1, %r2     /* Add %r1 and %r2 and store into %r2 */
stgs %r2, num_syscalls /* Store the result back into num_syscalls */
\end{lstlisting}
\caption{DIF Assembly}
  \label{fig:dif-asm}
\end{figure}

\begin{figure}
  \begin{lstlisting}
                  Thread 1                     Thread 2
            ldgs %r1, num_syscalls
                                          ldgs %r3, num_syscalls
                                          setx %r4, inttab[0]
                                          add  %r4, %r3, %r4
            setx %r2, inttab[0]
            add  %r2, %r1, %r2
            stgs %r2, num_syscalls
                                          stgs %r4, num_syscalls
  \end{lstlisting}
  \caption{Race Condition}
  \label{fig:race}
\end{figure}

\noindent
It is clear that the value in the \registerop{r2} register will be lost because
the register \registerop{r4} is stored to the same location afterwards. It is
worth noting that this behaviour is not observed because the thread was
preempted, but simply by the fact that DTrace does not guarantee any ordering
outside of each CPU core. This behaviour applies to all of the operations
performed on global variables and as a result, they should only be used in
probes that are guaranteed to fire on a single thread. \newline

\noindent
Often the desired behaviour with global variables can be achieved through
aggregations. The script shown in Figure~\ref{fig:global-var-usage} ought to be
written using an equivalent aggregation function, as shown in
Figure~\ref{fig:avoiding-the-race}.

\begin{figure}
  \begin{lstlisting}
syscall:::entry
{
    @num_syscalls = count();
}

dtrace:::END
{
    printa(@num_syscalls);
}
  \end{lstlisting}
  \caption{Avoiding the race condition}
  \label{fig:avoiding-the-race}
\end{figure}

\subsection{Thread-local variables}

As mentioned in Subsection~\ref{subsec:thread-local-variables}, thread-local
variables are only visible within a single thread.

\subsection{Clause-local variables}

\section{Aggregations}
\label{sec:aggregations}

The ability to aggregate data during data collection, and to then
process the data via several types of statistical analysis, is one of
the key features of OpenDTrace.  The data for an aggregation is
collected, like all other trace data, by the kernel, while the data
processing is carried out in user space by the \emph{libdtrace}
library functions.

\begin{table}
  \centering
  \begin{tabular}{l|l}
    Function & Explanation\\
    \hline
    count(arg) & x = x + arg (arg == 1 by default)\\
    min() & x = minimum of all values seen\\
    max() & x = maximum o all values seen\\
    avg() & x = sum(all values seen)/len(value list)\\
    sum() & x = x + val\\
    stddev() & \\
    quantize() & \\
    lquantize() & \\
    llquantize() &
  \end{tabular}
  \caption{Aggregation Functions}
  \label{tab:agg-func}
\end{table}

There are nine (9) aggregation functions, which are listed in
Table~\ref{tab:agg-func}.

