The D language is a language inspired by the AWK programming language
\cite{Aho:1987:APL:29361} and the C programming language
\cite{Kernighan:1988}\cite{DTrace2004}. In this chapter, we give a formal
definition of the D programming language that is deployed in OpenDTrace, as well
as elaborate on its properties in multithreaded environments.

%%% Local Variables:
%%% mode: latex
%%% TeX-master: "dtrace-specification"
%%% End:

\section{Grammar definition}

% TODO: Here we use something like the Extended Backus-Naur Form (EBNF) to
% describe the D language syntactically.

\section{Safety}

% TODO: Explain what the safety guarantees of the D language are, so that we can
% refer to them later on.

\section{Aggregations}

% TODO: Write the section on aggregations.

\section{Variables}
\label{sec:d-variables}
DTrace implements three different scopes of variables. The first scope of D
variables is the global scope. These are variables that are visible to every
probe and across all threads, allowing the user to write scripts that carry
state across multiple threads should it be necessary. The second is the
thread-local scope. These variables are prefixed with \texttt{self->} and are
only visible within a single software thread. The last scope of variables that D
implements are clause-local variables. These variables are implemented
per-thread and are identified by the prefix \texttt{this->}. Clause-local
variables should be initialised in each probe before their use, as the value is
otherwise considered undefined.

% TODO: We should fill this out with explanations about each type of variables.
% This is only for *variables*, NOT aggregations. They go in their own section.

\subsection{Global variables}

\subsection{Thread-local variables}
\label{subsec:thread-local-variables}

\subsection{Clause-local variables}

\section{Multithreading}

Software threads are a key component of a modern operating system and were the
reason many design decisions have been made in DTrace. These threads may be
supported by having many processors implemented in hardware, preemption of a
single processor or through preemption of multiple processors using the
scheduler.

\subsection{DTrace guarantees}

% XXX: This is very vague because saying that DTrace guarantees something is
% a very bold statement. For example, saying that it guarantees thread safety
% when creating dynamic variables would be a little over the top, as this has
% not been verified at all. We should think about this a bit more.

When tracing, DTrace guarantees that it can not be preempted inside of the
\function{dtrace_probe} function, but it does not guarantee that everything in
the executing DIF will be thread-safe. In addition to that, DTrace does not
allow access to locking primitives. This is because a programming error might
violate the safety guarantees that DTrace was designed to provide the user with.

\subsection{Global variables}

Global variables are not stored in thread-local storage,
while thread-local and clause-local variables are. This means that in a
multithreaded environment, global variables should be used sparingly. While it
is evident that storing to a global variable may be overwritten with another
probe that stores something else in the same one, there is more subtle behaviour
at hand. Consider the following example:

\begin{verbatim}
dtrace:::BEGIN
{
    num_syscalls = 0;
}

syscall:::entry
{
    num_syscalls++;
}

dtrace:::END
{
    printf("Number of syscalls: %d\n", num_syscalls);
}
\end{verbatim}

\noindent
Because DIF performs all of it operations on virtual machine's registers as
opposed to variables in memory, the ++ operator is not atomic. When we compile
the syscall:::entry clause, we get the following DIF output:

\begin{verbatim}
ldgs %r1, num_syscalls /* Load the current value into %r1 */
setx %r2, inttab[0]    /* Load 1 into %r2 */
add  %r2, %r1, %r2     /* Add %r1 and %r2 and store into %r2 */
stgs %r2, num_syscalls /* Store the result back into num_syscalls */
\end{verbatim}

\noindent
This DIF section is safe, as long as the num\_syscalls variable is not visible
from any other thread. If it is visible and accessible from another thread, it
suffers from a race condition which results in wrong information being given to
the user. Consider the following:

\begin{verbatim}
      Thread 1                     Thread 2
ldgs %r1, num_syscalls
                              ldgs %r3, num_syscalls
                              setx %r4, inttab[0]
                              add  %r4, %r3, %r4
setx %r2, inttab[0]
add  %r2, %r1, %r2
stgs %r2, num_syscalls
                              stgs %r4, num_syscalls
\end{verbatim}

\noindent
It is clear that the value in the \registerop{r2} register will be lost because
the register \registerop{r4} is stored to the same location afterwards. It is
worth noting that this behaviour is not observed because the thread was
preempted, but simply by the fact that DTrace does not guarantee any ordering
outside of each CPU core. This behaviour applies to all of the operations
performed on global variables and as a result, they should only be used in
probes that are guaranteed to fire on a single thread.

\noindent
Often the desired behaviour with global variables can be achieved through
aggregations. The above script ought to be written in the following way in order
to be thread-safe:

\begin{verbatim}
syscall:::entry
{
    @num_syscalls = count();
}

dtrace:::END
{
    printa(@num_syscalls);
}
\end{verbatim}

\subsection{Thread-local variables}

As mentioned in Subsection~\ref{subsec:thread-local-variables}, thread-local
variables are only visible within a single thread.

\subsection{Clause-local variables}
