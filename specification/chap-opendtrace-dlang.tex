The D language is a language inspired by the AWK programming language
\cite{Aho:1987:APL:29361} and the C programming language
\cite{Kernighan:1988}\cite{DTrace2004}. In this chapter, we give a
formal definition of the D programming language that is a part of
OpenDTrace, as well as elaborate on its properties in multithreaded
environments.

\section{Grammar definition}
\label{sec:grammar}

% TODO: Here we use something like the Extended Backus-Naur Form (EBNF) to
% describe the D language syntactically.

\section{Safety}
\label{sec:safety}

% TODO: Explain what the safety guarantees of the D language are, so that we can
% refer to them later on.

\section{Aggregations}
\label{sec:aggregations}

% TODO: Write the section on aggregations.

\section{Variables}
\label{sec:d-variables}
DTrace implements three different scopes of variables: global,
thread-local and clause-local. Global variables are visible to every
probe and across all threads, allowing the user to write scripts that
carry state across multiple threads should it be
necessary. Thread-local variables are only visible within a single
software thread, they are represented in source code as prefixed with
\texttt{self->}. Clause-local variables are implemented on a
per-thread basis and are identified by the prefix
\texttt{this->}. Clause-local variables should be initialised in each
probe before their use, as the value is otherwise considered
undefined.

% TODO: We should fill this out with explanations about each type of variables.
% This is only for *variables*, NOT aggregations. They go in their own section.

\subsection{Global variables}

\subsection{Thread-local variables}
\label{subsec:thread-local-variables}

\subsection{Clause-local variables}

\section{Multithreading}

% XXX: This is very vague because saying that DTrace guarantees
% something is a very bold statement. For example, saying that it
% guarantees thread safety when creating dynamic variables would be a
% little over the top, as this has not been verified at all. We should
% think about this a bit more.

When tracing, DTrace guarantees that it can not be preempted inside of
the \function{dtrace_probe} function, but it does not guarantee that
everything in the executing DIF will be thread-safe. DTrace does not
allow access to locking primitives, because a programming error might
violate the safety guarantees that OpenDTrace was designed to provide.

\subsection{Global variables}

Global variables are not stored in thread-local storage, while
thread-local and clause-local variables are. In a multithreaded
environment, global variables should be used sparingly. While it is
evident that a value stored in a global variable may be overwritten by
another probe at any time, there is more subtle behavior at
hand. Consider the following example:

\begin{figure}
  \begin{lstlisting}
    dtrace:::BEGIN
    {
      num_syscalls = 0;
    }
    
    syscall:::entry
    {
      num_syscalls++;
    }
    
    dtrace:::END
    {
      printf("Number of syscalls: %d\n", num_syscalls);
    }
  \end{lstlisting}
  \caption{Global Variable Usage}
  \label{fig:global-var-usage}
\end{figure}

\noindent
Because DIF performs all of sit operations on a virtual machine's
registers as opposed to variables in memory, the ++ operator is not
atomic. When we compile the syscall:::entry clause, we get the
following DIF output:

\begin{figure}
\begin{lstlisting}
ldgs %r1, num_syscalls /* Load the current value into %r1 */
setx %r2, inttab[0]    /* Load 1 into %r2 */
add  %r2, %r1, %r2     /* Add %r1 and %r2 and store into %r2 */
stgs %r2, num_syscalls /* Store the result back into num_syscalls */
\end{lstlisting}
\caption{DIF Assembly}
  \label{fig:dif-asm}
\end{figure}

\noindent
This DIF section is safe, as long as the num\_syscalls variable is not visible
from any other thread. If it is visible and accessible from another thread, it
suffers from a race condition which results in wrong information being given to
the user. Consider the following:

\begin{figure}
  \begin{lstlisting}
      Thread 1                     Thread 2
ldgs %r1, num_syscalls
                              ldgs %r3, num_syscalls
                              setx %r4, inttab[0]
                              add  %r4, %r3, %r4
setx %r2, inttab[0]
add  %r2, %r1, %r2
stgs %r2, num_syscalls
                              stgs %r4, num_syscalls
  \end{lstlisting}
  \caption{Race Condition}
  \label{fig:race}
\end{figure}

\noindent
It is clear that the value in the \registerop{r2} register will be lost because
the register \registerop{r4} is stored to the same location afterwards. It is
worth noting that this behaviour is not observed because the thread was
preempted, but simply by the fact that DTrace does not guarantee any ordering
outside of each CPU core. This behaviour applies to all of the operations
performed on global variables and as a result, they should only be used in
probes that are guaranteed to fire on a single thread.

\noindent
Often the desired behaviour with global variables can be achieved through
aggregations. The above script ought to be written in the following way in order
to be thread-safe:

\begin{figure}
  \begin{lstlisting}
    syscall:::entry
    {
      @num_syscalls = count();
    }

    dtrace:::END
    {
      printa(@num_syscalls);
    }
  \end{lstlisting}
  \caption{Avoiding the race condition}
  \label{fig:avoiding-the-race}
\end{figure}

\subsection{Thread-local variables}

As mentioned in Subsection~\ref{subsec:thread-local-variables}, thread-local
variables are only visible within a single thread.

\subsection{Clause-local variables}
