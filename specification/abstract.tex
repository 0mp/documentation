\section*{Abstract}

OpenDTrace is a dynamic tracing facility offering full-system instrumentation,
a high degree of flexibility, and portable semantics across a range of
operating systems.
Originally designed and implemented by Sun Microsystems (now Oracle),
user-facing aspects of OpenDTrace, such as the D language and
command-line tools,
are well defined and documented.
However, OpenDTrace's internal formats --
the DTrace Intermediate Format (DIF) and
DTrace Object Format (DOF) -- have primarily been documented through source
code comments rather than a structured specification.
This technical report specifies these formats in order to better support the
development of more comprehensive tests, new underlying execution substrates
(such as just-in-time compilation), and future extensions.
Throughout this report we use the name \texttt{OpenDTrace} to refer
to the open source project but retain the name \texttt{DTrace} when
referring to data structures such as the DTrace Intermediate Format.
OpenDTrace builds upon the foundations of the original DTrace code but
provides new features, which were not present in the original.
This document acts as a single source of truth for the current state
of OpenDTrace as it is currently implemented and deployed.
%DIF is a RISC bytecode into which D scripts are compiled, describing the
%specific executable actions in a script.
%DOF is the container format containing a set of headers and sections for
%variables, probe implementations, constants, and other content required to
%represent a complete script.
