\section*{Abstract}

OpenDTrace is a dynamic tracing facility offering full-system instrumentation,
a high degree of flexibility, and portable semantics across a range of
operating systems.
Originally designed and implemented by Sun Microsystems (now Oracle),
user-facing aspects of OpenDTrace, such as the D language and
command-line tools,
are well defined and documented.
However, OpenDTrace's internal formats --
the DTrace Intermediate Format (DIF),
DTrace Object Format (DOF) and Compact C Trace Format (CTF)
-- have primarily been documented through
source-code comments rather than a structured specification.
This technical report specifies these formats in order to better support the
development of more comprehensive tests, new underlying execution substrates
(such as just-in-time compilation), and future extensions.
We not only cover the data structures present in \texttt{OpenDTrace}
but also include a complete reference of all the low level
instructions that are used by the byte code interpreter,
all the built in global variables and subroutines.
Our goal with this report is to provide not only a list of
what is present in the code at any point in time, the \emph{what},
but also explanations of how the system works as a whole,
the \emph{how}, and motivations for various design decisions
that have been made along the way, the \emph{why}.
Throughout this report we use the name \texttt{OpenDTrace} to refer
to the open-source project but retain the name \texttt{DTrace} when
referring to data structures such as the DTrace Intermediate Format.
OpenDTrace builds upon the foundations of the original DTrace code but
provides new features, which were not present in the original.
This document acts as a single source of truth for the current state
of OpenDTrace as it is currently implemented and deployed.
%DIF is a RISC bytecode into which D scripts are compiled, describing the
%specific executable actions in a script.
%DOF is the container format containing a set of headers and sections for
%variables, probe implementations, constants, and other content required to
%represent a complete script.
