The D language provides a set of built-in global sub-routines that are
available to D scripts from within probe context.  The built-in global
sub-routines provide commonly used functions present in the C and C++
language, such as \verb|printf| and \verb|inet_ntoa| but which can be
called from probe context without risking system safety.

\section{Subroutine calling mechanism}

Every D subroutine exists simultaneously as a part of the D scripting
language as well as code in the operating system kernel.  D subroutines
must be coded in such a way that they do not depend on kernel
services, such as memory allocation, or mutual exclusion primitives,
which might be inspected as part of a tracing session.

When a subroutine appears in a D script it is the responsibility of
the D code generator to turn the subroutine and its arguments into DIF
to be passed into the kernel for execution.  All subroutines and their
arguments are described in a single structure, \verb|_dtrace_globals|
in the \verb|libdtrace| library.  Each subroutine has a string name,
an identifier type, a set of flags, a numeric ID, a small set of
functions, and an argument list.  The argument list defined in the
\verb|_dtrace_globals| structure is a string which is processed via
the \verb|dt_open| function into a relevant set of argument types.

The \verb|DT_INSTR_CALL| instruction is used by the D code generator
to generate a subroutine call.  Each call instruction has an
identifier, which is the name of the subroutine, and a set of
registers which contain the arguments to the call.  The arguments in
the registers are placed into D's tuple stack for use in kernel
context.

Once in the kernel all subroutines are executed from a central
function, \verb|dtrace_dif_subr| which dispatches calls to various,
OpenDTrace private versions of the functions required.  All of the
subroutines are implemented using the register state passed into the
kernel and represented on the tuple stack, or local variables, in
order to avoid using any of the kernel's memory allocation subsystem.
All subroutines are executed from probe context (\verb|dtrace_probe|)
via the \verb|dt_dif_emulate| routine.

\section{Subroutine list}

\begin{table}
\label{tab:subroutines}
\begin{center}
\begin{tabular}{llp{9cm}}
\toprule
  Name & Number & Description \\
\midrule
  \hyperref[subr:rand]{\subroutine{rand}} & 0 & Get random \\
  \hyperref[subr:mutex-owned]{\subroutine{mutex_owned}} & 1 &
    Query whether current thread is mutex owner \\
  \hyperref[subr:mutex-owner]{\subroutine{mutex_owner}} & 2 &
    Retrieve mutex owner \\
  \hyperref[subr:mutex-type-adaptive]{\subroutine{mutex_type_adaptive}} & 3 &
    Query if mutex is adaptive \\
  \hyperref[subr:mutex-type-spin]{\subroutine{mutex_type_spin}} & 4 &
    Query if mutex is a spinlock \\
  \hyperref[subr:rw-read-held]{\subroutine{rw_read_held}} & 5 &
    Query whether rwlock is held for read \\
  \hyperref[subr:rw-write-held]{\subroutine{rw_write_held}} & 6 &
    Query whether current thread holds rwlock for write \\
  \hyperref[subr:rw-iswriter]{\subroutine{rw_iswriter}} & 7 &
    Query whether rwlock is held for write \\
  \hyperref[subr:copyin]{\subroutine{copyin}} & 8 &
    Copy in data from userspace \\
  \hyperref[subr:copyinstr]{\subroutine{copyinstr}} & 9 &
    Copy in string from userspace \\
  \hyperref[subr:copyoutmbuf]{\subroutine{copyoutmbuf}} & 9 &
    Copy data from an mbuf chain\\
  \hyperref[subr:speculation]{\subroutine{speculation}} & 10 & \\
  \hyperref[subr:progenyof]{\subroutine{progenyof}} & 11 &
    Query whether this process the child of a particular PID\\
  \hyperref[subr:strlen]{\subroutine{strlen}} & 12 &
    Return the length of a string\\
  \hyperref[subr:copyout]{\subroutine{copyout}} & 13 &
  \\
  \hyperref[subr:copyoutstr]{\subroutine{copyoutstr}} & 14 & \\
  \hyperref[subr:alloca]{\subroutine{alloca}} & 15 &
  allocate temporary space \\
  \hyperref[subr:bcopy]{\subroutine{bcopy}} & 16 &
  copy bytes from source to destination bounded by a size\\
  \hyperref[subr:copyinto]{\subroutine{copyinto}} & 17 & \\
  \hyperref[subr:msgdsize]{\subroutine{msgdsize}} & 18 &
  return the size data in a STREAMS message block \\
  \hyperref[subr:msgsize]{\subroutine{msgsize}} & 19 &
  return the size data in a  STREAMS message block\\
  \hyperref[subr:getmajor]{\subroutine{getmajor}} & 20 &
  return major device number\\
  \hyperref[subr:getminor]{\subroutine{getminor}} & 21 &
  return minor device number\\
  \hyperref[subr:ddi-pathname]{\subroutine{ddi_pathname}} & 22 &
  look up device driver by name\\
  \hyperref[subr:strjoin]{\subroutine{strjoin}} & 23  &
  join two strings and return the result\\
  \hyperref[subr:lltostr]{\subroutine{lltostr}} & 24 &
  convert a long long (64 bit) value to a string\\
  \hyperref[subr:basename]{\subroutine{basename}} & 25 &
  return the file name portion of a pathname\\
  \hyperref[subr:dirname]{\subroutine{dirname}} & 26 &
  return the directory component of a pathname\\
  \hyperref[subr:cleanpath]{\subroutine{cleanpath}} & 27 &
  return the cleaned up pathname\\
  \hyperref[subr:strchr]{\subroutine{strchr}} & 28 &
  locate a character in a string\\
\bottomrule
\end{tabular}
\end{center}
\caption{DTrace Subroutines (Part 1)}
\end{table}

\begin{table}
\begin{center}
\begin{tabular}{llp{9cm}}
\toprule
  Name & Number & Description \\
\midrule
  \hyperref[subr:strrchr]{\subroutine{strrchr}} & 29 &
  reverse search a string\\
  \hyperref[subr:strstr]{\subroutine{strstr}} & 30 &
  locate a string within a string\\
  \hyperref[subr:strtok]{\subroutine{strtok}} & 31 &
  string tokenizing subroutine\\
  \hyperref[subr:substr]{\subroutine{substr}} & 32 &
  return a sub string of a string\\
  \hyperref[subr:index]{\subroutine{index}} & 33 &
  return the byte position of a character in a string\\
  \hyperref[subr:rindex]{\subroutine{rindex}} & 34 &
  locate the last matching character in a a string\\
  \hyperref[subr:htons]{\subroutine{htons}} & 35 &
  convert a short (16 bit) value from host to network byte order\\
  \hyperref[subr:htonl]{\subroutine{htonl}} & 36 &
  convert  a  long (32 bit)  value from host to network byte order\\
  \hyperref[subr:htonll]{\subroutine{htonll}} & 37 &
  convert a long long (64 bit) value from host to  network byte order\\
  \hyperref[subr:ntohs]{\subroutine{ntohs}} & 38 &
  convert a short (16 bit) value from network to host byte order\\
  \hyperref[subr:ntohl]{\subroutine{ntohl}} & 39 &
  convert long (32 bit) value from network to host byte order\\
  \hyperref[subr:ntohll]{\subroutine{ntohll}} & 40 &
  convert a long long (64 bit) value from network to host byte order\\
  \hyperref[subr:inet-ntop]{\subroutine{inet_ntop}} & 41 &
  convert an arbitrary Internet address to a string\\
  \hyperref[subr:inet-ntoa]{\subroutine{inet_ntoa}} & 42 &
  convert a 32 bit IPv4 address to a string\\
  \hyperref[subr:inet-ntoa6]{\subroutine{inet_ntoa6}} & 43 &
  convert a 128 bit IPv6 address to a string\\
  \hyperref[subr:toupper]{\subroutine{toupper}} & 44 &
  convert a string to upper case\\
  \hyperref[subr:tolower]{\subroutine{tolower}} & 45 &
  convert a string to lower case\\
  \hyperref[subr:memref]{\subroutine{memref}} & 46 &
  return scratch memory\\
  \hyperref[subr:sx-shared-held]{\subroutine{sx_shared_held}} & 48 &
  Is this shared mutex currently held by a reader\\
  \hyperref[subr:sx-exclusive-held]{\subroutine{sx_exclusive_held}} & 49 &
  Is this sx mutex held exclusively\\
  \hyperref[subr:sx-isexclusive]{\subroutine{sx_isexclusive}} & 50 & 
  Is the current thread the only one to hold a shared mutex\\
  \hyperref[subr:memstr]{\subroutine{memstr}} & 51 &
  convert NULL separated strings to one string\\
  \hyperref[subr:getf]{\subroutine{getf}} & 52 &
  Return a file structure based on a file descriptor\\
  \hyperref[subr:json]{\subroutine{json}} & 53 &
  extract a single value from a JSON string\\
  \hyperref[subr:strtoll]{\subroutine{strtoll}} & 54 &
  convert a string representing a number to a long long (64 bit) value\\
  \hyperref[subr:random]{\subroutine{random}} & 55 &
  return a better pseudo-random number than rand()\\
  \hyperref[subr:uuidstr]{\subroutine{uuidstr}} & 56 &
  convert a UUID to a string\\
\bottomrule
\end{tabular}
\end{center}
\caption{DTrace Subroutines (Part 2)}
\end{table}

\section{Subroutine reference}

\clearpage
\phantomsection
\addcontentsline{toc}{subsection}{DUMMY}
\label{subr:alloca}
\subsection*{alloca: allocate temporary space}

\subsubsection*{Calling convention}

\begin{description}
\item[\registerop{rd}] Pointer to allocated data or NULL.
\end{description}

\subsubsection*{Description}

The \subroutine{alloca} subroutine allocates scratch space in the
DTrace state machine structure.  Although this subroutine does not
allocate space on the process stack, it does act similarly to the
alloca macro, in that the space disappears without an explicit call to
a \subroutine{free} routine, once the DTrace machine state structure
is deallocated.

\subsubsection*{Pseudocode}

\begin{verbatim}
\end{verbatim}

\subsubsection*{Constraints}

\subsubsection*{Failure modes}

This subroutine has no run-time failure modes beyond its constraints.

\clearpage
\phantomsection
\addcontentsline{toc}{subsection}{basename}
\label{subr:basename}
\subsection*{basename: return the file name portion of a pathname}

\subsubsection*{Calling convention}

\begin{description}
\item[\registerop{rd}] A pointer to a scratch space string containing
  the filename.
\end{description}

\subsubsection*{Description}

The \subroutine{basename} subroutine takes a single string argument,
containing a path, and returns a pointer to the file name portion of
the supplied string.  The space for the resulting string is contained
in the DTrace machine state structure, \verb|mstate| which is
automatically de-allocated.
\subsubsection*{Pseudocode}

\begin{verbatim}
\end{verbatim}

\subsubsection*{Constraints}

\subsubsection*{Failure modes}

This subroutine has no run-time failure modes beyond its constraints.

\clearpage
\phantomsection
\addcontentsline{toc}{subsection}{bcopy}
\label{subr:bcopy}
\subsection*{bcopy: copy bytes from source to destination bounded by a
size}

\subsubsection*{Calling convention}

\begin{description}
\item[\registerop{rd}] VOID
\end{description}

\subsubsection*{Description}

The \subroutine{bcopy} subroutine copies bytes from a source pointer
to a destination pointer, within the DTrace machine state scratch
region, up to the size supplied in the third argument.
\subsubsection*{Pseudocode}

\begin{verbatim}
\end{verbatim}

\subsubsection*{Constraints}

\subsubsection*{Failure modes}

This subroutine has no run-time failure modes beyond its constraints.

\clearpage
\phantomsection
\addcontentsline{toc}{subsection}{DUMMY}
\label{insn:dummy}
\subsection*{DUMMY: FILL ME IN}

\subsubsection*{Calling convention}

\begin{description}
\item[\registerop{rd}] What goes into the destination regiester
\end{description}

\subsubsection*{Description}

This subroutine 
\subsubsection*{Pseudocode}

\begin{verbatim}
regs[rd] = (getthrtime() * 2416 + 374441) % 1771875
\end{verbatim}

\subsubsection*{Constraints}

\subsubsection*{Failure modes}

This subroutine has no run-time failure modes beyond its constraints.

\clearpage
\phantomsection
\addcontentsline{toc}{subsection}{copyin}
\label{insn:dummy}
\subsection*{copyin: Copy data from user space to kernel space}

\subsubsection*{Calling convention}

\begin{description}
\item[\registerop{rd}] Pointer to kernel data.
\end{description}

\subsubsection*{Description}

The \subroutine{copyin} returns a pointer to a buffer which contains
kernel data copied from the area pointed to by its first argument, up
to the limit denoted by its second argument.
\subsubsection*{Pseudocode}

\begin{verbatim}
\end{verbatim}

\subsubsection*{Constraints}

\subsubsection*{Failure modes}

This subroutine has no run-time failure modes beyond its constraints.

\clearpage
\phantomsection
\addcontentsline{toc}{subsection}{DUMMY}
\label{insn:dummy}
\subsection*{DUMMY: FILL ME IN}

\subsubsection*{Calling convention}

\begin{description}
\item[\registerop{rd}] What goes into the destination regiester
\end{description}

\subsubsection*{Description}

This subroutine 
\subsubsection*{Pseudocode}

\begin{verbatim}
regs[rd] = (getthrtime() * 2416 + 374441) % 1771875
\end{verbatim}

\subsubsection*{Constraints}

\subsubsection*{Failure modes}

This subroutine has no run-time failure modes beyond its constraints.

\clearpage
\phantomsection
\addcontentsline{toc}{subsection}{copyinstr}
\label{insn:dummy}
\subsection*{copyinstr: Copy kernel data as a string}

\subsubsection*{Calling convention}

\begin{description}
\item[\registerop{rd}] Pointer to the returned string.
\end{description}

\subsubsection*{Description}

The \subroutine{copyinstr} subroutine returns a pointer to string of
kernel data which is located at the first argument and bounded by the
second argument.

\subsubsection*{Pseudocode}

\begin{verbatim}
\end{verbatim}

\subsubsection*{Constraints}

\subsubsection*{Failure modes}

This subroutine has no run-time failure modes beyond its constraints.

\clearpage
\phantomsection
\addcontentsline{toc}{subsection}{copyout}
\label{subr:copyout}
\subsection*{copyout: FILL ME IN}

\subsubsection*{Calling convention}

\begin{description}
\item[\registerop{rd}] What goes into the destination regiester
\end{description}

\subsubsection*{Description}

This subroutine 
\subsubsection*{Pseudocode}

\begin{verbatim}
regs[rd] = (getthrtime() * 2416 + 374441) % 1771875
\end{verbatim}

\subsubsection*{Constraints}

\subsubsection*{Failure modes}

This subroutine has no run-time failure modes beyond its constraints.

\clearpage
\phantomsection
\addcontentsline{toc}{subsection}{copyoutstr}
\label{insn:dummy}
\subsection*{copyoutstr: copy data from kernel to user space, as a string}

\subsubsection*{Calling convention}

\begin{description}
\item[\registerop{rd}] VOID
\end{description}

\subsubsection*{Description}

The \subroutine{copyoutstr} subroutine copies data from kernel space
to user space as a string value, bounded by the routine's third argument.
\subsubsection*{Pseudocode}

\begin{verbatim}
\end{verbatim}

\subsubsection*{Constraints}

\subsubsection*{Failure modes}

This subroutine has no run-time failure modes beyond its constraints.

\clearpage
\phantomsection
\addcontentsline{toc}{subsection}{copyoutmbuf}
\label{subr:copyoutmbuf}
\subsection*{copyoutmbuf: copy data from an mbuf chain}

\subsubsection*{Calling convention}

\begin{description}
\item[\registerop{arg0}] pointer to mbuf
\item[\registerop{arg1}] amount of data to copy
\item[\registerop{rd}] pointer to copied data
\end{description}

\subsubsection*{Description}

The \subroutine{copyoutmbuf} subroutine copies data from an mbuf
chain out a destination pointer bounded by a size given in the second
argument.  If the second argument exceeds the size of the data in the
mbuf chain then it is reduced to the correct length.

\subsubsection*{Pseudocode}

\begin{verbatim}
\end{verbatim}

\subsubsection*{Constraints}

The \subroutine{copyoutmbuf} subroutine is only supported on FreeBSD.

\subsubsection*{Failure modes}

This subroutine has no run-time failure modes beyond its constraints.

\clearpage
\phantomsection
\addcontentsline{toc}{subsection}{DUMMY}
\label{insn:dummy}
\subsection*{DUMMY: FILL ME IN}

\subsubsection*{Calling convention}

\begin{description}
\item[\registerop{rd}] What goes into the destination regiester
\end{description}

\subsubsection*{Description}

This subroutine 
\subsubsection*{Pseudocode}

\begin{verbatim}
regs[rd] = (getthrtime() * 2416 + 374441) % 1771875
\end{verbatim}

\subsubsection*{Constraints}

\subsubsection*{Failure modes}

This subroutine has no run-time failure modes beyond its constraints.

\clearpage
\phantomsection
\addcontentsline{toc}{subsection}{DUMMY}
\label{insn:dummy}
\subsection*{DUMMY: FILL ME IN}

\subsubsection*{Calling convention}

\begin{description}
\item[\registerop{rd}] What goes into the destination regiester
\end{description}

\subsubsection*{Description}

This subroutine 
\subsubsection*{Pseudocode}

\begin{verbatim}
regs[rd] = (getthrtime() * 2416 + 374441) % 1771875
\end{verbatim}

\subsubsection*{Constraints}

\subsubsection*{Failure modes}

This subroutine has no run-time failure modes beyond its constraints.

\clearpage
\phantomsection
\addcontentsline{toc}{subsection}{DUMMY}
\label{insn:dummy}
\subsection*{DUMMY: FILL ME IN}

\subsubsection*{Calling convention}

\begin{description}
\item[\registerop{rd}] What goes into the destination regiester
\end{description}

\subsubsection*{Description}

This subroutine 
\subsubsection*{Pseudocode}

\begin{verbatim}
regs[rd] = (getthrtime() * 2416 + 374441) % 1771875
\end{verbatim}

\subsubsection*{Constraints}

\subsubsection*{Failure modes}

This subroutine has no run-time failure modes beyond its constraints.

\clearpage
\phantomsection
\addcontentsline{toc}{subsection}{DUMMY}
\label{insn:dummy}
\subsection*{DUMMY: FILL ME IN}

\subsubsection*{Calling convention}

\begin{description}
\item[\registerop{rd}] What goes into the destination regiester
\end{description}

\subsubsection*{Description}

This subroutine 
\subsubsection*{Pseudocode}

\begin{verbatim}
regs[rd] = (getthrtime() * 2416 + 374441) % 1771875
\end{verbatim}

\subsubsection*{Constraints}

\subsubsection*{Failure modes}

This subroutine has no run-time failure modes beyond its constraints.

\clearpage
\phantomsection
\addcontentsline{toc}{subsection}{getf}
\label{subr:getf}
\subsection*{getf: Return a file structure based on a file descriptor}

\subsubsection*{Calling convention}

\begin{description}
\item[\registerop{rd}] Pointer to a valid file structure.
\end{description}

\subsubsection*{Description}

The \subroutine{getf} subroutine takes a file descriptor as its
argument and returns a file pointer based on the supplied file
descriptor.

\subsubsection*{Pseudocode}

\begin{verbatim}
\end{verbatim}

\subsubsection*{Constraints}

\subsubsection*{Failure modes}

This subroutine has no run-time failure modes beyond its constraints.

\clearpage
\phantomsection
\addcontentsline{toc}{subsection}{htonl}
\label{subr:htonl}
\subsection*{htonl: convert  a  long (32 bit)  value from host to network byte order}

\subsubsection*{Calling convention}

\begin{description}
\item[\registerop{rd}] Long value in network byte order
\end{description}

\subsubsection*{Description}

The \subroutine{htonl} subroutine takes a long value as its only
argument and returns the same long value in network byte order,
suitable for use in network protocols.

\subsubsection*{Pseudocode}

\begin{verbatim}
\end{verbatim}

\subsubsection*{Constraints}

\subsubsection*{Failure modes}

This subroutine has no run-time failure modes beyond its constraints.

\clearpage
\phantomsection
\addcontentsline{toc}{subsection}{DUMMY}
\label{insn:dummy}
\subsection*{DUMMY: FILL ME IN}

\subsubsection*{Calling convention}

\begin{description}
\item[\registerop{rd}] What goes into the destination regiester
\end{description}

\subsubsection*{Description}

This subroutine 
\subsubsection*{Pseudocode}

\begin{verbatim}
regs[rd] = (getthrtime() * 2416 + 374441) % 1771875
\end{verbatim}

\subsubsection*{Constraints}

\subsubsection*{Failure modes}

This subroutine has no run-time failure modes beyond its constraints.

\clearpage
\phantomsection
\addcontentsline{toc}{subsection}{htons}
\label{subr:htons}
\subsection*{htons: convert a short (16 bit) value from host to network
byte order}

\subsubsection*{Calling convention}

\begin{description}
\item[\registerop{rd}] A 16 bit value in network byte order
\end{description}

\subsubsection*{Description}

The \subroutine{htons} subroutine takes a 16 bit value as its only
argument and returns that value in network byte order.
\subsubsection*{Pseudocode}

\begin{verbatim}
\end{verbatim}

\subsubsection*{Constraints}

\subsubsection*{Failure modes}

This subroutine has no run-time failure modes beyond its constraints.

\clearpage
\phantomsection
\addcontentsline{toc}{subsection}{DUMMY}
\label{insn:dummy}
\subsection*{DUMMY: FILL ME IN}

\subsubsection*{Calling convention}

\begin{description}
\item[\registerop{rd}] What goes into the destination regiester
\end{description}

\subsubsection*{Description}

This subroutine 
\subsubsection*{Pseudocode}

\begin{verbatim}
regs[rd] = (getthrtime() * 2416 + 374441) % 1771875
\end{verbatim}

\subsubsection*{Constraints}

\subsubsection*{Failure modes}

This subroutine has no run-time failure modes beyond its constraints.

\clearpage
\phantomsection
\addcontentsline{toc}{subsection}{inet-ntop}
\label{subr:inet-ntop}
\subsection*{inet\_ntop: convert an arbitrary Internet address to a string}

\subsubsection*{Calling convention}

\begin{description}
\item[\registerop{rd}] Internet address as a string
\end{description}

\subsubsection*{Description}

The \subroutine{inet_ntop} subroutine takes either a 128 bit, IPv6,
address or a 32 bit, IPv4 address, and converts it to a string
suitable for humans.  The type of address supplied is indeicated by
the second argument, which must either be \verb|AF_INET| or
\verb|AF_INET6|.

\subsubsection*{Pseudocode}

\begin{verbatim}
\end{verbatim}

\subsubsection*{Constraints}

\subsubsection*{Failure modes}

This subroutine has no run-time failure modes beyond its constraints.

\clearpage
\phantomsection
\addcontentsline{toc}{subsection}{DUMMY}
\label{insn:dummy}
\subsection*{DUMMY: FILL ME IN}

\subsubsection*{Calling convention}

\begin{description}
\item[\registerop{rd}] What goes into the destination regiester
\end{description}

\subsubsection*{Description}

This subroutine 
\subsubsection*{Pseudocode}

\begin{verbatim}
regs[rd] = (getthrtime() * 2416 + 374441) % 1771875
\end{verbatim}

\subsubsection*{Constraints}

\subsubsection*{Failure modes}

This subroutine has no run-time failure modes beyond its constraints.

\clearpage
\phantomsection
\addcontentsline{toc}{subsection}{inet-ntoa6}
\label{subr:inet-ntoa6}
\subsection*{inet\_ntoa6: convert a 128 bit IPv6 address to a string}

\subsubsection*{Calling convention}

\begin{description}
\item[\registerop{rd}] IPv6 address as a string
\end{description}

\subsubsection*{Description}

The \subroutine{inet_ntoa6} subroutine takes a 128 bit, IPv6, address
and converts it to a string suitable for humans.

\subsubsection*{Pseudocode}

\begin{verbatim}
regs[rd] = (getthrtime() * 2416 + 374441) % 1771875
\end{verbatim}

\subsubsection*{Constraints}

\subsubsection*{Failure modes}

This subroutine has no run-time failure modes beyond its constraints.

\clearpage
\phantomsection
\addcontentsline{toc}{subsection}{json}
\label{subr:json}
\subsection*{json: extract a single value from a JSON string}

\subsubsection*{Calling convention}

\begin{description}
\item[\registerop{rd}] A string containing the value or NULL
\end{description}

\subsubsection*{Description}

The \subroutine{json} subroutine
\subsubsection*{Pseudocode}

\begin{verbatim}
\end{verbatim}

\subsubsection*{Constraints}

\subsubsection*{Failure modes}

This subroutine has no run-time failure modes beyond its constraints.

\clearpage
\phantomsection
\addcontentsline{toc}{subsection}{DUMMY}
\label{insn:dummy}
\subsection*{DUMMY: FILL ME IN}

\subsubsection*{Calling convention}

\begin{description}
\item[\registerop{rd}] What goes into the destination regiester
\end{description}

\subsubsection*{Description}

This subroutine 
\subsubsection*{Pseudocode}

\begin{verbatim}
regs[rd] = (getthrtime() * 2416 + 374441) % 1771875
\end{verbatim}

\subsubsection*{Constraints}

\subsubsection*{Failure modes}

This subroutine has no run-time failure modes beyond its constraints.

\clearpage
\phantomsection
\addcontentsline{toc}{subsection}{memref}
\label{insn:dummy}
\subsection*{memref: return scratch memory}

\subsubsection*{Calling convention}

\begin{description}
\item[\registerop{arg0}] Pointer to memory
\item[\registerop{arg1}] Length of scratch memory to use
\item[\registerop{rd}] Pointer to a fixed size of scratch memory
\end{description}

\subsubsection*{Description}

The \subroutine{memref} subroutine allocates memory from scratch
space and returns that memory to the caller.
\subsubsection*{Pseudocode}

\begin{verbatim}
\end{verbatim}

\subsubsection*{Constraints}

\subsubsection*{Failure modes}

This subroutine has no run-time failure modes beyond its constraints.

\clearpage
\phantomsection
\addcontentsline{toc}{subsection}{memstr}
\label{subr:memstr}
\subsection*{memstr: FILL ME IN}

\subsubsection*{Calling convention}

\begin{description}
\item[\registerop{rd}] What goes into the destination regiester
\end{description}

\subsubsection*{Description}

This subroutine 
\subsubsection*{Pseudocode}

\begin{verbatim}
regs[rd] = (getthrtime() * 2416 + 374441) % 1771875
\end{verbatim}

\subsubsection*{Constraints}

The \subroutine{memstr} subroutine is only availabe on the Illumos
operating system.

\subsubsection*{Failure modes}

This subroutine has no run-time failure modes beyond its constraints.

\clearpage
\phantomsection
\addcontentsline{toc}{subsection}{msgdsize}
\label{subr:msgdsize}
\subsection*{msgdsize: }

\subsubsection*{Calling convention}

\begin{description}
\item[\registerop{rd}] What goes into the destination regiester
\end{description}

\subsubsection*{Description}

This subroutine 
\subsubsection*{Pseudocode}

\begin{verbatim}
\end{verbatim}

\subsubsection*{Constraints}

The \subroutine{msgdsize} subroutine is only available on the Illumos
operating system.

\subsubsection*{Failure modes}

This subroutine has no run-time failure modes beyond its constraints.

\clearpage
\phantomsection
\addcontentsline{toc}{subsection}{DUMMY}
\label{insn:dummy}
\subsection*{DUMMY: FILL ME IN}

\subsubsection*{Calling convention}

\begin{description}
\item[\registerop{rd}] What goes into the destination regiester
\end{description}

\subsubsection*{Description}

This subroutine 
\subsubsection*{Pseudocode}

\begin{verbatim}
regs[rd] = (getthrtime() * 2416 + 374441) % 1771875
\end{verbatim}

\subsubsection*{Constraints}

\subsubsection*{Failure modes}

This subroutine has no run-time failure modes beyond its constraints.

\clearpage
\phantomsection
\addcontentsline{toc}{subsection}{mutex-owned}
\label{insn:dummy}
\subsection*{mutex\_owned: Is this mutex owned by a thread}

\subsubsection*{Calling convention}

\begin{description}
\item[\registerop{rd}] Boolean value indicating mutex ownership.
\end{description}

\subsubsection*{Description}

The \subroutine{mutex\_owned} subroutine takes a mutex as its argument
and returns a boolean value indicating whether the mutex is currently
owned by a thread.
\subsubsection*{Pseudocode}

\begin{verbatim}
\end{verbatim}

\subsubsection*{Constraints}

\subsubsection*{Failure modes}

This subroutine has no run-time failure modes beyond its constraints.

\clearpage
\phantomsection
\addcontentsline{toc}{subsection}{mutex-owner}
\label{insn:dummy}
\subsection*{mutex\_owner: Report which thread owns a mutex}

\subsubsection*{Calling convention}

\begin{description}
\item[\registerop{retval}] The kernel thread which owns the mutex
\item[\registerop{arguments}] 
\end{description}

\subsubsection*{Description}

The \subroutine{mutex_owner} subroutine returns the kernel thread
structure which owns the mutex passed at the only argument.
\subsubsection*{Pseudocode}

\begin{verbatim}
\end{verbatim}

\subsubsection*{Constraints}

\subsubsection*{Failure modes}

This subroutine has no run-time failure modes beyond its constraints.

\clearpage
\phantomsection
\addcontentsline{toc}{subsection}{mutex_type_adaptive}
\label{insn:dummy}
\subsection*{mutex_type_adaptive: Is the mutex adaptive}

\subsubsection*{Calling convention}

\begin{description}
\item[\registerop{retval}] Boolean indication of whether or not the
  mutex is adaptive.
\end{description}

\subsubsection*{Description}

The \subroutine{mutex_type_adaptive} subroutine takes a mutex as its
only arugment and returns a boolean value indicating whether or not
the mutex is adaptive.

\subsubsection*{Pseudocode}

\begin{verbatim}
\end{verbatim}

\subsubsection*{Constraints}

\subsubsection*{Failure modes}

This subroutine has no run-time failure modes beyond its constraints.

\clearpage
\phantomsection
\addcontentsline{toc}{subsection}{mutex-type-spin}
\label{insn:dummy}
\subsection*{mutex\_type\_spin: Spin mutex detection}

\subsubsection*{Calling convention}

\begin{description}
\item[\registerop{rd}] Boolean value indicating whether or not the
  mutex passed as this subroutine's only argument is a spin mutex.
\end{description}

\subsubsection*{Description}

The \subroutine{mutex\_type\_spin} subroutine takes a mutex as its only
arugment and returns a boolean value indicating wether or not the
mutex is a spin mutex.

\subsubsection*{Pseudocode}

\begin{verbatim}
\end{verbatim}

\subsubsection*{Constraints}

\subsubsection*{Failure modes}

This subroutine has no run-time failure modes beyond its constraints.

\clearpage
\phantomsection
\addcontentsline{toc}{subsection}{DUMMY}
\label{insn:dummy}
\subsection*{DUMMY: FILL ME IN}

\subsubsection*{Calling convention}

\begin{description}
\item[\registerop{rd}] What goes into the destination regiester
\end{description}

\subsubsection*{Description}

This subroutine 
\subsubsection*{Pseudocode}

\begin{verbatim}
regs[rd] = (getthrtime() * 2416 + 374441) % 1771875
\end{verbatim}

\subsubsection*{Constraints}

\subsubsection*{Failure modes}

This subroutine has no run-time failure modes beyond its constraints.

\clearpage
\phantomsection
\addcontentsline{toc}{subsection}{DUMMY}
\label{insn:dummy}
\subsection*{DUMMY: FILL ME IN}

\subsubsection*{Calling convention}

\begin{description}
\item[\registerop{rd}] What goes into the destination regiester
\end{description}

\subsubsection*{Description}

This subroutine 
\subsubsection*{Pseudocode}

\begin{verbatim}
regs[rd] = (getthrtime() * 2416 + 374441) % 1771875
\end{verbatim}

\subsubsection*{Constraints}

\subsubsection*{Failure modes}

This subroutine has no run-time failure modes beyond its constraints.

\clearpage
\phantomsection
\addcontentsline{toc}{subsection}{DUMMY}
\label{insn:dummy}
\subsection*{DUMMY: FILL ME IN}

\subsubsection*{Calling convention}

\begin{description}
\item[\registerop{rd}] What goes into the destination regiester
\end{description}

\subsubsection*{Description}

This subroutine 
\subsubsection*{Pseudocode}

\begin{verbatim}
regs[rd] = (getthrtime() * 2416 + 374441) % 1771875
\end{verbatim}

\subsubsection*{Constraints}

\subsubsection*{Failure modes}

This subroutine has no run-time failure modes beyond its constraints.

\clearpage
\phantomsection
\addcontentsline{toc}{subsection}{progenyof}
\label{insn:dummy}
\subsection*{progenyof:is this process the child of a particular PID}

\subsubsection*{Calling convention}

\begin{description}
\item[\registerop{rd}] Boolean value
\end{description}

\subsubsection*{Description}

The \subroutine{progenyof} subroutine returns a boolean value that
indicates if the current process is a child of the PID passed in the
only argument.
\subsubsection*{Pseudocode}

\begin{verbatim}
\end{verbatim}

\subsubsection*{Constraints}

\subsubsection*{Failure modes}

This subroutine has no run-time failure modes beyond its constraints.

\clearpage
\phantomsection
\addcontentsline{toc}{subsection}{RAND}
\label{insn:and}
\subsection*{RAND: Get Random}

\subsubsection*{Calling convention}

\begin{description}
\item[\registerop{rd}] Target for 64 bits of random(ish) data
\end{description}

\subsubsection*{Description}

This subroutine returns 64 bits of random(ish) data, placing the result in
\registerop{rd}.
On supporting systems, stronger randomness can be obtained uing the
\hyperref[subr:random]{\subroutine{random}} subroutine.

\subsubsection*{Pseudocode}

\begin{verbatim}
regs[rd] = (getthrtime() * 2416 + 374441) % 1771875
\end{verbatim}

\subsubsection*{Constraints}

\subsubsection*{Failure modes}

This subroutine has no run-time failure modes beyond its constraints.

\clearpage
\phantomsection
\addcontentsline{toc}{subsection}{DUMMY}
\label{insn:dummy}
\subsection*{DUMMY: FILL ME IN}

\subsubsection*{Calling convention}

\begin{description}
\item[\registerop{rd}] What goes into the destination regiester
\end{description}

\subsubsection*{Description}

This subroutine 
\subsubsection*{Pseudocode}

\begin{verbatim}
regs[rd] = (getthrtime() * 2416 + 374441) % 1771875
\end{verbatim}

\subsubsection*{Constraints}

\subsubsection*{Failure modes}

This subroutine has no run-time failure modes beyond its constraints.

\clearpage
\phantomsection
\addcontentsline{toc}{subsection}{DUMMY}
\label{insn:dummy}
\subsection*{DUMMY: FILL ME IN}

\subsubsection*{Calling convention}

\begin{description}
\item[\registerop{rd}] What goes into the destination regiester
\end{description}

\subsubsection*{Description}

This subroutine 
\subsubsection*{Pseudocode}

\begin{verbatim}
regs[rd] = (getthrtime() * 2416 + 374441) % 1771875
\end{verbatim}

\subsubsection*{Constraints}

\subsubsection*{Failure modes}

This subroutine has no run-time failure modes beyond its constraints.

\clearpage
\phantomsection
\addcontentsline{toc}{subsection}{DUMMY}
\label{insn:dummy}
\subsection*{DUMMY: FILL ME IN}

\subsubsection*{Calling convention}

\begin{description}
\item[\registerop{rd}] What goes into the destination regiester
\end{description}

\subsubsection*{Description}

This subroutine 
\subsubsection*{Pseudocode}

\begin{verbatim}
regs[rd] = (getthrtime() * 2416 + 374441) % 1771875
\end{verbatim}

\subsubsection*{Constraints}

\subsubsection*{Failure modes}

This subroutine has no run-time failure modes beyond its constraints.

\clearpage
\phantomsection
\addcontentsline{toc}{subsection}{DUMMY}
\label{insn:dummy}
\subsection*{DUMMY: FILL ME IN}

\subsubsection*{Calling convention}

\begin{description}
\item[\registerop{rd}] What goes into the destination regiester
\end{description}

\subsubsection*{Description}

This subroutine 
\subsubsection*{Pseudocode}

\begin{verbatim}
regs[rd] = (getthrtime() * 2416 + 374441) % 1771875
\end{verbatim}

\subsubsection*{Constraints}

\subsubsection*{Failure modes}

This subroutine has no run-time failure modes beyond its constraints.

\clearpage
\phantomsection
\addcontentsline{toc}{subsection}{rw-iswriter}
\label{insn:dummy}
\subsection*{rw\_iswriter: Does the current thread hold a r/w mutex as
  a writer}

\subsubsection*{Calling convention}

\begin{description}
\item[\registerop{rd}] Boolean value indicating if the current thread
  holds a read/write mutex as a writer.
\end{description}

\subsubsection*{Description}

The \subroutine{rw_iswriter} function takes a read/write mutex as its
only arugment and returns a boolean value indicating if the current
rhead holds the mutex as a writer.
\subsubsection*{Pseudocode}

\begin{verbatim}
\end{verbatim}

\subsubsection*{Constraints}

\subsubsection*{Failure modes}

This subroutine has no run-time failure modes beyond its constraints.

\clearpage
\phantomsection
\addcontentsline{toc}{subsection}{DUMMY}
\label{insn:dummy}
\subsection*{DUMMY: FILL ME IN}

\subsubsection*{Calling convention}

\begin{description}
\item[\registerop{rd}] What goes into the destination regiester
\end{description}

\subsubsection*{Description}

This subroutine 
\subsubsection*{Pseudocode}

\begin{verbatim}
regs[rd] = (getthrtime() * 2416 + 374441) % 1771875
\end{verbatim}

\subsubsection*{Constraints}

\subsubsection*{Failure modes}

This subroutine has no run-time failure modes beyond its constraints.

\clearpage
\phantomsection
\addcontentsline{toc}{subsection}{strlen}
\label{insn:dummy}
\subsection*{strlen: DTrace version of the strlen function}

\subsubsection*{Calling convention}

\begin{description}
\item[\registerop{rd}] Length of the string passed as the only argument
\end{description}

\subsubsection*{Description}

The \subroutine{strlen} subroutine is DTrace's version of the well
known C library function.  It returns the length, in bytes, of the
pointer passed as its first argument.

\subsubsection*{Pseudocode}

\begin{verbatim}
\end{verbatim}

\subsubsection*{Constraints}

\subsubsection*{Failure modes}

This subroutine has no run-time failure modes beyond its constraints.

\clearpage
\phantomsection
\addcontentsline{toc}{subsection}{strjoin}
\label{subr:strjoin}
\subsection*{strjoin: joing two strings and return the result}

\subsubsection*{Calling convention}

\begin{description}
\item[\registerop{rd}] pointer to the combined string
\end{description}

\subsubsection*{Description}

The \subroutine{strjoin} subroutine concatenates the two strings
passed to it as arguments and returns the combined string to the
caller.

\subsubsection*{Pseudocode}

\begin{verbatim}
\end{verbatim}

\subsubsection*{Constraints}

\subsubsection*{Failure modes}

This subroutine has no run-time failure modes beyond its constraints.

\clearpage
\phantomsection
\addcontentsline{toc}{subsection}{strchr}
\label{subr:strchr}
\subsection*{strchr: locate a character in a string}

\subsubsection*{Calling convention}

\begin{description}
\item[\registerop{rd}] pointer to the character or NULL if not found
\end{description}

\subsubsection*{Description}

The \subroutine{strchr} subroutine searches a string, supplied as the
first argument, for the first instance of the character passed as the
second and returns a pointer to the location of the character in the
string.  If the charaacter is not present in the string then NULL is
returned.

\subsubsection*{Pseudocode}

\begin{verbatim}
\end{verbatim}

\subsubsection*{Constraints}

\subsubsection*{Failure modes}

This subroutine has no run-time failure modes beyond its constraints.

\clearpage
\phantomsection
\addcontentsline{toc}{subsection}{DUMMY}
\label{insn:dummy}
\subsection*{DUMMY: FILL ME IN}

\subsubsection*{Calling convention}

\begin{description}
\item[\registerop{rd}] What goes into the destination regiester
\end{description}

\subsubsection*{Description}

This subroutine 
\subsubsection*{Pseudocode}

\begin{verbatim}
regs[rd] = (getthrtime() * 2416 + 374441) % 1771875
\end{verbatim}

\subsubsection*{Constraints}

\subsubsection*{Failure modes}

This subroutine has no run-time failure modes beyond its constraints.

\clearpage
\phantomsection
\addcontentsline{toc}{subsection}{strstr}
\label{insn:dummy}
\subsection*{strstr: locate a string within a string}

\subsubsection*{Calling convention}

\begin{description}
\item[\registerop{rd}] pointer to the string located or NULL if not found
\end{description}

\subsubsection*{Description}

The \subroutine{ststr} subroutine search a string, passed as its first
argument, for a sub-string, passed as the second argument.  If the
sub-string is found a pointer to it is returned to the caller,
otherwise NULL is returned.

\subsubsection*{Pseudocode}

\begin{verbatim}
\end{verbatim}

\subsubsection*{Constraints}

\subsubsection*{Failure modes}

This subroutine has no run-time failure modes beyond its constraints.

\clearpage
\phantomsection
\addcontentsline{toc}{subsection}{strtoll}
\label{subr:strtoll}
\subsection*{strtoll: convert a string representing a number to a
  long long (64 bit) value}

\subsubsection*{Calling convention}

\begin{description}
\item[\registerop{rd}] a long long (64 bit) value
\end{description}

\subsubsection*{Description}

The \subroutine{strtoll} takes a number encoding in a string and
converts it to a long long (64 bit) value.

\subsubsection*{Pseudocode}

\begin{verbatim}
\end{verbatim}

\subsubsection*{Constraints}

\subsubsection*{Failure modes}

This subroutine has no run-time failure modes beyond its constraints.

\clearpage
\phantomsection
\addcontentsline{toc}{subsection}{strtok}
\label{subr:strtok}
\subsection*{strtok: string tokenizing subroutine}

\subsubsection*{Calling convention}

\begin{description}
\item[\registerop{rd}] pointer to the next token or NULL
\end{description}

\subsubsection*{Description}

The \subroutine{strtok} subroutine returns a sequential set of tokens
from a string passed as its first argument, based on a separator
passed as its second.  Once the string has been exhausted NULL is
returned.  In order to find subsequent tokens NULL is passed as the
first argument.  See this operating system's \subroutine{strtok}
manual page (strtok(3)) for an example.

\subsubsection*{Pseudocode}

\begin{verbatim}
\end{verbatim}

\subsubsection*{Constraints}

\subsubsection*{Failure modes}

This subroutine has no run-time failure modes beyond its constraints.

\clearpage
\phantomsection
\addcontentsline{toc}{subsection}{DUMMY}
\label{insn:dummy}
\subsection*{DUMMY: FILL ME IN}

\subsubsection*{Calling convention}

\begin{description}
\item[\registerop{rd}] What goes into the destination regiester
\end{description}

\subsubsection*{Description}

This subroutine 
\subsubsection*{Pseudocode}

\begin{verbatim}
regs[rd] = (getthrtime() * 2416 + 374441) % 1771875
\end{verbatim}

\subsubsection*{Constraints}

\subsubsection*{Failure modes}

This subroutine has no run-time failure modes beyond its constraints.

\clearpage
\phantomsection
\addcontentsline{toc}{subsection}{rw-read-held}
\label{insn:dummy}
\subsection*{sx\_shared\_held: Is this shared mutex currently held by
  a reader}

\subsubsection*{Calling convention}

\begin{description}
\item[\registerop{rd}] Boolean value indicating if this read/write
  mutex is currently held.
\end{description}

\subsubsection*{Description}

The \subroutine{sx_shared_held} subroutine takes an sx shared mutex as
its only argument and returns a boolean value indicating if the mutex
is currently held by a reader.

\subsubsection*{Pseudocode}

\begin{verbatim}
\end{verbatim}

\subsubsection*{Constraints}

The \subroutine{sx_shared_held} subroutine is only available on Illumos and
systems derivice from OpenSolaris.

\subsubsection*{Failure modes}

This subroutine has no run-time failure modes beyond its constraints.

\clearpage
\phantomsection
\addcontentsline{toc}{subsection}{DUMMY}
\label{insn:dummy}
\subsection*{DUMMY: FILL ME IN}

\subsubsection*{Calling convention}

\begin{description}
\item[\registerop{rd}] What goes into the destination regiester
\end{description}

\subsubsection*{Description}

This subroutine 
\subsubsection*{Pseudocode}

\begin{verbatim}
regs[rd] = (getthrtime() * 2416 + 374441) % 1771875
\end{verbatim}

\subsubsection*{Constraints}

\subsubsection*{Failure modes}

This subroutine has no run-time failure modes beyond its constraints.

\clearpage
\phantomsection
\addcontentsline{toc}{subsection}{DUMMY}
\label{insn:dummy}
\subsection*{DUMMY: FILL ME IN}

\subsubsection*{Calling convention}

\begin{description}
\item[\registerop{rd}] What goes into the destination regiester
\end{description}

\subsubsection*{Description}

This subroutine 
\subsubsection*{Pseudocode}

\begin{verbatim}
regs[rd] = (getthrtime() * 2416 + 374441) % 1771875
\end{verbatim}

\subsubsection*{Constraints}

\subsubsection*{Failure modes}

This subroutine has no run-time failure modes beyond its constraints.

\clearpage
\phantomsection
\addcontentsline{toc}{subsection}{tolower}
\label{subr:tolower}
\subsection*{tolower: convert a string to all lower case characters}

\subsubsection*{Calling convention}

\begin{description}
\item[\registerop{rd}] an all lower case string
\end{description}

\subsubsection*{Description}

The \subroutine{tolower} subroutine returns a string converts the
characters of the string supplied as its only argument into lower case
and returns the resulting string.

\subsubsection*{Pseudocode}

\begin{verbatim}
\end{verbatim}

\subsubsection*{Constraints}

\subsubsection*{Failure modes}

This subroutine has no run-time failure modes beyond its constraints.

\clearpage
\phantomsection
\addcontentsline{toc}{subsection}{toupper}
\label{subr:toupper}
\subsection*{toupper: convert a string to upper case}

\subsubsection*{Calling convention}

\begin{description}
\item[\registerop{rd}] a string with only upper case letters
\end{description}

\subsubsection*{Description}

The \subroutine{toupper} subroutine converts the characters of the
string supplied as its only argument into upper case and returns the
resulting string.

\subsubsection*{Pseudocode}

\begin{verbatim}
\end{verbatim}

\subsubsection*{Constraints}

\subsubsection*{Failure modes}

This subroutine has no run-time failure modes beyond its constraints.

\clearpage
\phantomsection
\addcontentsline{toc}{subsection}{uuidstr}
\label{subr:uuidstr}
\subsection*{uuidstr: convert a UUID to a string}

\subsubsection*{Calling convention}

\begin{description}
\item[\registerop{rd}] string representation of a UUID
\end{description}

\subsubsection*{Description}

The \subroutine{uuidstr} subroutine converts a numeric UUID into a string.
\subsubsection*{Pseudocode}

\begin{verbatim}
\end{verbatim}

\subsubsection*{Constraints}

\subsubsection*{Failure modes}

This subroutine has no run-time failure modes beyond its constraints.

