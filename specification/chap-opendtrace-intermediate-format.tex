\section{The DIF Interpreter}

The DTrace Intermediate Format (DIF) interpreter is a virtual machine
that executes instructions on behalf of D scripts that are associated
with predicates and actions.  DIF is a simple, RISC-like, instruction
set where each instruction consists of a 32-bit, native-endian integer
whose most significant 8 bits contain an opcode allowing the remainder
of the instruction to be decoded. While DIF is an interpreter on its own,
it is just one of many actions that DTrace can execute. Its purpose is to
gather arguments and set up state for all other actions.

Each DIF object is executed separately with its own register file, as
DIF does not have a notion of a stack. Each DIF object must end with
a return instruction which will cause the value in the returned register
to be written into the trace buffer.

Before DIF is executed, DTrace will perform a sanity check for each of the
DIF objects and ensure that they contain valid DIF. The constraints for
each DIF instruction will be enumerated in the instruction description in
Chapter~\ref{chap:opendtrace-instruction-reference}.

The following chapter describes the overall implementation of the DIF
interpreter as well as how the various instructions are implemented,
along with various implementation details. \footnote{This
  specification describes the DTrace Intermediate Format version 2, as
  shipped in Illumos 5, FreeBSD 8-12, and macOS 10.5-10.13}

A comprehensive description of OpenDTrace's instructions are given in
Chapter~\ref{chap:opendtrace-instruction-reference} and a full list
and description of the built-in subroutines are given in
Chapter~\ref{chap:opendtrace-subroutines}.

\subsection{Registers}
\label{sec:dif-registers}

The OpenDTrace virtual machine is made up of eight (8) integer registers
and eight (8) tuple registers.  The 0th integer register always
contains the value zero (0).  The tuple registers are used for
handling any data type beyond a simple integer, such as strings, and
pointers to memory.  Each of the tuple registers is made up of a size
and value, where the value is a pointer to memory and the size
indicates how much memory OpenDTrace will attempt to address.  It is
the tuple registers that allow D scripts to work with data by reference.

All operations are carried out using registers \registerop{r1} and
\registerop{r2} as operands and \registerop{rd} as the destination for
all results.

\subsection{Tables}
\label{sec:dif-tables}

Each DIF Object contains pointers to an integer, string and variable
table which are optionally filled in as they are needed.  The integer
table acts as integer constants that will be operated on during the
execution of a D program.  The string table contains any string data
allocated or used in a D program.  Any variables that are used in the
program are contained in the variable table.

\subsection{Math Instructions}
\label{sec:dif-math}

Instructions for mathematical operations in DIF have no concept of
over or underflow.  The division instructions set a flag to indicate a
division by zero error.

\subsection{Comparison and Test Instructions}
\label{sec:dif-cmp-tst}

\begin{table}
  \centering
    \begin{tabular}{|l|l|}
      \hline
      Variable & Meaning\\
      \hline
      cc\_r & Value of $r1 - r2$\\
      cc\_n & Comparison result is negative. \\
      cc\_z & Comparison result is 0.\\
      cc\_v & Overflow occurred.\\
      cc\_c & Is $r1 < r2$?\\
      \hline
  \end{tabular}
\label{tbl:cmp-vars}
\caption{Mathematical Operation Result Bits}
\end{table}

DIF has three comparison instructions, \instruction{cmp}, \instruction{scmp} and
\instruction{tst} which can set various result flags, shown in
Table~\ref{tbl:cmp-vars}.  The result flags are are later used by the
branch instructions to determine whether or not the branch is taken.
The result flags are never returned directly to the calling DIF program but are
only used internally by the interpretation routine.

\subsection{Branching Instructions}

DIF has eleven branch instructions split into two types: signed and
unsigned.  The signed branching instructions take into account that
the number may be negative, while the unsigned instructions are meant
to be used with exclusively positive numbers. One thing all of the
branching instructions have in common is that they load the new
\register{pc} register from the \registerop{label} field in the Branch
Format described in Subsection~\ref{subsec:b-format}.

\subsection{Subroutine Calls}
\label{sec:subroutines}

Within DIF subroutines are triggered via the \instruction{CALL} instruction.
The arguments to these subroutines are passed through the tuple stack.
The tuple stack itself is populated using \verb|pushtr| and \verb|pushtv|
instructions and the return values of the subroutines are provided
through the \registerop{rd} register. The subroutine identifier is
placed in the \registerop{idx} field of the wide-immediate format (W-Format)
described in Subsection~\ref{subsec:w-format}. Any subroutine that is provided
to DTrace \emph{must} go via these mechanisms. None of the arguments to subroutines
need to be validated before calling the subroutine, as they originate from the
previously validated data using other DIF instructions which themselves have
validated this data beforehand.

% \textbf{XXXDS: Are these comments related to the Subroutine Calls section, or
% overall?}

% \textbf{XXXRW: Something about implied status bits}

% \textbf{XXXRW: Something about scratch memory (alloca, user copyin)}

% \textbf{XXXRW: Something about the tuple stack}

% \textbf{XXXRW: Something about kernel memory access}

% \textbf{XXXRW: Something about user memory access}

% XXXDS: I've removed all the variable bits here, as their semantics and syntax is
% described in the DLang chapter. The only thing important for DIF is how they're
% represented.
\subsection{Variables}
\label{sec:dif-vars}

Variables in DIF are just numeric references to simple names within the D script.
The space for variables is statically allocated on each invocation of a script.
Additionally, variables are identified using the modified register format as described
in Subsection~\ref{subsec:r-format} when working with arrays and the W-Format described
in Subsection~\ref{subsec:w-format} when working with scalar variables.

\subsection{Scalars}
\label{sec:scalars}

Scalar variables are loaded into registers using \verb|LDGS|,
\verb|LDTS| and \verb|LDLS| and stored to memory from registers using
\verb|STGS|, \verb|STTS| and \verb|STLS| for global, thread-local and
clause-local scalar variables respectively.

\subsection{Arrays}
\label{sec:arrays}

%% FIXME: There are also LDGA and LDTA, which seem to be for built-in variables (args, uregs)
%% but I'm unable to determine that for sure. I'm not able to generate LDTA yet.
Similarly to scalar variables, array variables are loaded into registers using
\verb|LDGAA| and \verb|LDTAA| and stored to using \verb|STGAA| and \verb|STTAA|
for global and thread-local array variables respectively. Note that there are no
instructions for clause-local array variables.

\section{Instruction Format}

Each instruction consists of a 32-bit, native-endian integer whose most
significant 8 bits contain an opcode allowing the remainder of the instruction
to be decoded.
To ease parsing, three major formats (R, B, and W) are used for all OpenDTrace
instructions, capturing different types of operations: register-to-register
instructions accepting zero or more register operands; branch instructions
accepting a target label as a single operand; and wide-immediate instructions
that accept a 16-bit immediate used to capture both small constant values and
also indices into various tables.

\subsection{Register Format (R-Format)}
\label{subsec:r-format}
This format accepts zero or more register operands, supporting instructions
that include arithmetic and boolean operations, comparison and test
operations, load and store operations, tuple-stack operations, and the no-op
instruction.

\begin{center}
\begin{bytefield}[endianness=big,bitformatting=\scriptsize]{32}
\bitheader{0,7,8,15,16,23,24,31}\\
\bitbox{8}{op}
\bitbox{8}{r1}
\bitbox{8}{r2}
\bitbox{8}{rd}
\end{bytefield}
\end{center}

\begin{description}
\item[op] Mandatory 8-bit operation identifier
\item[r1, r2] Optional source registers providing input values to the
  operation
\item[rd] Optional destination register acting as the destination of the
  operation
\end{description}


A modified version of the Register Format is used when loading and
storing data in array variables in OpenDTrace. The main difference
between the regular Register Format and the modified one used for
arrays, is that the \registerop{r1} register location is used as the
variable identificator, the \registerop{r2} register itself contains
the optional index in the array.

\begin{center}
\begin{bytefield}[endianness=big,bitformatting=\scriptsize]{32}
\bitheader{0,7,8,15,16,23,24,31}\\
\bitbox{8}{op}
\bitbox{8}{var}
\bitbox{8}{r2}
\bitbox{8}{rd}
\end{bytefield}
\end{center}

\begin{description}
\item[op] Mandatory 8-bit operation identifier
\item[var] The variable identifier
\item[r2] Optional register that contains the index of the array
\item[rd] Optional destination register acting as the destination of the
  operation
\end{description}

\subsection{Branch Format (B-Format)}
\label{subsec:b-format}

This format accepts a single 24-bit integer operand identifying the label that
is the branch target.
It is used solely for the \textbf{BRANCH} instruction.

\begin{center}
\begin{bytefield}[endianness=big,bitformatting=\scriptsize]{32}
\bitheader{0,23,24,31}\\
\bitbox{8}{op}
\bitbox{24}{label}
\end{bytefield}
\end{center}

\begin{description}
\item[op] Mandatory 8-bit operation identifier
\item[label] Mandatory 24-bit integer label
\end{description}

\subsection{Wide-Immediate Format (W-Format)}
\label{subsec:w-format}

This format accepts an 8-bit register and 16-bit integer argument (frequently an
index).  It is used for a range of instructions including those to load values
from integer and string constant tables, as well as those that store scalar
values in variables. In addition to that, it is used in the \instruction{CALL}
instruction in order to specify the \registerop{rd} register and the subroutine
identifier.

\begin{center}
\begin{bytefield}[endianness=big,bitformatting=\scriptsize]{32}
\bitheader{0,7,8,23,24,31}\\
\bitbox{8}{op}
\bitbox{16}{idx}
\bitbox{8}{rs\textbar rd}
\end{bytefield}
\end{center}

\begin{description}
\item[op] Mandatory 8-bit operation identifier
\item[idx] Mandatory 16-bit integer index
\item[rs\textbar rd] Optional 8-bit register acting as the source or
  destination of the operation
\end{description}

