\section{The DIF Interpreter}

The DTrace Intermediate Format (DIF) interpreter is a virtual machine
that executes instructions on behalf of D scripts that are associated
with predicates and actions.  DIF is a simple, RISC-like, instruction
set where each instruction consists of a 32-bit, native-endian integer
whose most significant 8 bits contain an opcode allowing the remainder
of the instruction to be decoded.  Interpretation is executed in a
loop within the \function{dtrace_dif_emulate} function, which has, as
its first argument, a pointer to a DIF Object or
\struct{dtrace_difo_t}. Each of the DIF objects gets executed in its
own separate environment and must return a value using the
\instruction{ret} instruction. Instructions are executed one at a
time, until they are exhausted or an error causes interpretation to
end. The DIF objects are verified in the
\function{dtrace_difo_validate} function and the DIF interpreter
ignores any bounds checking within the \function{dtrace_dif_emulate}
function precisely because \function{dtrace_difo_validate} performs
the necessary checks.

The following chapter describes the overall implementation of the DIF
interpreter as well as how the various instructions are implemented,
along with various implementation details. \footnote{This
  specification describes the DTrace Intermediate Format version 2, as
  shipped in Illumos 5, FreeBSD 8-12, and macOS 10.5-10.13}

\subsection{Registers}
\label{sec:dif-registers}

The OpenDTrace virtual machine is made up of eight (8) integer registers
and eight (8) tuple registers.  The 0th integer register always
contains the value zero (0).  All operations are carried out using
registers \registerop{r1} and \registerop{r2} as operands and
\registerop{rd} as the destination for all results. A comprehensive
description of OpenDTrace's instructions are given in
Chapter~\ref{chap:opendtrace-instruction-reference} and a full list
and description of the built-in subroutines are given in
Chapter~\ref{chap:opendtrace-subroutines}.

\subsection{Math Instructions}
\label{sec:dif-math}

Instructions for mathematical operations in DIF have no concept of
over or underflow.  The division instructions set a flag to indicate a
division by zero error.

\subsection{Comparison and Test Instructions}
\label{sec:dif-cmp-tst}

\begin{table}
  \centering
    \begin{tabular}{|l|l|}
      \hline
      Variable & Meaning\\
      \hline
      cc\_r & Value of $r1 - r2$\\
      cc\_n & Comparison result is negative. \\
      cc\_z & Comparison result is 0.\\
      cc\_v & Overflow occurred.\\
      cc\_c & Is $r1 < r2$?\\
      \hline
  \end{tabular}
\label{tbl:cmp-vars}
\caption{Mathematical Operation Result Bits}
\end{table}

DIF has three instructions, \instruction{cmp}, \instruction{scmp} and
\instruction{tst} which can set various result flags, shown in
Table~\ref{tbl:cmp-vars}.  The result flags are are later used by the
branch instructions to determine whether or not the branch is taken.
The result flags are never returned directly to the calling DIF program but are
only used internally by the interpretation routine.

\textbf{XXXRW: Something about instructions}

\subsection{Branching Instructions}

DIF has eleven branch instructions split into two types: signed and
unsigned.  The signed branching instructions take into account that
the number may be negative, while the unsigned instructions are meant
to be used with exclusively positive numbers. One thing all of the
branching instructions have in common is that they load the new
\register{pc} register from the \registerop{label} field in the Branch
Format described in Subsection~\ref{subsec:b-format}.

\subsection{Subroutine Calls}
\label{sec:subroutines}

OpenDTrace provides an extensive set of subroutines for use by D programs.
The subroutines are implemented within the kernel code via a set of
functions which are centrally dispatched from the
\function{dtrace_dif_subr} function.  Within DIF subroutines are
triggered via the \instruction{CALL} instruction. The arguments to
these subroutines are passed through the \verb|tupregs| variable
through use of the \instruction{PUSHTR} and \instruction{PUSHTV}
instructions. The return values of the subroutines are provided
through the \registerop{rd} register. The subroutine identifier is
placed in the \registerop{idx} field of the wide-immediate format (W-Format)
described in Subsection~\ref{subsec:w-format}. Any subroutine that is provided
to DTrace \emph{must} go via these mechanisms.

\begin{quote}
  Notice that we don't bother validating the proper number of
  arguments or their types in the tuple stack.  This isn't needed
  because all argument interpretation is safe because of our load
  safety -- the worst that can happen is that a bogus program can
  obtain bogus results.
\label{fig:argcheck}
\end{quote}

According to a comment in the code, see Figure~\ref{fig:argcheck}, argument
checks are not carried out when a subroutine is called. These checks are
performed in the \function{dtrace_difo_validate} function at load time.

\textbf{XXXDS: Are these comments related to the Subroutine Calls section, or
overall?}

\textbf{XXXRW: Something about DIF registers, \nregs{}}

\textbf{XXXRW: Something about the zero register}

\textbf{XXXRW: Something about implied status bits}

\textbf{XXXRW: Something about scratch memory (alloca, user copyin)}

\textbf{XXXRW: Something about the tuple stack}

\textbf{XXXRW: Something about kernel memory access}

\textbf{XXXRW: Something about user memory access}

\textbf{XXXRW: Something about ``by reference''}

\textbf{XXXRW: Something about the integer table}

\textbf{XXXRW: Something about the string table}

\subsection{Local Variables}
\label{sec:local-vars}

Local variables are local to the D program clause.  In a D program
these variables are referenced using the \verb|this->| syntax.  

\subsection{Thread Local Variables}
\label{sec:thread-local-vars}

Thread local variables are usable in multiple D program clauses.  In a
D program the thread local variables are referenced using the
\verb|self->| syntax.

\subsection{Global Variables}
\label{sec:globals-vars}

Global variables are global to all the clauses in a D script and are references
to simple names within the D script.  Space for global variables is statically
allocated on each invocation of a script. Additionally, global variables are
identified using the modified register format as described in
Subsection~\ref{subsec:r-format} the case of arrays and the W-Format
which is described in Subsection~\ref{subsec:w-format} in the case
of scalar variables inside of the \function{dtrace_dif_variable} function.

\subsubsection{Dynamic Variables}
\label{sec:sec:dynamic-vars}

% From Samuel Lepetit
% It would be super useful to have a section on how dynamic variables
% work and how the hashing magic in dtrace_dynvar work for those.

\textbf{XXXRW: Something about scalars}

\textbf{XXXRW: Something about arrays}

\textbf{XXXRW: Something about aggregations}

\textbf{XXXRW: Something about exceptions}

\section{Instruction Format}

Each instruction consists of a 32-bit, native-endian integer whose most
significant 8 bits contain an opcode allowing the remainder of the instruction
to be decoded.
To ease parsing, three major formats (R, B, and W) are used for all OpenDTrace
instructions, capturing different types of operations: register-to-register
instructions accepting zero or more register operands; branch instructions
accepting a target label as a single operand; and wide-immediate instructions
that accept a 16-bit immediate used to capture both small constant values and
also indices into various tables.

\subsection{Register Format (R-Format)}
\label{subsec:r-format}
This format accepts zero or more register operands, supporting instructions
that include arithmetic and boolean operations, comparison and test
operations, load and store operations, tuple-stack operations, and the no-op
instruction.

\begin{center}
\begin{bytefield}[endianness=big,bitformatting=\scriptsize]{32}
\bitheader{0,7,8,15,16,23,24,31}\\
\bitbox{8}{op}
\bitbox{8}{r1}
\bitbox{8}{r2}
\bitbox{8}{rd}
\end{bytefield}
\end{center}

\begin{description}
\item[op] Mandatory 8-bit operation identifier
\item[r1, r2] Optional source registers providing input values to the
  operation
\item[rd] Optional destination register acting as the destination of the
  operation
\end{description}


A modified version of the Register Format is used when loading and
storing data in array variables in OpenDTrace. The main difference
between the regular Register Format and the modified one used for
arrays, is that the \registerop{r1} register location is used as the
variable identificator, the \registerop{r2} register itself contains
the optional index in the array.

\begin{center}
\begin{bytefield}[endianness=big,bitformatting=\scriptsize]{32}
\bitheader{0,7,8,15,16,23,24,31}\\
\bitbox{8}{op}
\bitbox{8}{var}
\bitbox{8}{r2}
\bitbox{8}{rd}
\end{bytefield}
\end{center}

\begin{description}
\item[op] Mandatory 8-bit operation identifier
\item[var] The variable identifier
\item[r2] Optional register that contains the index of the array
\item[rd] Optional destination register acting as the destination of the
  operation
\end{description}

\subsection{Branch Format (B-Format)}
\label{subsec:b-format}

This format accepts a single 24-bit integer operand identifying the label that
is the branch target.
It is used solely for the \textbf{BRANCH} instruction.

\begin{center}
\begin{bytefield}[endianness=big,bitformatting=\scriptsize]{32}
\bitheader{0,23,24,31}\\
\bitbox{8}{op}
\bitbox{24}{label}
\end{bytefield}
\end{center}

\begin{description}
\item[op] Mandatory 8-bit operation identifier
\item[label] Mandatory 24-bit integer label
\end{description}

\subsection{Wide-Immediate Format (W-Format)}
\label{subsec:w-format}

This format accepts an 8-bit register and 16-bit integer argument (frequently an
index).  It is used for a range of instructions including those to load values
from integer and string constant tables, as well as those that store scalar
values in variables. In addition to that, it is used in the \instruction{CALL}
instruction in order to specify the \registerop{rd} register and the subroutine
identifier.

\begin{center}
\begin{bytefield}[endianness=big,bitformatting=\scriptsize]{32}
\bitheader{0,7,8,23,24,31}\\
\bitbox{8}{op}
\bitbox{16}{idx}
\bitbox{8}{rs\textbar rd}
\end{bytefield}
\end{center}

\begin{description}
\item[op] Mandatory 8-bit operation identifier
\item[idx] Mandatory 16-bit integer index
\item[rs\textbar rd] Optional 8-bit register acting as the source or
  destination of the operation
\end{description}

