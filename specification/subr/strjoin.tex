\clearpage
\phantomsection
\addcontentsline{toc}{subsection}{strjoin}
\label{subr:strjoin}
\subsection*{strjoin: joing two strings and return the result}

\subsubsection*{Subroutine prototype}

\begin{verbatim}
char * strjoin(const char *first, const char *second);
\end{verbatim}

\subsubsection*{Calling convention}

\begin{description}
\item[\registerop{arg0}] Pointer to the first string
\item[\registerop{arg1}] Pointer to the second string
\item[\registerop{rd}] Pointer to the combined string
\end{description}

\subsubsection*{Description}

The \subroutine{strjoin} subroutine concatenates the two strings
passed to it as arguments and returns the combined string to the
caller.

\subsubsection*{Pseudocode}

\begin{verbatim}
first = stack[0].value
second = stack[1].value
combined = scratch_space

for i = 0 ... len(first):
    combined(i) = first(i)

for j = 0 ... len(second):
    combined(i + j) = second(j)

scratch_space += len(combined)
%rd = combined
\end{verbatim}

\subsubsection*{Constraints}

\subsubsection*{Failure modes}

This subroutine has no run-time failure modes beyond its constraints.
