\clearpage
\phantomsection
\addcontentsline{toc}{subsection}{strstr}
\label{subr:strstr}
\subsection*{strstr: locate a string within a string}

\subsubsection*{Subroutine prototype}

\begin{verbatim}
char * strstr(const char *big, const char *little);
\end{verbatim}

\subsubsection*{Calling convention}

\begin{description}
\item[\registerop{arg0}] Pointer to the string to be searched through
\item[\registerop{arg1}] Pointer to the string to search for
\item[\registerop{rd}] Pointer to the string located or NULL if not found
\end{description}

\subsubsection*{Description}

The \subroutine{strstr} subroutine search a string, passed as its first
argument, for a sub-string, passed as the second argument. If the
sub-string is found a pointer to it is returned to the caller,
otherwise NULL is returned.

\subsubsection*{Pseudocode}

\begin{verbatim}
big = stack[0].value
little = stack[1].value
%rd = strstr(big, little)
\end{verbatim}

\subsubsection*{Failure modes}

This subroutine has no run-time failure modes beyond its constraints.
