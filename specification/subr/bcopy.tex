\clearpage
\phantomsection
\addcontentsline{toc}{subsection}{bcopy}
\label{subr:bcopy}
\subsection*{bcopy: copy bytes from source to destination bounded by a
size}

\subsubsection*{Subroutine prototype}

\begin{verbatim}
void bcopy(const void *source, void *destination, size_t length);
\end{verbatim}

\subsubsection*{Calling convention}

\begin{description}
\item[\registerop{arg0}] Pointer to the source memory
\item[\registerop{arg1}] Pointer to the destination scratch memory
\item[\registerop{arg2}] Amount of bytes to copy
\item[\registerop{rd}] VOID
\end{description}

\subsubsection*{Description}

The \subroutine{bcopy} subroutine copies bytes from a source pointer
to a destination pointer, within the DTrace machine state scratch
region, up to the size supplied in the third argument.
\subsubsection*{Pseudocode}

\begin{verbatim}
source = stack[0].value
destination = stack[1].value
length = stack[2].value

if destination not in scratch:
    return

if not can_load(source):
    %rd = 0
    return

for i = 0 ... length:
    destination(i) = source(i)
\end{verbatim}

\subsubsection*{Constraints}

\subsubsection*{Failure modes}

This subroutine has no run-time failure modes beyond its constraints.
