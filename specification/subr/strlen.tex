\clearpage
\phantomsection
\addcontentsline{toc}{subsection}{strlen}
\label{subr:strlen}
\subsection*{strlen: DTrace version of the strlen function}

\subsubsection*{Calling convention}

\begin{description}
\item[\registerop{rd}] Length of the string passed as the only argument
\end{description}

\subsubsection*{Description}

The \subroutine{strlen} subroutine is DTrace's version of the well
known C library function.  It returns the length, in bytes, of the string
pointed to by the pointer passed in as its first argument. The string must be
NULL terminated.

\subsubsection*{Pseudocode}

\begin{verbatim}
string = stack[0].value
%rd = strlen(string)
\end{verbatim}

\subsubsection*{Constraints}

\subsubsection*{Failure modes}

This subroutine has no run-time failure modes beyond its constraints.
