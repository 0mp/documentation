\clearpage
\phantomsection
\addcontentsline{toc}{subsection}{progenyof}
\label{subr:progenyof}
\subsection*{progenyof:is this process the child of a particular PID}

\subsubsection*{Calling convention}

\begin{description}
\item[\registerop{rd}] Boolean value
\end{description}

\subsubsection*{Description}

The \subroutine{progenyof} subroutine returns a boolean value that
indicates if the current process is a child of the PID passed in the
only argument.

\subsubsection*{Failure modes}

This subroutine has no run-time failure modes beyond its constraints.
