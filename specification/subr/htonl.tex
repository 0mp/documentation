\clearpage
\phantomsection
\addcontentsline{toc}{subsection}{htonl}
\label{subr:htonl}
\subsection*{htonl: convert  a  long (32 bit)  value from host to network byte order}

\subsubsection*{Calling convention}

\begin{description}
\item[\registerop{rd}] Long value in network byte order
\end{description}

\subsubsection*{Description}

The \subroutine{htonl} subroutine takes a long value as its only
argument and returns the same long value in network byte order,
suitable for use in network protocols.

\subsubsection*{Pseudocode}

\begin{verbatim}
\end{verbatim}

\subsubsection*{Constraints}

\subsubsection*{Failure modes}

This subroutine has no run-time failure modes beyond its constraints.
