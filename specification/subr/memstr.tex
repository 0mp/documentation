\clearpage
\phantomsection
\addcontentsline{toc}{subsection}{memstr}
\label{subr:memstr}
\subsection*{memstr: convert NULL separated strings to one string}

\subsubsection*{Calling convention}

\begin{description}
\item[\registerop{arg0}] pointer to memory
\item[\registerop{arg1}] separation character
\item[\registerop{arg2}] length of memory to convert
\item[\registerop{rd}] converted string
\end{description}

\subsubsection*{Description}

The \subroutine{memstr} subroutine converts a set of NULL separated
strings into a single string.  The string is bounded by the caller.


\subsubsection*{Constraints}

The maximum length of string to be converted is limited to 4096 bytes
by default.  The \subroutine{memstr} subroutine is only available on
the FreeBSD operating system.

\subsubsection*{Failure modes}

This subroutine has no run-time failure modes beyond its constraints.
