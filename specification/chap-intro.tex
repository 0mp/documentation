OpenDTrace is a dynamic tracing facility integrated into the Solaris,
FreeBSD, and macOS operating systems\textemdash with ports also available for
Linux and Windows.  Dynamic tracing allows system administrators and
software developers to develop short scripts (in the D programming
language) that instruct OpenDTrace to instrument aspects of system
operation, gather data, and present it for human interpretation or
mechanical processing.  While there is excellent documentation
available for the D programming language, command-line tools, and
OpenDTrace-based investigation and operation, the internal formats to
OpenDTrace are generally documented via the source code.  This report
acts as a \textit{de facto} specification for those formats, including the
DTrace Intermediate Format (DIF), which is a bytecode that D scripts
are compiled into for safe execution within the kernel, and the DTrace
Object Format (DOF), which bundles together complete scripts along
with their associated constants and metadata.

%\section{Motivation}
\section{Background}

The original DTrace code was designed and developed by Sun
Microsystems to solve a particular problem, being able to instrument
systems that were currently deployed, without requiring the
recompilation of any code \cite{DTrace2004}.  The DTrace system was
written in a portable style typical of code from the Sun Microsystems
Kernel Development group in the early 2000s.  Shortly after the
release of the original DTrace system a port was made, by John Birrel,
to the FreeBSD Operating System.  A port was also made by Apple to
their macOS at about the same time.  DTrace gained popularity as a
dynamic tracing system throughout the first decade of the 21st Century
and its usage is well documented
\cite{mckusick2014design}\cite{Microsystems2008a}\cite{Gregg:2011:DDT:1971960}.

The OpenDTrace system is meant to capture information about systems at
run time, without the need to stop the program or kernel being
investigated.  
A tracing system captures the program state of a running program and
can show changes in that state over time.  The person who is
initiating the trace must decide before starting what information they
wish to capture.  Tracing systems have an important design constraint,
which is the need to make the tracing system itself have as low an
impact on overall system performance as possible.

From the perspective of the user the OpenDTrace model is one of
\emph{Plan}, \emph{Capture} and \emph{Analyze}.  The \emph{Plan} phase
is where the user writes brief scripts, in the D langauge, that
describe the probe points from which they wish to capture data.
Conditions can be placed upon when these probe points are active, so
that the amount of data captured in the next phase, can be narrowed
down to only what is absolutely neessary to feed the analysis and
answer the question we are asking of the system.  The \emph{Capture}
phase is triggered by the \texttt{dtrace} program pushing the plan,
in the form of compiled code, into the operating system's kernel which
activates the required probe points.  The OS kernel captures the data
into bufers which are eventually fed out to user space, where they can
be analyzed.  The \emph{Analysis} is undertaken in user space where
the previously written plan, in the form of D scripts, directs the
OpenDTrace library to extract, display and or aggregate the captured
data.  Many workflows currently require some form of post-processing
of the data captured for analysis, and this post-processing is
currently carried out on unstructured text.

OpenDTrace is made up of several components, including kernel code,
user space libraries, and command line tools.  The OpenDTrace system
uses information generated during code compilation to expose a set of
trace points with which users and programs can interact.  These trace
points can be the entry and exit points of functions as well as system
calls, or they can be arbitrary points in the instruction stream,
marked out with a set of standardized macros.  From the user's point
of view tracing is activated by a command line program,
\texttt{dtrace}, but any program that is compiled with the OpenDTrace
libraries may initiate tracing, so long as it has sufficient
privileges.

The OpenDTrace privilege model is relatively simple, any program that
wishes to trace another program must be running with \emph{root}
privileges.  Some operating systems, such as Illumos, provide a more
nuanced privilege model, the details of which are discussed further in
Section~\ref{sec:privilege}.

Tracepoints are collected into one of many \emph{providers} which dictate
the capabilites of the tracepoint and how it interacts with the overall
tracing system.  Providers exist for system calls (\texttt{syscall}),
function boundary tracing (\texttt{fbt}), timing services (\texttt{profile}),
as well as specific subsystems such as the network (\texttt{ip}, \texttt{tcp}),
filesystem (\texttt{vfs}) and process scheduler (\texttt{proc}).
Arbitrary trace points can be added to the kernel via the
statically defined trace point (\texttt{sdt}) provider.  User space programs
are traced either with the \texttt{pid} provider or using the 
statically defined trace point (\texttt{usdt}) provider.  A complete
list of providers is given in Appendix~\ref{chap:opendtrace-providers}.

\section{Version History}

\begin{description}
\item[0.1] This is the first version of the \textit{OpenDTrace Formats
  Specification}, made available for early review and collaborative
  development.
\end{description}

\section{Document Structure}

This report specifies a number of aspects of OpenDTrace's operation:

\begin{description}

\item[The Operational Model] described in
  Chapter~\ref{chap:opendtrace-operation} gives a general overview of
  the internals of the OpenDTrace system, including the privilege
  model, trace-point format and a description of how user space
  programs are traced.

\item[The D Langauge] described in
  Chaper~\ref{chap:opendtrace-dlang} provides a full description of
  the D language, which is the domain specific scripting language used
  to create more complex data queries and to perform data reduction
  after tracepoint data has been captured.

\item[The Compact Trace Format (CTF)] described in
  Chapter~\ref{chap:opendtrace-ctf} explains the data extracted from
  compiled object code that is used by OpenDTrace to create trace
  points and extract function arguments and types.

\item[The OpenDTrace Object Format (DOF)] described in
  Chapter~\ref{chap:opendtrace-object-format} is a file-like format linking
  together a set of sections describing OpenDTrace code, string constants, and
  other aspects of a complete compiled OpenDTrace script.

\item[The OpenDTrace Intermediate Format (DIF)] is the bytecode that the
  executable elements of OpenDTrace scripts are compiled to.
  This is a simple RISC-like instruction set with constrained execution
  properties (e.g., only forward branches).
  Chapter~\ref{chap:opendtrace-intermediate-format} describes the instruction
  format and common instruction semantics.

\item[DTrace Instructions] are the individual RISC instructions
  performing a variety of operations including register access, memory
  access, arithmetic operations, and calling various built-in
  subroutines available to scripts in execution.
  Chapter~\ref{chap:opendtrace-instruction-reference} enumerates the
  instructions, their arguments, and their semantics.

\item[Variable Records] describe the set of variables (local, global,
  or thread-local) operated on by a OpenDTrace script.
  Chapter~\ref{chap:opendtrace-variable-records} specifies this record format.

\item[Built-in Global Variables] are a set of implementation-defined
  variables always available to scripts.
  This includes DTrace state (such as the current probe ID) and state from the
  instrumented probe context (e.g., the current process ID).
  Chapter~\ref{chap:opendtrace-global-vars} specifies these variables.

\item[Built-in Subroutines] are available to scripts, providing access to
  higher-level behavior, such as memory copying, string comparison, and so on.
  Chapter~\ref{chap:opendtrace-subroutines} describes the available built-in
  subroutines.

\item[Code Organization] for the DTrace implementation varies by operating
  system.
  Appendix~\ref{chap:opendtrace-code} describes the high-level layout of the
  DTrace code in several operating systems incorporating DTrace support.

\item[XXX] There is also an appendix on Providers that needs a description
  and reference here.

\end{description}
