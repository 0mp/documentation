The components that make up OpenDTrace interact with each other to
implement an operational model for dynamic tracing.  At the highest
level there are three components to OpenDTrace: tools, such as
\texttt{ctfconvert} which take compiled object code and generate new
ELF/DWARF sections that capture type information, the kernel module,
which is responsible for adding and removing trace points at run time,
and the libraries, which tie it all together.

\section{Privilege Model}
\label{sec:privilege}

\section{Tracepoint Format}
\label{sec:tracepoint-format}

\section{User Space Tracing}
\label{sec:user-space}

%%% Local Variables:
%%% mode: latex
%%% TeX-master: "dtrace-specification"
%%% End:
