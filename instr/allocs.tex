\clearpage
\phantomsection
\addcontentsline{toc}{subsection}{ALLOCS}
\label{insn:allocs}
\subsection*{ALLOCS: allocate a string}

\subsubsection*{Format}

\textrm{ALLOCS \%rd, \%r1}

\begin{center}
\begin{bytefield}[endianness=big,bitformatting=\scriptsize]{32}
\bitheader{0,7,8,15,16,23,24,31} \\
\bitbox{8}{0x3A}
\bitbox{8}{r1}
\bitbox{8}{r2}
\bitbox{8}{rd}
\end{bytefield}
\end{center}

\subsubsection*{Description}

The \instruction{allocs} instruction allocates a string in the DIF
scratch space, based on the size in \registerop{r1} and returns the
pointer to that string in register \registerop{rd}.  A failed
allocation returns a 0.

\subsubsection*{Pseudocode}

\begin{verbatim}
ptr = scratch_space;
scratch_space += size;
%rd = ptr
\end{verbatim}

\subsubsection*{Constraints}

\subsubsection*{Failure modes}

This instruction has no run-time failure modes beyond its constraints.
