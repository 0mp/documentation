\clearpage
\phantomsection
\addcontentsline{toc}{subsection}{COPYS}
\label{insn:copys}
\subsection*{COPYS: copy a string}

\subsubsection*{Format}

\textrm{COPYS \%rd, \%r1, \%r2}

\begin{center}
\begin{bytefield}[endianness=big,bitformatting=\scriptsize]{32}
\bitheader{0,7,8,15,16,23,24,31} \\
\bitbox{8}{0x3B}
\bitbox{8}{r1}
\bitbox{8}{r2}
\bitbox{8}{rd}
\end{bytefield}
\end{center}

\subsubsection*{Description}

The \instruction{copys} instruction copies bytes from the string
pointed to by \registerop{r1} and returns them in \registerop{rd}
bounded by a size placed into \registerop{r2}.

\subsubsection*{Pseudocode}

\begin{verbatim}
%rd = copy(r1, r2)
\end{verbatim}

\subsubsection*{Constraints}

\subsubsection*{Failure modes}

This instruction has no run-time failure modes beyond its constraints.
