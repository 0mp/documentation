\clearpage
\phantomsection
\addcontentsline{toc}{subsection}{AND}
\label{insn:and}
\subsection*{AND: Bitwise And}

\subsubsection*{Format}

\textrm{AND \%rd, \%r1, \%r2}

\begin{center}
\begin{bytefield}[endianness=big,bitformatting=\scriptsize]{32}
\bitheader{0,7,8,15,16,23,24,31} \\
\bitbox{8}{0x03}
\bitbox{8}{r1}
\bitbox{8}{r2}
\bitbox{8}{rd}
\end{bytefield}
\end{center}

\subsubsection*{Description}

This instruction calculates the bitwise and of the values found in registers
\registerop{r1} and \registerop{r2}, placing the results in register
\registerop{rd}.

\subsubsection*{Pseudocode}

\begin{verbatim}
regs[rd] = regs[r1] & regs[r2]
\end{verbatim}

\subsubsection*{Constraints}

\registerop{r1}, \registerop{r2}, and \registerop{rd} must be less than
\nregs{}.

\medskip
\noindent
\registerop{rd} cannot be \register{r0}.

\subsubsection*{Failure modes}

This instruction has no run-time failure modes beyond its constraints.
