\clearpage
\phantomsection
\addcontentsline{toc}{subsection}{MUL}
\label{insn:mul}
\subsection*{MUL: multiply two numbers}

\subsubsection*{Format}

\textrm{MUL \%rd, \%r1, \%r2}

\begin{center}
\begin{bytefield}[endianness=big,bitformatting=\scriptsize]{32}
\bitheader{0,7,8,15,16,23,24,31} \\
\bitbox{8}{0x08}
\bitbox{8}{r1}
\bitbox{8}{r2}
\bitbox{8}{rd}
\end{bytefield}
\end{center}

\subsubsection*{Description}

The \instruction{mul} instruction multiplies two numbers, contained in
\registerop{r1} and \registerop{r2}, together and places the result in
\registerop{rd}.

\subsubsection*{Pseudocode}

\begin{verbatim}
regs[rd] = regs[r1] * regs[r2]
\end{verbatim}

\subsubsection*{Constraints}

\subsubsection*{Failure modes}

This instruction has no run-time failure modes beyond its constraints.
