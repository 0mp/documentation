\clearpage
\phantomsection
\addcontentsline{toc}{subsection}{CMP}
\label{insn:dummy}
\subsection*{CMP: compare two values}

\subsubsection*{Format}

\textrm{CMP \%rd, \%r1, \%r2}

\begin{center}
\begin{bytefield}[endianness=big,bitformatting=\scriptsize]{32}
\bitheader{0,7,8,15,16,23,24,31} \\
\bitbox{8}{0x01}
\bitbox{8}{r1}
\bitbox{8}{r2}
\bitbox{8}{rd}
\end{bytefield}
\end{center}

\subsubsection*{Description}

The \instruction{cmp} instruction compares the values in
\registerop{r1} and \registerop{r2}, via subtraction, and sets the
various comparison bits based on the results.

\subsubsection*{Pseudocode}

\begin{verbatim}
%rd = %r1 | %r2
\end{verbatim}

\subsubsection*{Constraints}

\subsubsection*{Failure modes}

This instruction has no run-time failure modes beyond its constraints.
