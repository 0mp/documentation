\clearpage
\phantomsection
\addcontentsline{toc}{subsection}{SCMP}
\label{insn:scmp}
\subsection*{SCMP: compare two strings}

\subsubsection*{Format}

\textrm{SCMP \%r1, \%r2}

\begin{center}
\begin{bytefield}[endianness=big,bitformatting=\scriptsize]{32}
\bitheader{0,7,8,15,16,23,24,31} \\
\bitbox{8}{0x27}
\bitbox{8}{r1}
\bitbox{8}{r2}
\bitbox{8}{rd}
\end{bytefield}
\end{center}

\subsubsection*{Description}

The \instruction{scmp} intruction compares the strings pointed to by
\registerop{r1} and \registerop{r2} and sets the comparison bits for
the DIF interpreter based on the result.  The lenght of the comparison
is bounded by the \verb|DTRACEOPT_STRSIZE| option set for the system.

\subsubsection*{Pseudocode}

\begin{verbatim}
cc_r = strncmp(r1, r2, size);

cc_n = cc_r < 0;
cc_z = cc_r == 0;
cc_v = cc_c = 0;
\end{verbatim}

\subsubsection*{Constraints}

\subsubsection*{Failure modes}

This instruction has no run-time failure modes beyond its constraints.
