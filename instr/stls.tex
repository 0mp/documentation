\clearpage
\phantomsection
\addcontentsline{toc}{subsection}{STLS}
\label{insn:stls}
\subsection*{STLS: store a value in a local variable}

\subsubsection*{Format}

\textrm{STLS variable, \%rd}

\begin{center}
\begin{bytefield}[endianness=big,bitformatting=\scriptsize]{32}
\bitheader{0,7,8,23,24,31} \\
\bitbox{8}{0x49}
\bitbox{16}{variable}
\bitbox{8}{rd}
\end{bytefield}
\end{center}

\subsubsection*{Description}

The \instruction{stls} instruction takes a value from the
\registerop{rd} register and stores it in a variable.  

\subsubsection*{Pseudocode}

\begin{verbatim}
var = %rd
\end{verbatim}

\subsubsection*{Constraints}

\subsubsection*{Failure modes}

This instruction has no run-time failure modes beyond its constraints.
