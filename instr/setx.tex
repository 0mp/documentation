\clearpage
\phantomsection
\addcontentsline{toc}{subsection}{SETX}
\label{insn:setx}
\subsection*{SETX: retrieve an integer from the integer table}

\subsubsection*{Format}

\textrm{SETX intindex, \%rd}

\begin{center}
\begin{bytefield}[endianness=big,bitformatting=\scriptsize]{32}
\bitheader{0,7,8,15,16,31} \\
\bitbox{8}{0x25}
\bitbox{8}{rd}
\bitbox{16}{index}
\end{bytefield}
\end{center}

\subsubsection*{Description}

The \instruction{setx} instruction looks up an integer value stored in
the DIF integer table and places it into \registerop{rd}.
\subsubsection*{Pseudocode}

\begin{verbatim}
%rd = inttab[index]
\end{verbatim}

\subsubsection*{Constraints}

\subsubsection*{Failure modes}

This instruction has no run-time failure modes beyond its constraints.
