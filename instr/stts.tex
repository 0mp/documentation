\clearpage
\phantomsection
\addcontentsline{toc}{subsection}{STTS}
\label{insn:stts}
\subsection*{STTS: Store a value into thread local storage}

\subsubsection*{Format}

\textrm{STTS \%rd, \%r1, \%r2}

\begin{center}
\begin{bytefield}[endianness=big,bitformatting=\scriptsize]{32}
\bitheader{0,7,8,23,24,31} \\
\bitbox{8}{0x2D}
\bitbox{8}{variable}
\bitbox{8}{rd}
\end{bytefield}
\end{center}

\subsubsection*{Description}

The \instruction{stts} instruction takes the value stored in
\registerop{rd} and stores it directly, or by reference into a thread
local variable.  The \verb|DIF_TF_BYREF| flag is used to determine the
appropriate lookup.

\subsubsection*{Pseudocode}

\begin{verbatim}
var = %rd
\end{verbatim}

\subsubsection*{Constraints}

\subsubsection*{Failure modes}

This instruction has no run-time failure modes beyond its constraints.
