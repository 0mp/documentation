\clearpage
\phantomsection
\addcontentsline{toc}{subsection}{RLDX}
\label{insn:rldx}
\subsection*{RLDX: restricted load of a 64 bit quantity }

\subsubsection*{Format}

\textrm{RLDX \%rd, \%r1, \%r2}

\begin{center}
\begin{bytefield}[endianness=big,bitformatting=\scriptsize]{32}
\bitheader{0,7,8,15,16,23,24,31} \\
\bitbox{8}{0x4D}
\bitbox{8}{r1}
\bitbox{8}{r2}
\bitbox{8}{rd}
\end{bytefield}
\end{center}

\subsubsection*{Description}

The \instruction{rldx} instruction performs a privilege check on the
memory it is about to read from before loading a 64 bit quantity into
\registerop{rd}, indexed by \registerop{r1}.

\subsubsection*{Pseudocode}

\begin{verbatim}
%rd = mem[%r1]
\end{verbatim}

\subsubsection*{Constraints}

\subsubsection*{Failure modes}

This instruction has no run-time failure modes beyond its constraints.
