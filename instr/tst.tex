\clearpage
\phantomsection
\addcontentsline{toc}{subsection}{TST}
\label{insn:dummy}
\subsection*{TST: Test the value in r1}

\subsubsection*{Format}

\textrm{TST \%r1}

\begin{center}
\begin{bytefield}[endianness=big,bitformatting=\scriptsize]{32}
\bitheader{0,7,8,15,16,23,24,31} \\
\bitbox{8}{0x0F}
\bitbox{8}{r1}
\bitbox{8}{r2}
\bitbox{8}{rd}
\end{bytefield}
\end{center}

\subsubsection*{Description}

The \instruction{tst} instruction checks the value in \registerop{r1}
to see if it is zero (0).  Only the Z bit (cc\_z) is set by this
instruction, all other comparison result registers, listed in
Table\ref{tbl:cmp-vars} are cleared.

\subsubsection*{Pseudocode}

\begin{verbatim}
cc_n = cc_v = cc_c = 0;
cc_z = regs[r1] == 0;
\end{verbatim}

\subsubsection*{Constraints}

\subsubsection*{Failure modes}

This instruction has no run-time failure modes beyond its constraints.
