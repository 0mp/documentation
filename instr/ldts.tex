\clearpage
\phantomsection
\addcontentsline{toc}{subsection}{LDTS}
\label{insn:ldts}
\subsection*{LDTS: load a value from a thread local variable}

\subsubsection*{Format}

\textrm{LDTS \%rd, \%r1, \%r2}

\begin{center}
\begin{bytefield}[endianness=big,bitformatting=\scriptsize]{32}
\bitheader{0,7,8,23,24,31} \\
\bitbox{8}{0x2C}
\bitbox{16}{variable}
\bitbox{8}{rd}
\end{bytefield}
\end{center}

\subsubsection*{Description}

The \instruction{ldts} instruction loads data from a thread local
variable into the \registerop{rd} register by reference or by value.
The \verb|DIF_TF_BYREF| flag is used to determine the appropriate lookup.

\subsubsection*{Pseudocode}

\begin{verbatim}
%rd = var
\end{verbatim}

\subsubsection*{Constraints}

\subsubsection*{Failure modes}

This instruction has no run-time failure modes beyond its constraints.
