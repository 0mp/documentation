DTrace was originally developed on \pathname{OpenSolaris} which had a
unique way of organizing code.  One key thing to note is that there
are \em{two different} \pathname{dtrace.h} include files, one for the
kernel and one for the user space code.

\section{Illumos}
\label{sec:illumos-code}

The original source of DTrace came from OpenSolaris which has morphed
into Illumos.  As this was the original place that the code resided
there was no reason to split things along OS or license boundaries.
The main DTrace command resides in \pathname{cmd}, the supporting
libraries are in \pathname{lib/libdtrace} and the kernel code is in
the \pathname{uts/common}, \pathname{uts/intel}, \pathname{uts/sparc},
and related directories.

\section{FreeBSD}
\label{sec:freebsd-code}

Within FreeBSD the DTrace code has been split between that which came
from Sun's OpenSolaris (now Illumos) and is therefore under the CDDL
and the code which has been written natively on FreeBSD, and is
therefore under a BSD license.  There are two locations for the cddl
code, one in the root of the tree, \pathname{/usr/src} and one in the
kernel directory \pathname{/usr/src/sys}.  Native FreeBSD scripts are
located in the \pathname{/usr/share/dtrace} directory.

Because of the user space and kernel split for the cddl code the
FreeBSD tree has three, separate, \pathname{dtrace.h} files:

\begin{figure}
  \centering
\begin{description}
\item [sys/cddl/contrib/opensolaris/uts/common/sys/dtrace.h] The one
  you care about.
\item [cddl/contrib/opensolaris/lib/libdtrace/common/dtrace.h] Library APIs
\item [cddl/compat/opensolaris/include/dtrace.h] Compatability include
\end{description}
  \caption{The various versions of dtrace.h}
  \label{fig:freebsd-dtrace-h}
\end{figure}

\section{Mac OS}
\label{sec:macos-code}

Open source code from Apple is supplied in discrete pacakges.  The
DTrace code on Mac OS is split between the \pathname{xnu} kernel and
the rest of the code which is contained in a \pathname{dtrace} code
drop.  The kernel includes a very small number of files that are
absolutely necessary to build the kernel itself, including the driver
code.  All of the kernle code is collected into the
\pathname{xnu/bsd/dev/dtrace/} directory with the Mac OS translators,
the D files that know about the internals of kernel data structures,
are contained in the \pathname{scripts} sub-directory.  In the
\pathname{OpenDTrace} repositories there Mac OS kernel code resides in
\url{https://github.com/opendtrace/xnu} while the rest of the code
resides in \url{https://github.com/opendtrace/macos-dtrace}.  These
repositories are updated as soon as Apple drops their tarballs onto
\url{https://opensource.apple.com/tarballs/}.

%%% Local Variables:
%%% mode: latex
%%% TeX-master: "dtrace-specification"
%%% End:
