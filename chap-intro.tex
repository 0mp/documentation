DTrace is a dynamic tracing facility integrated into the Solaris, FreeBSD,
and Mac OS X operating systems -- with ports also available for Linux and
Windows.
Dynamic tracing allows system administrators and software developers to
develop short scripts (in the D programming language) that instruct DTrace to
instrument aspects of system operation, gather data, and present it for human
interpretation or mechanical processing.
While there is excellent documentation available for the D programming
language, command-line tools, and general DTrace-based investigation and
operation, the internal formats to DTrace are generally documented via the
source code.
This report acts as a de facto specification for those formats, including
the DTrace Intermediate Format (DIF), which is a bytecode that D scripts are
compiled into for safe execution within the kernel, and the DTrace Object
Format (DOF), which bundles together complete scripts along with their
associated constants and metadata.

%\section{Motivation}
\section{Background}

\textbf{XXRW: Ideally, there would be a more detailed description of both the
usage model, and also the architectural elements, of DTrace here.
That and some citations to DTrace documentation, the FreeBSD/Solaris books,
etc.}

\section{Version History}

\begin{description}
\item[0.1] This is the first version of the \textit{DTrace Formats
  Specification}, made available for early review and collaborative
  development.
\end{description}

\section{Document Structure}

This report specifies a number of aspects of DTrace's operation:

\begin{description}
\item[The DTrace Object Format (DOF)] described in
  Chapter~\ref{chap:dtrace-object-format} is a file-like format linking
  together a set of sections describing DTrace code, string constants, and
  other aspects of a complete compiled DTrace script.

\item[The DTrace Intermediate Format (DIF)] is the bytecode that the
  executable elements of DTrace scripts are compiled to.
  This is a simple RISC-like instruction set with constrained execution
  properties (e.g., only forward branches).
  Chapter~\ref{chap:dtrace-intermediate-format} describes the instruction
  format and common instruction semantics.

\item[DTrace Instructions] are the individual RISC instructions performing
  a variety of operations including register access, memory access, arithmetic
  operations, and calling various built-in subroutines available to scripts in
  execution.
  Chapter~\ref{chap:dtrace-instruction-reference} enumerates the instructions,
  their arguments, and their semantics.

\item[DTrace Variable Records] describe the set of variables (local, global,
  or thread-local) operated on by a DTrace script.
  Chapter~\ref{chap:dtrace-variable-records} specifies this record format.

\item[DTrace Subroutines] are available to scripts, providing access to
  higher-level behavior, such as memory copying, string comparison, and so on.
  Chapter~\ref{chap:dtrace-subroutines} describes the available built-in
  subroutines.

\end{description}
