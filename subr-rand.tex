\clearpage
\phantomsection
\addcontentsline{toc}{subsection}{RAND}
\label{insn:and}
\subsection*{RAND: Get Random}

\subsubsection*{Calling convention}

\begin{description}
\item[\registerop{rd}] Target for 64 bits of random(ish) data
\end{description}

\subsubsection*{Description}

This subroutine returns 64 bits of random(ish) data, placing the result in
\registerop{rd}.
On supporting systems, stronger randomness can be obtained uing the
\hyperref[subr:random]{\subroutine{random}} subroutine.

\subsubsection*{Pseudocode}

\begin{verbatim}
regs[rd] = (getthrtime() * 2416 + 374441) % 1771875
\end{verbatim}

\subsubsection*{Constraints}

\subsubsection*{Failure modes}

This subroutine has no run-time failure modes beyond its constraints.
